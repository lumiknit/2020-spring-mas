\section*{Problem 3}

Let a power series \(f = \sum_{n=1}^{\infty} a_n z^n\) be uniformly convergent on any compact subset of \(\bbC\). Let \(C\) be a close curve in \(\bbC\).
\begin{enumerate}[label=(\arabic*)]
\item Prove that for any nonnegative integer \(n\), \(\int_{C} z^n dz = 0\).
\item Using the fact in (1), prove that \(\int_C f(z) dz = 0\).
\end{enumerate}

\subsection*{Proof of (1)}

Let \(n \in \bbZ^{\ge 0}\).
Since \(z^n\) is an analytic polynomial, it's analytic everywhere in \(\bbC\) (i.e. entire).
Thus, there are antiderivatives of \(z^n\).
For example, \(\frac{1}{n + 1}z^{n + 1}\) is an entire function (\(\because\) it's a analytic polynomial) and the derivative is \(z^n\).
Let \(\sigma: [0, 1] \to \bbC\) be a function of the closed curve \(C\).
Then, \(\sigma(0) = \sigma(1)\) by closedness of \(C\) and,
\begin{align*}
\int_C z^n dz
&= \int_{0}^{1} \sigma(t)^n \sigma'(t) dt
\\&= \left[ \frac{1}{n + 1} \sigma(t)^{n + 1} \right]_{t=0}^{1}
= \frac{1}{n + 1} (\sigma(1)^{n + 1} - \sigma(0)^{n + 1})
= \frac{1}{n + 1} (\sigma(0)^{n + 1} - \sigma(0)^{n + 1})
= 0
\end{align*}
by Proposition 4.12.
\qedsq

\subsection*{Proof of (2)}

Let \(f_m(z) = \sum_{n=1}^{m} a_n z^n\).
Because \(C\) is a closed curve in \(\bbC\), there is a compact subset \(D \subseteq \bbC\) such that
\(C \subseteq D \subsetneq \bbC\).
(e.g. \(D = C\). \(\because\) \(C\) is a closed curve implies that there is a continuous function
\(\sigma: [0, 1] \to \bbC\) such that \(C = \sigma([0, 1])\).
Since \([0, 1]\) is closed, bounded, and compact, the continuous image \(C\) of \(\sigma\) is also compact.)
By the assumption in the problem, \(f_m \to f\) uniformly in the compact domain \(D\).
Since each \(f_m\) is continuous (\(\because\) it's a polynomial)
and \(f_m \to f\) uniformly in the compact set \(D\), by Proposition 4.11,
\[
\int_C f(z) dz =
\lim_{m \to \infty} \int_C f_m(z) dz
\]
holds.
Note that,
\begin{align*}
\int_C f_m(z) dz
&= \int_C \sum_{n=1}^{m} a_n z^m dz
\\&= \sum_{n=1}^{m} a_n \int_C z^m dz
= \sum_{n=1}^{m} a_n \cdot 0 = 0
\end{align*}
because we can reorder finite sum and integration, and because of the fact we proved in (1).
Therefore,
\begin{align*}
\int_C f(z) dz
&= \lim_{m \to \infty} \int_C f_m(z) dz
\\&= \lim_{m \to \infty} 0 = 0
\end{align*}
\qedsq