\section*{Problem 8}

Find all analytic functions \(f\) on the closed unit disc in \(\bbC\) such that
\(|f(z)| \le 1\) for \(|z| \le 1\),
\(|f(z)| = 1\) for \(|z| = 1\),
and \(f\) is injective.

\subsection*{Answer}
\[\left\{z \mapsto \frac{Kz + \alpha}{1 + \overline{\alpha}Kz}
  \mid
  \alpha \in \bbD;
  K \in \bbC;
  |K| = 1
  \right\}\]

\subsection*{Proof}

I'll denote \(\bbD = D(0; 1)\).
\newline

\noindent
Abstract:
First, I'll show that the function of \(f\) restricted by \(\bbD\) is bijective and it's inverse is also analytic in \(\bbD\).
Then, using Blaschke Factor to make bijective analytic function \(g\) which satisfies every properties given in problem and \(g(0) = 0\).
Next, applying Schwarz's Lemma to \(g\) and \(g^{-1}\) and get \(g(z) = Kz\) for some \(K \in \bbC\) such that \(|K| = 1\).
Finally, transform \(g(z)\) to \(f(z)\).
\newline

\noindent
Before the proof, note that we learned about the Blaschke Factor:
The Blaschke Factor for \(\alpha \in \bbD\) is:
\[B_\alpha(z) = \frac{z - \alpha}{1 - \overline{\alpha}z}\]
As we proved in the lecture, it has below properties:
\begin{itemize}
\item \(B_\alpha\) is analytic in \(\bbD\).
\item \(B_\alpha\) maps \(\bbD \mapsto \bbD\), \(\partial\bbD \mapsto \partial\bbD\), \(0 \mapsto -\alpha\), and \(\alpha \mapsto 0\).
\item \(B_\alpha\) is bijective with its inverse \(B_\alpha^{-1} = B_{-\alpha}\)
\end{itemize}
Also, \(B_\alpha\) is continuous in \(\overline{\bbD}\) because for \(z \in \overline{\bbD}\), the denominator of \(B_\alpha(z)\), \(1 - \overline{\alpha}z\) cannot be zero since \(|\alpha| < 1\) and denominator and numerator of \(B_\alpha\) are continuous (as linear polynomial).
\newline

\noindent
Suppose that \(f\) is analytic in \(\overline{\bbD}\), \(\forall z \in \overline{\bbD}: |f(z)| \le 1\), \(\forall z \in \partial\bbD: |f(z)| = 1\) and \(f\) is injective.
\newline

First of all, note that \(f\) cannot be constant.
Because, if \(f\) is a constant function, \(f(1) = f(-1)\) and it violates injectivity of \(f\).
\newline

Also, note that \(f\) maps all elements of \(\bbD\) into \(\bbD\).
Because, by Maximum-Modulus Theorem, every interier point of \(\overline{\bbD}\)
cannot has maximum modulus of \(f\).
Since \(f\) has maximum modulus, 1, at every points of \(\partial\bbD\),
\(|f(z)| < 1\) for each element of \(z \in \bbD\).
\newline

Let \(h = \left. f \right|_{\bbD}\), which is the function obtained from \(f\) by restricting the domain to \(\bbD\).
We will show that \(h\) is surjective.

Let \(t\) be an arbitrary element of \(\bbD\).
Then, we can take the Blaschke Factor of \(t\).
Let \(F = B_t \circ f\).
Then, \(F(t) = 0\).
Because \(\overline{\bbD}\) is non-empty compact
and \(|F|\) is continuous (\(\because\) \(|\cdot|\), \(f\) and \(B_t\) are continuous in \(\overline{\bbD}\)),
\(S = |F|(\overline{\bbD}) = \{|F(z)| \mid z \in \overline{\bbD}\}\) is non-empty compact.
Since \(S\) is a non-empty compact set of real values,
\(S\) contains a minimum value.
Let \(v \in \overline{\bbD}\) and \(|F(v)|\) is minimum.

Suppose that \(v \in \partial\bbD\).
\(|f(v)| = 1\) because of the property of \(f\),
and \(|F(v)| = |B_t(f(v))| = 1\) because of the property of \(B_t\).
Thus, \(|F(v)| = 1\) is the minimum value of \(S\).
Since the image of \(F\) is a subset of \(\overline{\bbD}\),
\(s \le 1\) for every \(s \in S\).
Then, \(1 = |F(v)| \le s \le 1\) and \(s = 1\) for every \(s \in S\).
In other words, for every \(z \in \overline{\bbD}\), \(|F(z)| = 1\).
But it's a contradiction
because \(0 \in \bbD\),
but \(|F(0)| = |B_t(f(0))| = 1\) implies \(f(0) \in \partial\bbD\) and \(0 \in \partial\bbD\).

Thus, \(v \in \bbD\).
But in this case, \(v\) is a minimum point of the open connected set \(\bbD\)
for the analytic function \(F\).
By Minimum-Modulus Theorem, \(|F(v)| = 0\) and \(F(v) = 0\).
Because of the property of \(B_t\), \(F(v) = B_t(f(v)) = 0\) means \(f(v) = t\).
Therefore, for every \(t \in \bbD\), there is \(v \in \bbD\) such that \(f(v) = t\).
And this shows that \(h\), obtained from \(f\) by restricting the domain to \(\bbD\), is surjective.
\newline

Because \(h\) is injective (\(\because\) \(f\) is injective) and surjective,
there is an inverse function \(h^{-1}\).
What we next to check is \(h^{-1}\) is analytic in \(\bbD\).

As Proposition 3.5, if \(h\) is differentiable at \(z_1 = h^{-1}(z_0)\)
and \(h'(z_1) \neq 0\), then \(h^{-1}\) is differentiable at \(z_0\).
Since \(h\) is analytic in \(\bbD\) (\(\because\) \(f\) is analytic in \(\overline{\bbD}\)), we know that \(h'(z_0)\) exists for every \(z_0 \in \bbD\).
Suppose that \(Z = \{z \in \bbD \mid h'(z) = 0\}\) is non-empty.
Let \(z_0\) be any element of \(Z\).
If \((D(z_0; r) \cap Z) \bs \{z_0\} \neq \emptyset\) for every \(r > 0\), it contradicts.
(\(\because\) If \((D(z_0; r) \cap Z) \bs \{z_0\} \neq \emptyset\) for every \(r > 0\),
 \(z_0\) is a limit point of \(Z\).
 Then, since \(Z\) is a zero of \(h'\), by Uniqueness Theorem, \(h' \equiv 0\) in \(\bbD\).
 Then, \(h\) should be constant function in \(\bbD\),
 but it's a contradiction because \(h(1/42) = h(0)\) violates the injectivity of \(h\).)
Therefore, there is \(r > 0\) such that \(D(z_0; r) \cap Z = \{z_0\}\).
In other words, for \(z \in D(z_0; r) \bs \{z_0\}\), \(h'(z) \neq 0\).
Since \(h\) is analytic in \(\bbD\), by Open Mapping Theorem,
\(T = h(D(z_0; r))\) is also open.
Since \(h(z_0)\) is an interier point of the open set \(T\),
take \(s > 0\) such that \(D(h(z_0); s) \subseteq T\).
Because \(h'\) is non-zero for every points of \(D(z_0; r) \bs \{z_0\}\),
\(h^{-1}\) is differentiable at every points of \(T \bs \{h(z_0)\}\),
and \(h^{-1}\) is differentiable at every points of \(D(h(z_0); s) \bs \{h(z_0)\}\).
Let \(L = \{x + iy \mid y = \text{Im } h(z_0), x = [-\frac{s}{2} + \text{Re } h(z_0), \frac{s}{2} + \text{Re } h(z_0)]\}\).
Since \(h^{-1}\) is continuous in \(\bbD\)
(\(\because\) \(h\) is an open map because \(h\) is analytic in \(\bbD\) and because of Open Mapping Theorem. Since \(h\) is an inverse map of \(h^{-1}\), inverse map of \(h^{-1}\) is an open map in \(\bbD\). Thus, \((h^{-1})^{-1}\) maps open sets to open sets and it implies \(h^{-1}\) is continuous in \(\bbD\).)
and analytic in \(D(h(z_0), s)\) except the line segment \(L\)
(\(\because\) \(h^{-1}\) is differentiable in the open set \(D(h(z_0), s) \bs L\)),
by Theorem 7.7, \(h^{-1}\) is analytic throughout \(D(h(z_0), s)\).
Then, since \(h(z_0) \in D(h(z_0), s)\), \(h^{-1}\) is differentiable at \(h(z_0)\).
However, it's a contradiction because if \(h^{-1}\) is differentiable at \(h(z_0)\), since \(h\) is differentiable at \(z_0\) and,
\[0 = (h^{-1})'(h(z_0)) \cdot 0 = (h^{-1})'(h(z_0))h'(z_0) \neq
  (h^{-1} \circ h)'(z_0) = \text{Id}_{\bbD}'(z_0) = 1\]
It violates Chain Rule.
Thus, \(Z\) should be empty.
And \(h'\) is nowhere zero and \(h^{-1}\) is analytic in \(\bbD\).
\newline

So to sum up at this point, for \(h = \left. f \right|_{\bbD}\), \(h: \bbD \to \bbD\) is bijective, continuous and analytic in \(\bbD\) and so is \(h^{-1}: \bbD \to \bbD\).
\newline

\noindent
Next, we will show that \(B_q \circ h\) is a linear function for some \(q \in \bbD\).
\newline

Let \(\alpha = h(0)\).
Then, let \(g: \bbD \to \bbD\) such that \(g = B_\alpha \circ h\).
Because of the properties of \(h\) and \(B_\alpha\),
\begin{itemize}
\item \(g\) is analytic in \(\bbD\).
\item \(g\) is continuous in \(\bbD\).
\item \(g\) is bijective in \(\bbD\).
\item \(g(0) = B_\alpha(h(0)) = B_\alpha(\alpha) = 0\).
\item \(g\) maps every element of \(\bbD\) into \(\bbD\).
  It's because \(B_\alpha\) and \(h\) are automorphisms of \(\bbD\).
  Thus, \(|g(z)| < 1\).
\item \(g'\) is nowhere zero in \(\bbD\),
  and \(g^{-1}\) is continuous and analytic in \(\bbD\).
  Because \(g\) is a bijective function in \(\bbD \to \bbD\),
  we can show this continuity and differentiability in \(\bbD\)
  in exactly same way to show that \(h^{-1}\) is analytic in \(\bbD\) above.
\item \(g^{-1}(0) = 0\) and \(g^{-1}(z) < 1\) for every \(z \in \bbD\).
  It's because \(g\) is an automorphism of \(\bbD\) and because of above properties.
\end{itemize}

Since \(g\) and \(g^{-1}\) are analytic in the unit disc \(\bbD\), \(g \ll 1\), \(g^{-1} \ll 1\), \(g(0) = g^{-1}(0) = 0\), we can apply Schwarz's Lemma to \(g\) and \(g^{-1}\).

By applying Schwarz's Lemma to \(g^{-1}\), we obtain \(\left|(g^{-1})'(0)\right| \le 1\) and,
\begin{gather*}
  \frac{1}{\left|g'(0)\right|}
    = \left|(g^{-1})'(g(0))\right|
    = \left|(g^{-1})'(0)\right| \le 1 \\
  \left|g'(0)\right| \ge 1
\end{gather*}

And by applying Schwarz's Lemma to \(g\), we obtain
\[\left|g'(0)\right| \le 1\]

By combining above two inequalities, we obtain
\[\left|g'(0)\right| = 1\]
and by Schwarz's Lemma, there is some \(K \in \bbC\) such that \(|K| = 1\) and
\[g(z) = Kz\]

Because \(B_{-\alpha} = B_{\alpha}^{-1}\),
\begin{align*}
  h(z)
  &= B_{-\alpha}(B_\alpha(h(z)))
  \\&= B_{-\alpha}(g(z))
  \\&= B_{-\alpha}(Kz)
  \\&= \frac{Kz - (-\alpha)}{1 - (\overline{-\alpha})Kz}
  \\&= \frac{Kz + \alpha}{1 + \overline{\alpha}Kz} \;\;\cdots\text{(*)}
\end{align*}

Because \(\alpha = h(0) \in \bbD\) (i.e. \(|\alpha| < 1\)) and \(|K| = 1\),
\(|\overline{\alpha}Kz| < 1\) for \(z \in \overline{\bbD}\).
Thus, for every \(z \in \overline{\bbD}\), \(1 + \overline{\alpha}Kz \neq 0\).
Also, since the numerator and denominator of (*) are entire as linear polynomial, (*) is analytic in \(\overline{\bbD}\).
Since \(h(z) = f(z)\) for every \(z \in \bbD\) and \(h\) is analytic,
by Uniqueness Theorem,
\[f(z) = \frac{Kz + \alpha}{1 + \overline{\alpha}Kz}\]
for \(\alpha = h(0) = f(0)\), \(|K| = 1\).
\newline

Let \(K \in \bbC\) such that \(|K| = 1\) and \(\alpha \in \bbD\).
And let \(f(z) = B_{-\alpha}(Kz) = \frac{Kz + \alpha}{1 + \overline{\alpha}Kz}\).
First, as we shown above, the numerator and denominator of \(f(z)\) are entire as linear polynomial and \(1 + \overline{\alpha}Kz\) cannot be zero for every \(z \in \overline{\bbD}\) because \(|\alpha| < 1\).
Thus \(f(z)\) is entire.

And let \(g = B_\alpha \circ f\). Then,
\begin{align*}
  g(z)
  &= B_{\alpha}(f(z))
  \\&= B_{-\alpha}(B_{\alpha}(Kz))
  \\&= Kz
\end{align*}
Since \(Kz\) is a linear polynomial as \(|K| = 1 > 0\), \(Kz\) is bijective, and \(g(z)\) is bijective since \(B_\alpha, B_{-\alpha}, Kz\) are bijective.
For \(|z| \le 1\), \(|g(z)| = |K|\cdot |z| = 1 \cdot |z| = |z| \le 1\).
In other words, since \(B_{\alpha}\) is a bijection which maps \(\overline{\bbD}\) to \(\overline{\bbD}\),
\(|f(z)| \le 1\) for \(|z| \le 1\).
Also, for \(|z| = 1\), \(|g(z)| = |K|\cdot |z| = |z| = 1\),
and since \(B_{\alpha}\) is a bijection which maps \(\partial\bbD\) to \(\partial\bbD\), 
\(|f(z)| = 1\) for \(|z| = 1\).

Thus, the \(f(z) = \frac{Kz + \alpha}{1 + \overline{\alpha}Kz}\) satisfies all condition given in the problem.
\newline

\noindent
In conclusion. The set consisting of all analytic functions \(f\) on the closed unit disc in \(\bbC\) such that \(|f(z)| \le 1\) for \(|z| \le 1\), \(|f(z)| = 1\) for \(|z| = 1\), and \(f\) is injective is:
\[\left\{z \mapsto \frac{Kz + \alpha}{1 + \overline{\alpha}Kz}
  \mid
  \alpha \in \bbD;
  K \in \bbC;
  |K| = 1
  \right\}\]
\qedsq