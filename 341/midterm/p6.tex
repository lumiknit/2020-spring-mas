\section*{Problem 6}

Let \(f = u + iv\) be an entire function. Prove that
\[
\det
\begin{pmatrix}
u_{xx} & u_{xy} \\ u_{yx} & u_{yy}
\end{pmatrix}
=
\det
\begin{pmatrix}
v_{xx} & v_{xy} \\ v_{yx} & v_{yy}
\end{pmatrix}
\le 0
\]

\subsection*{Lemmata}

Note that if \(f = u + iv\) is entire, \(f_x = u_x + iv_x\), \(f_y = u_y + iv_y\) exists by Proposition 3.1.
\begin{lemma}\label{lem-6-exist-pd}
  Let \(f = u + iv\) be an entire function where \(u, v : \bbC \to \bbR\).
  \(f' = u_x + iv_x = v_y - iu_y\).
\end{lemma}
\begin{proof}
  Because \(f\) is analytic, \(f'\) is equal to every directional derivation.
  Thus,
  \begin{align*}
    f'(z)
    &= \lim_{h \to 0} \frac{f(z + h) - f(z)}{h}
    \\&= \lim_{h \in \bbR, h \to 0} \frac{f(z + h) - f(z)}{h}
    \\&= \lim_{h \in \bbR, h \to 0} \frac{u(z + h) - u(z) + i(v(z + h) - v(z))}{h}
    \\&= u_x(z) + iv_x(z)
  \end{align*}
  And by Cauchy-Riemann Equation, \(f' = u_x + iv_x = v_y - iu_y\).
\end{proof}
\begin{lemma}\label{lem-6-2nd-class}
  Let \(f = u + iv\) be an entire function where \(u, v : \bbC \to \bbR\).
  Then, at least \(u, v \in \calC^2(\bbC)\) (i.e. \(u, v\) are complex functions which have the second derivatives which are continuous).
\end{lemma}
\begin{proof}
  Since \(f\) is entire, \(f' = u_x + iv_x\).
  Also, derivative of entire function is entire.
  Thus, \(f'\) is entire.
  Then, \(f'' = u_{xx} + iv_{xx}\).
  In the similar way, there is \(f^{(3)}\) with \(f''_x\) and \(f''_y\).
  Because \(f''_x = u_{xxx} + iv_{xxx}\) and \(f''_y = u_{xxy} + iv_{xxy}\),
  \(u_{xxx}, v_{xxx}, u_{xxy}, v_{xxy}\) exists.
  It means \(u_{xx}\) and \(v_{xx}\) are differentiable and continuous.

  Because Cauchy-Riemann Equation holds as \(if'_x = f'_y\),
  we can take \(f'' = v_{xy} - iu_{xy}\) instead of \(u_{xx} + iv_{xx}\).
  Then, in the similar way, we can conclude that \(v_{xy}, u_{xy}\) are differentiable and continuous.
  
  Also, using Cauchy-Riemann Equation,
  for \(f' = v_y - iu_y\) instead of \(u_x + iv_x\),
  by following above processes, we can conclude that
  \(v_{yx}, u_{yx}, v_{yy}, u_{yy}\) are differentiable and continuous.

  Thus, \(u, v \in \calC^{2}(\bbC)\).
\end{proof}

\subsection*{Proof}

Note that,
\[
\det
\begin{pmatrix}
u_{xx} & u_{xy} \\ u_{yx} & u_{yy}
\end{pmatrix}
= u_{xx} u_{yy} - u_{xy} u_{yx},\ \ 
\det
\begin{pmatrix}
v_{xx} & v_{xy} \\ v_{yx} & v_{yy}
\end{pmatrix}
= v_{xx} v_{yy} - v_{xy} v_{yx}
\]

Since \(f\) is entire,
\(u_x = v_y\) and \(u_y = -v_x\) by Cauchy-Riemann Equation.
Thus,
\begin{gather*}
   u_{xx} = \frac{\partial u_x}{\partial x} 
   = \frac{\partial v_y}{\partial x} = v_{yx}
\\ u_{xy} = \frac{\partial u_x}{\partial y} 
   = \frac{\partial v_y}{\partial y} = v_{yy}
\\ u_{yx} = \frac{\partial u_y}{\partial x} 
   = \frac{\partial (-v_x)}{\partial x} = -v_{xx}
\\ u_{yy} = \frac{\partial u_y}{\partial y} 
   = \frac{\partial (-v_x)}{\partial y} = -v_{xy}
\end{gather*}

Therefore,
\begin{align*}
\det
\begin{pmatrix}
u_{xx} & u_{xy} \\ u_{yx} & u_{yy}
\end{pmatrix}
  = u_{xx} u_{yy} - u_{xy} u_{yx}
  &= v_{yx} (- v_{xy}) - v_{yy} (- v_{xx})
  \\&= -v_{yx}v_{xy} + v_{yy} v_{xx}
  = v_{xx}v_{yy} - v_{xy} v_{yx}
  =
  \det
\begin{pmatrix}
v_{xx} & v_{xy} \\ v_{yx} & v_{yy}
\end{pmatrix}
\end{align*}
holds.

Lemma \ref{lem-6-2nd-class} shows that \(u, v \in \calC^{2}(\bbC)\).
It means, we can change the order of partial derivatives like:
\[ u_{xy} = u_{yx}, v_{xy} = v_{yx} \]
Thus,
\begin{align*}
  u_{xx}u_{yy} - u_{xy}u_{yx}
  &= -v_{yx}v_{xy} - u_{xy}u_{yx}
  \\&= -v_{xy}^2 - u_{xy}^2
\end{align*}
Since \(u\) and \(v\) are real-valued functions, \(v_{xy}^2 \ge 0\) and \(u_{xy}^2 \ge 0\).
So \(-v_{xy}^2 - u_{xy}^2 \le 0\).

Therefore,
\begin{align*}
\det
\begin{pmatrix}
u_{xx} & u_{xy} \\ u_{yx} & u_{yy}
\end{pmatrix}
=
\det
\begin{pmatrix}
v_{xx} & v_{xy} \\ v_{yx} & v_{yy}
\end{pmatrix}
= u_{xx}u_{yy} - u_{xy}u_{yx}
&= -v_{xy}^2 - u_{xy}^2
\le 0
\end{align*}
\qedsq