\section*{Problem 2}

Find all entire functions \(f\) such that
\(f (n\pi) = 0\) for any \(n \in \bbZ\)
and \(\left|f(x+iy)\right| \le Ce^{\left|y\right|} < \infty\),
\(x, y \in \bbR\) for some \(C > 0\).

\subsection*{Answer}

\[\{z \mapsto K \sin z \mid K \in \bbC\}\]

\subsection*{Proof}

First of all, note that \(f = K\sin\) for \(K \in \bbC\) satisfies all conditions in the problem:
\begin{itemize}
\item \(\sin\) is entire. Thus \(K \sin\) is also entire.
\item \(K\sin(n\pi) = K \cdot 0 = 0\) for every \(n \in \bbZ\).
\item Let \(C = \max(1, 2 |K|) > 0\). Then, for \(x, y \in \bbR\),
  \begin{align*}
    |K \sin (x + iy)|
    &= |K| \cdot \left| \frac{e^{-y} e^{ix} - e^{y} e^{-ix}}{2i} \right|
    \\&\le |K| \cdot \left| e^{-y} e^{ix} - e^{y} e^{-ix} \right|
    \\&\le |K| \cdot (\left| e^{-y} e^{ix} \right| + \left| e^{y} e^{-ix} \right|)
    \\&\le |K| \cdot (\left| e^{-y} \right| + \left| e^{y} \right|)
    \\&= |K| (e^y + e^{-y})
    \le 2 |K| e^{|y|} \le C e^{|y|}
  \end{align*}
\end{itemize}

Suppose that \(f\) is some function satisfies all conditions given in the problem: 
\(f\) is entire, \(\forall n \in \bbZ: f(n\pi) = 0\), there is \(C > 0\), which satisfies \(\forall x, y \in \bbR: |f(x + iy)| \le Ce^{|y|} < \infty\).

First of all, let's consider
\[\frac{f(z)}{\sin z}\]
It's well-defined only for \(\bbC \bs \pi \bbZ\), where \(\sin z\) is non-zero.
(\(\pi \bbZ = \{\pi n \mid b \in \bbZ\}\))
We know that \(\sin z\) is expanded as the Taylor series:
\begin{align*}
  \sin z
  &= \sum_{k=0}^{\infty} (-1)^k \frac{z^{2k + 1}}{(2k + 1)!}
  \\&= z - \frac{z^3}{3!} + \frac{z^5}{5!} - \cdots
\end{align*}
And since \(\sin(z - n\pi) = (-1)^n\sin(z)\) for \(n \in \bbZ\)
(\(\because\)
  \(\sin(z - n\pi)
    = (e^{i(z - n\pi)} - e^{-i(z - n\pi)})/(2i)
    = (e^{iz}e^{-n\pi i} - e^{-iz}e^{n\pi i})/(2i)
    = ((-1)^n e^{iz} - (-1)^n e^{-iz})/(2i)
    = (-1)^n(e^{iz} - e^{-iz})/(2i)
    = (-1)^n \sin z\))
\begin{align*}
  \sin z
  &= (-1)^n \sin (z - n\pi)
  \\&= (-1)^n \sum_{k=0}^{\infty} (-1)^k \frac{(z - n\pi)^{2k + 1}}{(2k + 1)!}
  \\&= (-1)^n \left( (z - n\pi) - \frac{(z - n\pi)^3}{3!} + \frac{(z - n\pi)^5}{5!} - \cdots \right)
\end{align*}

Because \(f\) is entire, we can expand \(f\) as:
\[f(z) = \sum_{k=0}^{\infty} b_{n,k} (z - n\pi)^k\]
Since \(f(0\pi) = f(0) = 0\), \(b_{n, 0} = 0\). Thus,
\[f(z) = \sum_{k=1}^{\infty} b_{n,k} (z - n\pi)^k\]
and,

Note that for an entire function \(\phi\), for every \(a \in \bbC\),
\[\psi(z) = \left\{\begin{array}{ll}
  \frac{\phi(z) - \phi(a)}{z - a} & (z \neq a) \\
  \phi'(a) & (z = a)
\end{array}\right.\]
is entire by Proposition 5.8, continuous and converges at every points.
Thus,
\begin{align*}
  \lim_{z \to n\pi} \frac{f(z)}{\sin z}
  &= \lim_{z \to n\pi}
    \frac {\sum_{k=1}^{\infty} b_{n,k} (z - n\pi)^k}
    {(-1)^n \left( (z - n\pi) - \frac{(z - n\pi)^3}{3!} + \frac{(z - n\pi)^5}{5!} - \cdots \right)}
  \\&= \lim_{z \to n\pi}
    \frac {
      \frac{\sum_{k=1}^{\infty} b_{n,k} (z - n\pi)^k}{z - n\pi}}
    {(-1)^n \frac{ (z - n\pi) - \frac{(z - n\pi)^3}{3!} + \frac{(z - n\pi)^5}{5!} - \cdots }{z - n\pi}}
  \\&= 
  \lim_{z \to n\pi}
    (-1)^n
    \frac{\sum_{k=1}^{\infty} b_{n,k} (z - n\pi)^{k - 1}}
    { 1 - \frac{(z - n\pi)^2}{3!} + \frac{(z - n\pi)^4}{5!} - \cdots}
  \\&= 
    (-1)^n
    \frac{\lim_{z \to n\pi} \sum_{k=1}^{\infty} b_{n,k} (z - n\pi)^{k - 1}}
    {\lim_{z \to n\pi} \left( 1 - \frac{(z - n\pi)^2}{3!} + \frac{(z - n\pi)^4}{5!} - \cdots \right)}
  \\&= 
    (-1)^n
    \frac{\left. \sum_{k=1}^{\infty} b_{n,k} (z - n\pi)^{k - 1} \right|_{z = n\pi}}
    {\left. 1 - \frac{(z - n\pi)^2}{3!} + \frac{(z - n\pi)^4}{5!} - \cdots \right|_{z = n\pi}}
  = (-1)^n b_{n,1}
\end{align*}

Thus, if we take
\[g(z) = \left\{
  \begin{array}{ll}
    (-1)^n b_{n,1} & (z = n\pi \in \pi\bbZ) \\
    f(z) / \sin z & (z \in \bbC \bs \pi\bbZ)
  \end{array}
\right.\]
\(g\) is continuous.

Since \(f\), \(\sin\) are entire and \(\sin\) is non-zero at \(\bbC \bs \pi\bbZ\), \(g\) is analytic in \(\bbC \bs \pi\bbZ\).
For each \(n\pi \in \pi\bbZ\), we can take \(D(n\pi; 1/2)\),
and \(D(n\pi; 1/2) \cap \{k\pi\}_{k \in \bbZ} = \{n\pi\}\) holds.
Thus, \(g\) is analytic in \(D(n\pi; 1/2)\) exception \(n\pi\).
Let \(L = \{x + iy \mid y = 0, x \in [n\pi - 1/4, n\pi + 1/4]\}\).
Then, \(g\) is continuous in \(\bbC\), analytic in the open set \(D(n\pi; 1/2)\) except the line segment \(L\).
By Theorem 7.7, \(g\) is analytic in \(D(n\pi; 1/2)\).
This implies that \(g\) is differentiable in each \(n\pi \in \pi\bbZ\).
Therefore, \(g\) is differentiable in \((\bbC \bs \pi\bbZ) \cup (\pi\bbZ) = \bbC\),
and \(g\) is entire.

Let's show that \(g\) is bounded.

Let \(z = x + iy\) where \(x, y \in \bbR\),
\begin{align*}
  |\sin(z)| = 
  |\sin(x + iy)|
  &= \frac{1}{2} \left| e^{ix - y} - e^{-ix + y} \right|
  \\&= \frac{1}{2} \left| e^{-y} e^{ix} - e^{y} e^{-ix} \right|
  \\&\ge \frac{1}{2} \left| |e^{-y} e^{ix}| - |e^{y} e^{-ix}| \right|
  \\&\ge \frac{1}{2} \left| e^{-y} - e^{y} \right|
\end{align*}
and, for \(y \neq 0\),
\[\frac{1}{|\sin(z)|} \le \frac{2}{\left| e^{-y} - e^{y} \right|}\]

Suppose that \(|y| \ge 1\).
Note that \(e^{-2|y|} \le e^{-2} \simeq 0.135 < 0.5\).
\begin{align*}
  \frac{1}{|\sin(z)|}
  \le \frac{2}{\left| e^{-y} - e^{y} \right|}
  = \frac{2}{ e^{|y|} - e^{-|y|} }
  \le \frac{2}{e^{|y|} (1 - e^{-2|y|})}
  \le \frac{2}{e^{|y|} (1 - e^{-2})}
  \le \frac{2}{0.5 e^{|y|}} = 4e^{-|y|}
\end{align*}
And,
\begin{align*}
  |g(z)|
  &= \left| \frac{f(x + iy)}{\sin (x + iy)} \right|
  \\&= |f(x + iy)| \left| \frac{1}{\sin (x + iy)} \right|
  \\&\le Ce^{|y|} \cdot 4e^{-|y|} = 4C
\end{align*}
and \(g(z)\) is bounded by \(4C\) for \(z \in \{x + iy \mid x, y \in \bbR, |y| \ge 1\}\).

Suppose that \(|y| \le 1\).

If \(x = n\pi + \frac{\pi}{2}\) for \(n \in \bbN\), \(e^{ix} = -e^{-ix}\) because \((ix) - (-ix) = 2n\pi + \pi\), and
\[|\sin(z)|
  = \frac{1}{2} \left| e^{-y} e^{ix} - e^{y} e^{-ix} \right|
  = \frac{1}{2} \left| (e^{-y} + e^{y}) e^{ix} \right|
  = \frac{|e^{-y} + e^{y}|}{2}
  = \frac{e^{-y} + e^y}{2} \ge e^{-|y|} \]
And for this \(z\),
\[|g(z)|
  = \frac{|f(z)|}{|\sin (z)|}
  \le \frac{Ce^{|y|}}{e^{-|y|}}
  = C e^{2|y|}
  \le e^2 C\]

If \(|y| = 1\), as we shown above, \(g(z)\) is bounded by \(4C\).

Take \(D_n = (\frac{\pi}{2} - n\pi, \frac{\pi}{2} + n\pi) \times (-1, 1)\) in the complex plane for \(n \ge 2\).
Since \((\frac{\pi}{2} - n\pi, \frac{\pi}{2} + n\pi), (-1, 1)\) are non-empty open interval, \(D_n\) is open and connected.
Then, \(z \in \partial D_n\) satisfies \(\text{Re } z = \frac{\pi}{2} \pm n\pi\)
or \(\text{Im } z = 1\).
Thus, \(|g(z)| \le \max(4C, e^2C) = e^2C\) for \(z \in \partial D_n\).
Let \(M = \max_{z \in \partial D_n} |g(z)|\).
(\(M\) exists because \(g\) and \(|\cdot|\) are continuous, \(\partial D_n\) is non-empty, bounded, closed and compact, continuous image of compact set is compact, and a non-empty compact set has maximum.)
Then, \(M \le e^2C\).
Then, by Maximum-Modulus Theorem,
every \(z \in D_n\), \(|g(z)|\) cannot be a maximum of \(\overline{D_n}\) and \(|g(z)| \le M \le e^2C\).

Let \(z = x + iy \in \bbC\) such that \(|y| \le 1\).
And let \(n = 2 + \lceil \frac{|x|}{\pi} \rceil\).
Then, \(z \in D_n\) holds and \(|g(z)| \le e^2C\) as we shown above.
Therefore, modulus of \(g\) is bounded by \(e^2C\) for \(\{z = x + iy \mid x, y \in \bbR, |y| \le 1\}\).

Lastly, let \(U = \max(4C, e^2C) = e^2C\).
For \(z\) such that \(\left|\text{Im } z\right| \ge 1\), \(|g(z)| \le 4C \le U\).
For \(z\) such that \(\left|\text{Im } z\right| \le 1\), \(|g(z)| \le e^2C \le U\).
Thus, modulus of \(g\) is bounded by \(U\) in \(\bbC\).

Because \(g\) is entire and bounded,
by Liouville's Theorem,
\(g\) is a constant.
Let \(K \in \bbC\) such that
\[g(z) = K\]
Then, for \(z \not\in \pi\bbZ\),
\[f(z) = K \sin z\]

Let \(h(z) = f(z) - K\sin z\).
Since \(f(z) = K \sin z\) at \(z \not\in \pi\bbZ\),
\(h(z) = 0\) for \(z \in [\frac{\pi}{4}, \frac{3\pi}{4}]\).
Since \([\frac{\pi}{4}, \frac{3\pi}{4}]\) is compact,
it has a limit point in \(\bbC\),
such as \(\frac{\pi}{2}\).
Since \(f\) and \(\sin\) are entire, \(h\) is entire.
Thus, by Uniqueness Theorem, \(h(z) \equiv 0\) and \(f(z) = K\sin z\) for \(z \in \bbC\).

In conclusion, if \(f\) is entire, \(\forall n \in \bbZ: f(n\pi) = 0\), and \(\forall x, y \in \bbR: |f(x + iy)| \le C e^{|y|} < \infty\) for some fixed \(C > 0\), \(f = K\sin z\) for some \(K \in \bbR\).
It means, \(\{z \mapsto K \sin z \mid K \in \bbC\}\)
is a set consisting of all functions which satisfies every condition in the problem.
\qedsq