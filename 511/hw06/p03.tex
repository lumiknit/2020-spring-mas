\section*{Problem 3}

Let \(M, N\) be \(R\)-modules.
And let
\(P_\bullet \xrightarrow{\epsilon} M\) be a projective resolution
and
\(N \xrightarrow{\iota} I^\bullet\) be an injective resolution

Let \(T^\bullet := \Tot(\Hom_R(P_\bullet, I^\bullet)) = \{T^n\}_{n \ge 0}\)
with
\(T^n = \Tot(\Hom_R(P_\bullet, I^\bullet))) = \bigoplus_{i + j = n} \Hom_R(P_i, I^j)\).
We note that we have two natural morphisms of complexes:
\[\begin{tikzcd}
  & \Hom_R(P_\bullet, N) \ar["\iota_*"]{d} \\
  \Hom_R(M, I^\bullet) \ar["\epsilon^*"]{r} & T^*
\end{tikzcd}\]

Prove that the above \(\iota_*\) and \(\epsilon^*\) are both quasi-isomorphisms.

(Hint:
Show that \(\Hom_R(P_\bullet, -)\) preserves quasi-isomorphisms between cohomological complexes
and \(\Hom_R(-, I^\bullet)\) preserved quasi-isomorphisms between
homological complexes.)

\subsection*{Lemmata}

\begin{lemma}
  Let \(F\) be a covariant exact functor.
  \(F\) preserves (1) kernels; (2) images; (3) cokernels; (4) (co)homology; (5) quasi-isomorphisms.
\end{lemma}
\begin{proof}
  (1)
  Let \(f: A \to B\) be a homomorphism.
  Let's show \(F(\ker f) = \ker F(f)\).
  Take \(0 \to \ker f \to A \xrightarrow{f} \im f \to 0\).
  This is exact.
  We obtain \(0 \to F(\ker f) \to F(A) \xrightarrow{F(f)} F(\im f) \to 0\).
  Because \(F\) is an exact functor,
  that is exact.
  Thus, \(\ker F(f) = \im (F(\ker f) \to F(A)) = F(\ker f)\).
  It means \(F\) preserves kernels.

  (2)
  Let \(f: A \to B\) be a homomorphism.
  \(\im F(f) = F(f)(F(A)) = F(f(A)) = \im F(f)\).
  It means \(F\) preserves images.

  (3)
  Let \(f: A \to B\) be a homomorphism.
  Then, we can construct an exact sequence
  \(0 \to \coim f \xrightarrow{f} B \to \coker f \to 0\).
  By \(f\), we obtain an exact sequence
  \(0 \to F(\coim f) \xrightarrow{F(f)} F(B) \to F(\coker f) \to 0\).
  and,
  \(F(\coker f) \simeq F(B) / \im F(f) = \coker F(f)\).

  (4)
  Let \(A_\bullet\) be a chain complex with differentials \(\partial_\bullet\).
  Take an exact sequence
  \(0 \to \im\partial_{k - 1} \xrightarrow{f} \ker\partial_k \to \coker f \to 0\) where \(f\) is an injection.
  Then,
  \(0 \to \im F(\partial_{k - 1}) = F(\im\partial_{k - 1}) \xrightarrow{F(f)} \ker F(\partial_k) = F(\ker\partial_k) \to F(\coker f) \to 0\) is exact.
  Then,
  \(\coker f = \ker\partial_k / \im f \simeq \ker\partial_k / \im \partial_{k - 1} = H_k(A_\bullet)\),
  \(F(\coker f) \simeq \coker F(f) = \ker F(\partial_k) / \im F(f) \simeq \ker F(\partial_k) / \im F(\partial_{k - 1}) = H_k(F(A_\bullet))\).
  Thus, \(F(H_k(A_\bullet)) = F(\coker f) = H_k(F(A_\bullet))\).

  (5)
  Let \(A_\bullet\), \(B_\bullet\) be (co)chain complexes and \(f_\bullet: A_\bullet \to B_\bullet\) be a quasi-isomorphism.
  Then, \(f_k\) restricted to \(H_k(A_\bullet) \to H_k(B_\bullet)\) is an isomorphism (bijective homomorphism).
  In this case, \(0 \to H_k(A_\bullet) \xrightarrow{f_k} H_k(B_\bullet) \to 0 \to 0\) is a short exact sequence.
  By \(F\),
  \(0 \to F(H_k(A_\bullet)) \xrightarrow{F(f_k)} F(H_k(B_\bullet)) \to 0 \to 0\)
  is exact.
  Since \(F\) preserve (co)homologies, we obtain an exact sequence
  \(0 \to H_k(F(A_\bullet)) \xrightarrow{F(f_k)} H_k(F(B_\bullet)) \to 0 \to 0\).
  This shows, \(F(f_\bullet)\) is a quasi-isomorphism from \(F(A_\bullet)\) to \(F(B_\bullet)\).
\end{proof}

\begin{lemma}
  Let \(F\) be a contravariant exact functor.
  \(F\) preserves a quasi-isomorphism.
\end{lemma}
\begin{proof}
  It's a dual of above lemma.
\end{proof}

\begin{lemma}\label{lem-pres}
  \(\Hom_R(P_k, -)\) preserves a quasi-isomorphism.
  \(\Hom_R(-, I^k)\) preserves a quasi-isomorphism.
\end{lemma}
\begin{proof}
  Note that \(\Hom_R(P_k, -)\) and \(\Hom_R(-, I^k)\) are exact.
\end{proof}

\begin{lemma}\label{lem-comp}
  Let \(A_j^\bullet\) and \(B_j^\bullet\) be cochain complexes for each \(j \in \bbN\).
  And let \(f_j^\bullet: A_j^\bullet \to B_j^\bullet\) be a quasi-isomorphism for each \(j \in \bbN\).
  Let
  \(A_\bbN^\bullet = \bigoplus_{j \in \bbN} A_j^\bullet\)
  \(B_\bbN^\bullet = \bigoplus_{j \in \bbN} B_j^\bullet\)
  and \(f_\bbN^\bullet = \bigoplus_{j \in \bbN} f_j^\bullet\).
  Then, \(f_\bbN^\bullet: A_\bbN^\bullet \to B_\bbN^\bullet\) is a quasi-isomorphism.
\end{lemma}
\begin{proof}
  Let
  \(\alpha_j^\bullet\) be a differential of \(A_j^\bullet\)
  \(\beta_j^\bullet\) be a differential of \(B_j^\bullet\).

  Note that
  \(f_\bbN^\bullet\) maps
  \(j\)-th entry of \(A_\bbN^\bullet\)
  which is an element of \(A_j^\bullet\)
  to the \(j\)-th entry of \(B_\bbN^\bullet\)
  which is an element of \(B_j^\bullet\).

  Since \(f_\bbN^\bullet\) maps zero entries to zero entries,
  \(f_\bbN^\bullet\) is a well-defined homomorphism
  even it's a direct sum of infinitely many modules.

  Note that \(\ker \alpha_\bbN^k = \bigoplus_{j \in \bbN} \ker \alpha_j^k\)
  for each \(k \in \bbZ\),
  because for \(a \in A_\bbN^k\), \(\alpha_\bbN^k(a) = 0\) iff each \(j\)-th entry of \(a\) is in \(\ker \alpha_j^k\).

  Also, \(\im \alpha_\bbN^k = \bigoplus_{j \in \bbN} \im \alpha_j^k\)
  for each \(k \in \bbZ\),
  because for \(b \in A_\bbN^{k+1}\), there is \(a \in A_\bbN^k\)
  such that \(\alpha_\bbN^k(a) = b\)
  iff each \(j\)-th entries of \(a\) are mapped to the \(j\)-th entry of \(b\) by \(f_j^k\).

  In the same way,
  \(\ker \beta_\bbN^k = \bigoplus_{j \in \bbN} \ker \beta_j^k\)
  and
  \(\im \beta_\bbN^k = \bigoplus_{j \in \bbN} \im \beta_j^k\).

  Thus,
  \(H^k(A_\bbN^\bullet)
    = \ker \alpha_\bbN^k / \im \alpha_\bbN^{k-1}
    \bigoplus_{j \in \bbN} \ker \alpha_j^k / \im \alpha_j^k \)
  and
  \(H^k(B_\bbN^\bullet)
    = \ker \beta_\bbN^k / \im \beta_\bbN^{k-1}
    \bigoplus_{j \in \bbN} \ker \beta_j^k / \im \beta_j^k \)

  Because \(f_\bbN^\bullet\) is an isomorphism
  between each \(j\)-th entry of \(H^\bullet(A_\bbN^\bullet)\) and \(H^\bullet(B_\bbN^\bullet)\).
  Therefore, \(f_\bbN^\bullet\) is an isomorphism between \(H^\bullet(A_\bbN^\bullet)\) and \(H^\bullet(B_\bbN^\bullet)\).

  This shows \(f_\bbN^\bullet\) is a quasi-isomorphism.
\end{proof}

\subsection*{Proof}

Because \(P_\bullet \to M\) is a projective resolution,
we have a quasi isomorphism:
\[\begin{tikzcd}
  \cdots \ar{r} & P_2 \ar{d} \ar{r} & P_1 \ar{d} \ar{r} & P_0 \ar["\epsilon"]{d} \ar{r} & 0 \ar{d} \ar{r} & \cdots \\
  \cdots \ar{r} & 0 \ar{r} & 0 \ar{r} & M \ar{r} & 0 \ar{r} & \cdots
\end{tikzcd}\]
Then, \(\Hom_R(-, I^n)\) maps above as
\[\begin{tikzcd}
  \cdots \ar{r} & 0 \ar{d} \ar{r} & \Hom_R(M, I^n) \ar["\Hom_R(\epsilon{,}I^n)"]{d} \ar{r} & 0 \ar{d} \ar{r} & 0 \ar{d} \ar{r} & \cdots \\
  \cdots \ar{r} & 0 \ar{r} & \Hom_R(P_0, I^n) \ar{r} & \Hom_R(P_1, I^n) \ar{r} & \Hom_R(P_2, I^n) \ar{r} & \cdots
\end{tikzcd}\]
where the chain map between two rows is a quasi-isomorphism by Lemma \ref{lem-pres}.

Let \(A_n^{\bullet - n}\) be the above rows and \(B_n^{\bullet - n}\) be the below rows of the above diagram.
In other words,
\[A_n^k = \left\{\begin{array}{ll}
\Hom_R(M, I^k) & (k = n) \\
0 & (k \neq n) \\
\end{array}\right.
,
B_n^k = \left\{\begin{array}{ll}
\Hom_R(P_{k - n}, I^k) & (k \ge n) \\
0 & (k < n) \\
\end{array}\right.
\text{ where }
k \in \bbZ
\]
Let \(f_n^\bullet: A_n^\bullet \to B_n^\bullet\) be a quasi-isomorphism
such that
\(f_n^n = \Hom_R(\epsilon, I^n)\), \(f_n^k = 0\) for \(k \neq n\).

Let
\(A_\bbN^\bullet = \bigoplus_{j\in\bbN} A_j^\bullet\),
\(B_\bbN^\bullet = \bigoplus_{j\in\bbN} B_j^\bullet\),
and
\(f_\bbN^\bullet = \bigoplus_{j\in\bbN} f_j^\bullet\).

Then,
\[A_\bbN^k = \left\{\begin{array}{ll}
\Hom_R(M, I^k) & (k \ge 0) \\
0 & (k < 0) \\
\end{array}\right.
,
B_\bbN^k = \left\{\begin{array}{ll}
\bigoplus_{i+j=k} \Hom_R(P_i, I^j) & (k \ge 0) \\
0 & (k < 0) \\
\end{array}\right.
\text{ where }
k \in \bbZ
\]
Note that
\(A_\bbN^k = \Hom_R(M, I^k)\),
and
\(B_\bbN^k = T^k\) since \(P_\bullet, I^\bullet\) of negative degree are zero.
In other words \(f_\bbN^\bullet\) is a chain map
from \(\Hom_R(M, I^\bullet)\)
to \(T^k\).

By Lemma \ref{lem-comp},
\(f_\bbN^\bullet = \bigoplus_{j \in \bbN} f_j^\bullet: A_\bbN^\bullet \to B_\bbN^\bullet\)
is a quasi-isomorphism.
Let \(\epsilon^* = f_\bbN^\bullet\).

%-----------------------------------------------------------
\br
\noindent
In the same way, we have a quasi-isomorphism from an injective resolution:
\[\begin{tikzcd}
  \cdots \ar{r} & 0 \ar{d} \ar{r} & N \ar["\iota"]{d} \ar{r} & 0 \ar{d} \ar{r} & 0 \ar{d} \ar{r} & \cdots \\
  \cdots \ar{r} & 0 \ar{r} & I^0 \ar{r} & I^1 \ar{r} & I^2 \ar{r} & \cdots
\end{tikzcd}\]
and we obtain the below diagram:
\[\begin{tikzcd}
  \cdots \ar{r} & 0 \ar{d} \ar{r} & \Hom_R(P_n, N) \ar["\Hom_R(P_n{,} \iota)"]{d} \ar{r} & 0 \ar{d} \ar{r} & 0 \ar{d} \ar{r} & \cdots \\
  \cdots \ar{r} & 0 \ar{r} & \Hom_R(P_n, I^0) \ar{r} & \Hom_R(P_n, I^1) \ar{r} & \Hom_R(P_n, I^2) \ar{r} & \cdots
\end{tikzcd}\]
where the chain map between two rows is a quasi-isomorphism
by \ref{lem-pres}.

Define \(C_n^{\bullet - n}\), \(D_n^{\bullet - n}\) as:
\[C_n^k = \left\{\begin{array}{ll}
\Hom_R(P_k, N) & (k = n) \\
0 & (k \neq n) \\
\end{array}\right.
,
D_n^k = \left\{\begin{array}{ll}
\Hom_R(P_k, I^{k - n}) & (k \ge n) \\
0 & (k < n) \\
\end{array}\right.
\text{ where }
k \in \bbZ
\]
Let \(g_n^\bullet: C_n^\bullet \to D_n^\bullet\) be a quasi-isomorphism
such that
\(g_n^n = \Hom_R(P_n, \iota)\), \(g_n^k = 0\) for \(k \neq n\).

Let
\(C_\bbN^\bullet = \bigoplus_{j\in\bbN} C_j^\bullet\),
\(D_\bbN^\bullet = \bigoplus_{j\in\bbN} D_j^\bullet\),
and
\(g_\bbN^\bullet = \bigoplus_{j\in\bbN} g_j^\bullet\).

Then,
\[C_\bbN^k = \left\{\begin{array}{ll}
\Hom_R(P_k, N) & (k \ge 0) \\
0 & (k < 0) \\
\end{array}\right.
,
D_\bbN^k = \left\{\begin{array}{ll}
\bigoplus_{i+j=k} \Hom_R(P_i, I^j) & (k \ge 0) \\
0 & (k < 0) \\
\end{array}\right.
\text{ where }
k \in \bbZ
\]
Note that
\(C_\bbN^k = \Hom_R(P_k, N)\),
\(D_\bbN^k = T^k\).
Also, by Lemma \ref{lem-comp},
\(g_\bbN^\bullet = \bigoplus_{j \in \bbN} g_j^\bullet: C_\bbN^\bullet \to D_\bbN^\bullet\)
is a quasi-isomorphism.
Let \(\iota^* = g_\bbN^\bullet\).
\br
\noindent
Therefore,
\(\epsilon_* = f_\bbN^\bullet: \Hom_R(M, I^\bullet) \to T^\bullet\)
and
\(\iota_* = g_\bbN^\bullet: \Hom_R(P_\bullet, N) \to T^\bullet\)
are quasi-isomorphisms.
\qedsq