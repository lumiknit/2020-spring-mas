\section*{Problem 6}

Note:
Let \(e_i: 0 \to N \to T_i \to M \to 0\) be s.e.s for \(i = 1, 2\).

Consider the pull-back \(T'\) of \(T_1 \to M \leftarrow T_2\)
i.e. \(T' \subseteq T_1 \times T_2\)
consisting of \((t_1, t_2)\) whose images in \(M\) coincide.
Let \(D \subseteq T'\) be generated by \((-n ,n)\) for \(n \in N\)
and let \(T = T'/D\).

This gives a s.e.s \(e: 0 \to N \to T \to M \to 0\), which is a Baer sum of \(e_1\) and \(e_2\).
\br
\noindent
For \(e_1, e_2 \in \Ext_R(M, N)\),
prove that defining \(e_1 + e_2 = e\) gives an abelian group structure on \(\Ext_R(M, N)\).

For \(0 \to N \to T \to M \to 0\), what is the inverse in the group?

\subsection*{Proof}

Let the below be the diagram of pullback \(T'\):
\[\begin{tikzcd}
  T' \ar["\pi_1"]{d} \ar["\pi_2"]{r} & T_2 \ar["\mu_2"]{d} \\
  T_1 \ar["\mu_1"]{r} & M
\end{tikzcd}\]
Also, let \(\nu_j: N \to T_j\) for \(j = 1, 2, 3\).

Let \(e = e_1 + e_2: 0 \to N \xrightarrow{\nu} T \xrightarrow{\mu} M \to 0\) such that
\(\nu(n) = (\nu_1(n), 0) + D = (0, \nu_2(n)) + D\)
and
\(\mu((t_1, t_2) + D) = \mu_1(t_1) = \mu_2(t_2)\).
Note that they are well-defined.
First, \((\nu_1(-n), \nu_2(n)) \in D\) for any \(n \in N\).
This shows that
\((\nu_1(n), 0) + D = (0, \nu_2(n)) + D\).
Second, for \(t_1 \in T_1\) and \(t_2 \in T_2\),
\((t_1, t_2) \in T\) implies \(\mu_1(t_1) = \mu_2(t_2)\).
Also, if there are \(t_1, t'_1 \in T_1\) and \(t_2, t'_2 \in T_2\)
such that \((t_1, t_2) + D = (t'_1, t'_2) + D\),
\((t_1 - t'_1, t_2 - t'_2) \in D\).
Then, there is \(n \in N\) such that \(\nu_1(n) = t_1 - t'_1\) and \(\nu_2(n) = t_2 - t'_2\).
Since \(\mu_j \circ \nu_j = 0\),
\(\mu_1(t_1 - t'_1) = \mu_2(t_2 - t'_2) = 0\).
This shows \(\mu\) is well-defined.

First, \(\Ext_R(M, N)\) is closed under \(+\).

\((\Ext_R(M, N), +)\) is commutative.
It's because \(T_1 \times T_2 \simeq T_2 \times T_1\)
and \(T_1 \times T_2\) and \(T_2 \times T_1\) gives isomorphic pullbacks.
Let \(\varphi\) be an isomorphism from \(T_1 \times T_2\) to \(T_2 \times T_1\).
Then, let \(U = T_1 + T_2\) be \(T' / D\) obtained from \(T_1 \times T_2\),
and let \(V = T_2 + T_1\) be \(T' / D\) obtained from \(T_2 \times T_1\).
Then, \(U \simeq V\) by an isomorphism induced by \(\varphi\).
Therefore, \(0 \to N \to U \to M \to 0\) and \(0 \to N \to V \to M \to 0\) are equivalent by \(\varphi\).
This shows \(T_1 + T_2 = T_2 + T_1\).

\((\Ext_R(M, N), +)\) is associative.
Let we have three extensions of \(M\) by \(N\), through \(T_1, T_2, T_3\).
Let \(T'_1 = \{(t_1, t_2) \mid \mu_1(t_1) = \mu_2(t_2)\} \subseteq T_1 \times T_2\).
Let \(D_1\) be generated by \((-n, n) \in T'_1\) for \(n \in N\).
Then, we obtain induced \(\tilde\mu_1: T'_1/D_1 \to T\)
such that \(\tilde\mu_1((t_1, t_2) + D_1) = \mu_1(t_1) = \mu_2(t_2)\),
and \(\tilde\nu_1(n) = (\nu_1(n), 0) + D_1 = (0, \nu_2(n)) + D_1\).
It's well-defined because \(D_1\) is a submodule which makes \(\mu_1(t_1) = \mu_2(t_2) = 0\) because \(\mu_k \circ \nu_k = 0\).
Then, let \(T''_1 = \{(t_1, t_3) \mid \tilde\mu_1(t_1) = \mu_3(t_3)\} \subseteq T'_1 / D_1 \times T_3\).
Let \(D'_1\) be generated by \((-n, n) \in T''_1\) for \(n \in N\).
In the same way, we can generate \(T'_2, D_2, \tilde\mu_2\) from \(T_2\) and \(T_3\),
and we can generate \(T''_2, D'_2\) from \(T'_2\).
What we need to show is \(T''_1 / D'_1 \simeq T''_2 / D'_2\).
Let \(t_k \in T_k\) for \(k = 1, 2, 3\).
Then, let \(\tau: T''_1 / D'_1 \to T''_2 / D'_2\) such that
\[\tau: ((t_1, t_2) + D_1, t_3) + D'_1 \mapsto (t_1, (t_2, t_3) + D_2) + D'_2\]
First, let's check it's well-defined.
Suppose that \(((t_1, t_2) + D_1, t_3) + D'_1 = ((t'_1, t'_2) + D_1, t'_3) + D'_1 \in T''_1 / D'_1\)
for \(t_k, t'_k \in T_k\) for \(k = 1, 2, 3\).
Let \(d_k = t_k - t'_k\) for \(k = 1, 2, 3\).
Then, \(0 = ((d_1, d_2) + D_1, d_3) + D'_1\).
This shows \(((d_1, d_2) + D_1, d_3) \in D'_1\), and there is \(n \in N\) such that
\(d_3 = \nu_3(n)\) and
\((d_1, d_2) + D_1 = \tilde\nu_1(n) = (\nu_1(n), 0) + D_1\).
Then, there is \(n' \in N\) such that
\(d_1 = \nu_1(n - n')\),
\(d_2 = \nu_2(n')\).
Then, \((\nu_1(n-n'), (\nu_2(n'), \nu_3(n)) + D_2) + D'_2 =
(0, (\nu_2(n), \nu_3(n)) + D_2) + D'_2
= (0, (0, 0) + D_2) + D'_2\).
Thus, \(\tau\) maps \(((t_1, t_2) + D_1, t_3) + D'_1\) and \(((t'_1, t'_2) + D_1, t'_3) + D'_1\) to the same value if they are same.
\(\tau\) is a homomorphism, since it's a composition of homomorphism.
Thus, because of the equivalence of extension and there is a homomorphism \(\tau\),
\(0 \to N \to T''_1 / D'_1 \to M \to 0\) and \(0 \to N \to T''_2 / D'_2 \to M \to 0\)
are equivalent.
This shows that \((T_1 + T_2) + T_3 = T_1 + (T_2 + T_3)\).

\(\Ext_R(M, N)\) contains an identity, \(N \oplus M\).
Suppose that \(T_2 = N \oplus M\) and \(\mu_2\) be a canonical projection of \(N \oplus M\) to \(M\).
Let \(t \in T_1, n \in N, m \in M\).
\((t, (n, m)) + D \in T' / D\) if \(\mu_1(t) = \mu_2(n, m) = m\).
Since, \((-n, n)\) for \(n \in N\) is mapped to \((\nu_1(-n), (n, 0))\) by \((\nu_1, \nu_2)\),
and \(\mu_1(\nu_1(-n)) = 0 = \mu_2(n, 0)\) because \(\mu_1 \circ \nu_1 = \mu_2 \circ \nu_2 = 0\),
\((\nu_1(-n), (n, 0)) \in T'\).
Thus, \(D = \{(\nu_1(-n), (n, 0)) \mid n \in N\}\).
Let \(\varphi: T_1 \to T' / D\) such that \(\varphi: t \mapsto (t, (0, \mu_1(t))) + D\).
This is injective, because if \((t, (0, \mu_1(t))) + D = D\),
\((t, (0, \mu_1(t))) \in D\),
and \(t = \nu_1(0)\).
Since \(\nu_1\) is injective, \(t = 0\).
This shows \(\ker \varphi = \{0\}\).
\(\varphi\) is surjective.
If \((t, (n, m)) + D \in T' / D\),
since \((\nu_1(-n), (n, 0)) \in D\),
\((t, (n, m)) + D = (t + \nu_1(n), (0, m)) + D\).
Note that \((t, (n, m)) \in T'\), \(\mu_1(t) = \mu_2(n, m) = m\).
Thus, \(\mu_1(t) = \mu_1(t + \nu_1(n)) = \mu_2(0, m) = m\), as \(\mu_1 \circ \nu_1 = 0\).
Therefore, \(t + \nu_1(n)\) is mapped to \((t, (n, m)) + D \in T' / D\) by \(\varphi\)
and \(\varphi\) is surjective.
Lastly, \(\varphi\) is a homomorphism because it's a composition of homomorphisms.
Therefore, \(\varphi\) is an isomorphism and
\(T_1 \simeq T' / D\).
This shows, \(N \oplus M\) is the right identity.
Since \(\Ext_R(N, M)\) is commutative, \(N \oplus M\) is also the left identity.

\(\Ext_R(M, N)\) contains an inverse of arbitrary \(T_1\).
Let \(T_2 = T_1\), \(\nu_2 = -\nu_1\) and \(\mu_2 = \mu_1\).
Let's show that \(T_2\) is an inverse of \(T_1\).
Then, \(T' \subseteq T_1 \times T_2 = T_1^2\).
And, \((t_1, t_2) \in T'\), \(\mu_1(t_1) = \mu_2(t_2) = \mu_2(t_2)\).
In other words, \(T' = \{(t_1, t_2) \mid t_1 - t_2 \in \ker \mu_1 = \im \nu_1\}\).
And \(D = \{(\nu_1(-n), \nu_2(n)) \mid n \in N\} = \{(t, t) \mid n \in N, t = -\nu_1(t) = \nu_2(t)\}\).
Then, let \(\varphi: M \to T' / D\) such that \(\varphi(m) = (t, t) + D\) where \(\mu_1(t) = \mu_2(t) = m\).
Let \(t_1, t_2 \in T_1 = T_2\).
If \(\mu_1(t_1) = \mu_1(t_2) = m\),
\(\mu_1(t_1 - t_2) = 0\),
\(t_1 - t_2 \in \ker \mu_1 = \im \nu_1\).
Then, there is \(n \in N\) such that \(\nu_1(n) = t_1 - t_2\).
This shows \((t_1 - t_2, t_1 - t_2) \in D\), and \((t_1, t_1) + D = (t_2, t_2) + D\).
Thus, \(\varphi\) is well-defined.
And since it's an homomorphism because it's an inverse of homomorphism.
Therefore, \(0 \to N \to T'/D \to M \to 0\) is split.
This shows \(T'/D \simeq N \otimes M\), which is an identity of Baer sum.
Therefore, the \(T_2 = T_1\) with \(\nu_2 = -\nu_1\) and \(\mu_2 = \mu_1\) is an inverse of \(T_1\).

Therefore, \((\Ext_R(M, N), +)\) is closed, associative, commutative, and contains an identity and all inverses of its elements.
\qedsq
