\section*{Problem 7}

With the above Baer sum group structure on \(\Ext_R(M, N)\),
prove that \(\delta: \Ext_R(M, N) \to \Ext_R^1(M, N)\) is a group homomorphism.

\subsection*{Proof}

Let
\(e_1: [0 \to N \to T_1 \to M \to 0]\)
and
\(e_2: [0 \to N \to T_2 \to M \to 0]\).
Then,
\(e_3 = e_1 + e_2: [0 \to N \to (T_1 \times_M T_2) / D \to M \to 0]\).

\(\delta(e_1) = \partial_1(\Id_M)\),
\(\delta(e_2) = \partial_2(\Id_M)\),
\(\delta(e_3) = \delta(e_1 + e_2) = \partial_3(\Id_M)\)
where each \(\partial_1\), \(\partial_2\), \(\partial_3\) are come from the \(\Ext\) long exact sequence of \(e_1\), \(e_2\), \(e_1 + e_2\).
\br
\noindent
Note that
\(\partial_1:
  (\ker (\Hom(M, I^{0}) \to \Hom(M, I^{1}))
        /
   \im (\Hom(M, I^{-1}) \to \Hom(M, I^{0}))
        )
  \to
  (\ker (\Hom(N, I^{1}) \to \Hom(N, I^{2}))
        /
   \im (\Hom(N, I^{0}) \to \Hom(N, I^{1}))
        )
\).
\br
\noindent
Note that, if we apply Snake Lemma to below diagram with exact rows,
\[\begin{tikzcd}
  & A \ar["f"]{d} \ar["\alpha"]{r} & B \ar["g"]{d} \ar["\beta"]{r} & C \ar["h"]{d} \ar{r} & 0 \\
  0 \ar{r} & A' \ar["\alpha'"]{r} & B \ar["\beta'"]{r} & C &
\end{tikzcd}\]
we obtain the exact sequence
\[\begin{tikzcd}
\ker f \ar["\tilde\alpha"]{r} & \ker g \ar["\tilde\beta"]{r} & \ker h \ar["\partial"]{r} & \coker f \ar["\tilde\alpha'"]{r} & \coker g \ar["\tilde\beta'"]{r} & \coker h
\end{tikzcd}\]
where the funcitons with tilde are induced from ones without tilde.
Also \(\partial\) is constructed as \(\tilde\alpha'^{-1} \circ g \circ \tilde\beta^{-1}\) for the restricted domain \(\ker h\).
\br
\noindent
Suppose that
\(0 \to N \xrightarrow{\nu} T \xrightarrow{\mu} M \to 0\)
is an exact sequence.
Let
\(N \xrightarrow{f} I_N^\bullet\),
\(T \xrightarrow{g} I_T^\bullet\),
\(M \xrightarrow{h} I_M^\bullet\)
be injective resolutions.
We obtain \(\Ext\) beginning from
\[\begin{tikzcd}[column sep=huge]
  0 \ar{r} & \Hom(M, I_N^k) \ar["\Hom(M{,}f^k)"]{d} \ar["\Hom(M{,}\nu^k)"]{r} & \Hom(M, I_T^k) \ar["\Hom(M{,}g^k)"]{d} \ar["\Hom(M{,}\mu^k)"]{r} & \Hom(M, I_M^k) \ar["\Hom(M{,}h^k)"]{d} \ar{r} & 0 \\
  0 \ar{r} & \Hom(M, I_N^{k+1}) \ar["\Hom(M{,}\nu^{k+1})"]{r} & \Hom(M, I_T^{k+1}) \ar["\Hom(M{,}\mu^{k+1})"]{r} & \Hom(M, I_M^{k+1}) \ar{r} & 0
\end{tikzcd}\]
Note that \(\Ext\) long exact sequence is obtained by applying
the Snake Lemma twice to the above diagram. (See Problem 2)

At the first application of the Snake Lemma,
we obtain an exact sequence
\begin{align*}
  &\ker \Hom(M, f^k) \xrightarrow{\phi}
  \ker \Hom(M, g^k) \xrightarrow{\psi}
  \ker \Hom(M, h^k) \\
  \xrightarrow{\partial}
  &\coker \Hom(M, f^k) \xrightarrow{\phi'}
  \coker \Hom(M, g^k) \xrightarrow{\psi'}
  \coker \Hom(M, h^k)
\end{align*}
Note that each \(\phi, \psi, \phi', \psi'\) is induced from
\(\Hom(M, \nu^k), \Hom(M, \mu^k), \Hom(M, \nu^{k + 1}), \Hom(M, \mu^{k + 1})\).
Also, each \(\Hom(M, f^k), \Hom(M, g^k), \Hom(M, h^k)\) induces to maps from cokernels to kernels.

And at the second application of the Snake Lemma,
we obtain a map
\[\partial: \coker \Hom(M, h^{k - 1})^* \to \ker \Hom(M, f^{k + 1})^*\]
where \(\Hom(M, h^{k - 1})^*\) and \(\Hom(M, f^{k + 1})^*\) are induced one.
Note that \(\partial: \Ext_R^k(M, N) \to \Ext_R^{k + 1}(M, N)\),
and \(\partial = (\phi')^{-1} \circ \Hom(M, g^{k})^* \circ (\psi)^{-1}\)
for the restricted domain following applications of the Snake Lemma.
\br
\noindent
Let
\[0 \to N \xrightarrow{\nu_j} T_j \xrightarrow{\mu_j} M \to 0\]
be exact sequences for \(j = 1, 2, 3\).
Note that we can induce \(\partial_j\) from above sequences.
Then, for each injective resolutions
\(N \xrightarrow{f^\bullet} I_N^\bullet\),
\(T_j \xrightarrow{g_j^\bullet} I_{Tj}^\bullet\),
\(M \xrightarrow{h^\bullet} I_M^\bullet\),
exact sequence \(0 \to I_N^\bullet \xrightarrow{\nu_j^\bullet} I_{Tj}^\bullet \xrightarrow{\mu_j^\bullet} I_M^\bullet \to 0\),
\[\partial_j =
  (\Hom(M, \nu_j^1)^*)^{-1} \circ \Hom(M, g_j^0)^* \circ (\Hom(M, \mu_j^0)^*)^{-1}\]
where each maps with a superscript \(*\) are induced one in the diagram just before the second application of the Snake Lemma.

What we need to show is \(\partial_3(\Id_M) = \partial_1(\Id_M) + \partial_2(\Id_M)\).

Note that
\begin{align*}
  \partial_j(\Id_M)
  &= ((\nu_j^1)^*)^{-1} \circ (g_j^0)^* \circ ((\mu_j^0)^*)^{-1} \circ \Id_M
  \\&= ((\nu_j^1)^*)^{-1} \circ (g_j^0)^* \circ ((\mu_j^0)^*)^{-1}
\end{align*}
which is a homomorphism from \(M\) to \(\Cod(\nu_j^1)\).

Let \(m \in \Dom(\im\partial_j)\).
Let \(\partial_j(m) = \overline{n_j}\) for \(n_j \in N\).

As we noted at the Problem 6,
\(\mu_3(\overline{(t_1, t_2)}) = \mu_1(t_1) = \mu_2(t_2)\).
And since Baer sum is defined as quotient of pullback,
\(\overline{(t_1, t_2)}
= \overline{(\nu_1(n_1), \nu_2(n_2))}
= \overline{(\nu_1(n_1), 0)} + \overline{(0, \nu_2(n_2))}\)
And,
\(\nu_3^{-1}(\overline{(t_1, t_2)}) = n_1 + n_2\).
Thus, for every \(m\),
\[\partial_3(\Id_M)(m) = n_1 + n_2 = \partial_1(\Id_M)(m) + \partial_2(\Id_M)(m)\]

Therefore,
\[\delta(e_1 + e_2) = \delta(e_3) = \partial_3(\Id_M) = \partial_1(\Id_M) + \partial_2(\Id_M) = \delta(e_1) + \delta(e_2)\]
\qedsq