From the below problem, \(\Hom\) without subscript means \(\Hom_R\).

\section*{Problem 4}

Let \(R\) be a ring with a unity and
let \(M, N\) be \(R\)-modules.

Note:
\(\Ext_R(M, N)\) is the set of equivalence classes of extensions of \(M\) by \(N\) (i.e. eq cls by \(0 \to N \to T \to M \to 0\) and \(~\) is there is a homo \(T\) to \(T'\))

For given \(e := [0 \to N \to T \to M \to 0] \in \Ext_R(M, N)\),
we can make \(\Ext_R(M, -)\) long sequence:
\[0 \to \Ext_R^0(M, N) \to \Ext_R^0(M, T) \to \Ext_R^0(M, M) \xrightarrow{\partial} \Ext_R^1(M, N) \to \cdots\]
Note that \(\Id_M \in \Hom_R(M, M) = \Ext_R^0(M, M)\).

Let \(\delta: \Ext_R(M, N) \to \Ext_R^1(M, N)\).
Then, let \(\delta(e) = \partial(\Id_M)\).

We can find \(0 \to K \xrightarrow{\alpha} P \to M \to 0\)
where \(P\) is projective.
Then, from long exact sequence, we obtain
\[\Hom_R(P, N) \to \Hom_R(K, N) \xrightarrow{\partial'} \ext_R^1(M, N) \to 0 = \ext_R^1(P, N)\]

Let \(x \in \Ext_R^1(M, N) \simeq \ext_R^1(M, N)\).
There is some \(\beta \in \Hom_R(K, N)\) such that \(\partial'(\beta) = x\).

Consider the push-out \(T\) of \(N \xleftarrow{\beta} K \xrightarrow{\alpha} P\)
which is \(\coker(K \to P \oplus N)\) for the map \(k \mapsto (\alpha(k), -\beta(k))\).
Let \(i: N \to T\) be the natural induced morphism.
Also, because of surjection \(\pi: P \to M\) and \(\phi: P \to T\), we obtain \(\psi: T \to M\) such that \(\psi = \pi\phi^{-1}\).

In this case, we obtain the below diagram with exact rows:
\[\begin{tikzcd}
  0 \ar{r} & K \ar["\beta"]{d} \ar["\alpha"]{r} & P \ar["\phi"]{d} \ar["\pi"]{r} & M \ar[equal]{d} \ar{r} & 0 \\
  0 \ar{r} & N \ar["i"]{r} & T \ar["\psi"]{r} & M \ar{r} & 0
\end{tikzcd}\]

Let \(e = [0 \to N \to T \to M \to 0]\).

Prove that \(\delta(e) = x\).

%-------------------------------------------------------------------------------
\subsection*{Proof}



%-----------------------------------------------------------
% From Problem 7 (Injective)
Note that, if we apply Snake Lemma to below diagram with exact rows,
\[\begin{tikzcd}
  & A \ar["f"]{d} \ar["\alpha"]{r} & B \ar["g"]{d} \ar["\beta"]{r} & C \ar["h"]{d} \ar{r} & 0 \\
  0 \ar{r} & A' \ar["\alpha'"]{r} & B \ar["\beta'"]{r} & C &
\end{tikzcd}\]
we obtain the exact sequence
\[\begin{tikzcd}
\ker f \ar["\tilde\alpha"]{r} & \ker g \ar["\tilde\beta"]{r} & \ker h \ar["\partial"]{r} & \coker f \ar["\tilde\alpha'"]{r} & \coker g \ar["\tilde\beta'"]{r} & \coker h
\end{tikzcd}\]
where the funcitons with tilde are induced from ones without tilde.
Also \(\partial\) is constructed as \(\tilde\alpha'^{-1} \circ g \circ \tilde\beta^{-1}\) for the restricted domain \(\ker h\).
\br
\noindent
Suppose that
\(0 \to N \xrightarrow{\alpha} T \xrightarrow{\pi} M \to 0\)
is an exact sequence.
Let
\(N \xrightarrow{f} I_N^\bullet\),
\(T \xrightarrow{g} I_T^\bullet\),
\(M \xrightarrow{h} I_M^\bullet\)
be injective resolutions.
We obtain \(\Ext\) beginning from
\[\begin{tikzcd}[column sep=huge]
  0 \ar{r} & \Hom(M, I_N^k) \ar["\Hom(M{,}f^k)"]{d} \ar["\Hom(M{,}\alpha^k)"]{r} & \Hom(M, I_T^k) \ar["\Hom(M{,}g^k)"]{d} \ar["\Hom(M{,}\pi^k)"]{r} & \Hom(M, I_M^k) \ar["\Hom(M{,}h^k)"]{d} \ar{r} & 0 \\
  0 \ar{r} & \Hom(M, I_N^{k+1}) \ar["\Hom(M{,}\alpha^{k+1})"]{r} & \Hom(M, I_T^{k+1}) \ar["\Hom(M{,}\pi^{k+1})"]{r} & \Hom(M, I_M^{k+1}) \ar{r} & 0
\end{tikzcd}\]
Note that \(\Ext\) long exact sequence is obtained by applying
the Snake Lemma twice to the above diagram. (See Problem 2)

At the first application of the Snake Lemma,
we obtain an exact sequence
\begin{align*}
  &\ker \Hom(M, f^k) \xrightarrow{\sigma}
  \ker \Hom(M, g^k) \xrightarrow{\upsilon}
  \ker \Hom(M, h^k) \\
  \xrightarrow{\partial}
  &\coker \Hom(M, f^k) \xrightarrow{\sigma'}
  \coker \Hom(M, g^k) \xrightarrow{\upsilon'}
  \coker \Hom(M, h^k)
\end{align*}
Note that each \(\sigma, \upsilon, \sigma', \upsilon'\) is induced from
\(\Hom(M, \alpha^k), \Hom(M, \pi^k), \Hom(M, \alpha^{k + 1}), \Hom(M, \pi^{k + 1})\).
Also, each \(\Hom(M, f^k), \Hom(M, g^k), \Hom(M, h^k)\) induces to maps from cokernels to kernels.

And at the second application of the Snake Lemma,
we obtain a map
\[\partial: \coker \Hom(M, h^{k - 1})^* \to \ker \Hom(M, f^{k + 1})^*\]
where \(\Hom(M, h^{k - 1})^*\) and \(\Hom(M, f^{k + 1})^*\) are induced one.
Note that \(\partial: \Ext_R^k(M, N) \to \Ext_R^{k + 1}(M, N)\),
and \(\partial = (\sigma')^{-1} \circ \Hom(M, g^{k})^* \circ (\upsilon)^{-1}\)
for the restricted domain following applications of the Snake Lemma.
\br
\noindent
Note that the exact sequence
\[0 \to N \xrightarrow{\alpha} T \xrightarrow{\pi} M \to 0\]
was given.
Note that we can induce \(\partial_j\) from above sequences.
Then, for the injective resolutions
\(T \xrightarrow{g} I^\bullet\) and differentials \(g^\bullet\) of \(I^\bullet\),
\[\partial =
  (\Hom(M, \alpha^1)^*)^{-1} \circ \Hom(M, g^0)^* \circ (\Hom(M, \pi^0)^*)^{-1}\]
where each maps with a superscript \(*\) are induced one in the diagram just before the second application of the Snake Lemma.

%-----------------------------------------------------------
% Original (Projective)
\br\noindent
In the similar way,
For the given exact sequence \(0 \to K \xrightarrow{i} P \xrightarrow{\psi} M \to 0\),
\[\partial': \ext_R^0(K, N) \to \ext_R^1(M, N)\]
obtained by \(\ext\) long exact sequence is
\(\partial' = (\Hom(\psi_{k + 1}, N)^*)^{-1} \circ \Hom(g_k, N)^* \circ (\Hom(i_k, N)^*)^{-1}\)
where
supserscript \(*\) means `induced',
\(0 \to P_{K\bullet} \xrightarrow{i_\bullet} P_{P\bullet} \xrightarrow{\psi_\bullet} P_{M\bullet} \to 0\)
is exact,
\(P_{K\bullet} \xrightarrow{f} K\),
\(P_{P\bullet} \xrightarrow{g} P\),
\(P_{M\bullet} \xrightarrow{h} M\)
are projective resolutions,
and \(g_\bullet\) are differentials of \(P_{P\bullet}\).

%-------------------------------------------------------------------------------
\br\noindent
Because \(x = \partial'(\beta)\) and \(\partial(\Id_M) = \delta(e)\),
it's enough to show that \(\partial(\Id_M) = \partial'(\beta)\).

Note that,
\[\partial(\Id_M) =
  ((\alpha^1)^*)^{-1} \circ (g^0)^* \circ ((\pi^0)^*)^{-1} \circ \Id^M =
  ((\alpha^1)^*)^{-1} \circ (g^0)^* \circ ((\pi^0)^*)^{-1}\]
\[\partial'(\beta) =
  \beta \circ ((i_0)^*)^{-1} \circ (g_0)^* \circ ((\psi'_1)^*)^{-1}\]

%-------------------------------------------------------------------------------
Then, from the below given diagram,
\[\begin{tikzcd}
  0 \ar{r} & K \ar["\beta"]{d} \ar["\alpha"]{r} & P \ar["\phi"]{d} \ar["\pi"]{r} & M \ar[equal]{d} \ar{r} & 0 \\
  0 \ar{r} & N \ar["i"]{r} & T \ar["\psi"]{r} & M \ar{r} & 0
\end{tikzcd}\]

We obtain the below commutative diagram:

\[\begin{tikzcd}
  K \ar["\beta"]{dd} & \ar[swap, "\alpha^\op"]{l} P &
  \\
  & & M \ar[swap, "\pi^\op"]{lu} \ar[blue, "\psi^\op"]{ld}
  \\
  N & \ar[blue, "i^\op"]{l} T &
\end{tikzcd}\]

Since \(\alpha^1\), \(\pi^0\), \(i_0\), \(\psi_1\)
are induced by \(\alpha\), \(\pi\), \(i\), \(\psi\),
and \(g_\bullet\) and \(g^\bullet\) are transitions between resolutions,
the composition of the above black arrows induces \(\partial'(\beta)\)
and
the composition of the below blue arrows induces \(\partial(\Id_M)\).

Therefore, \(\partial(\Id_M) = \partial'(\beta)\).
\qedsq