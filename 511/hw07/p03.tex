\section*{Problem 3}

Give an example of an integral domain \(R\) of dimension \(\ge 2\),
where the first theorem fails:
namely \(M\) is a free of finite rank,
and \(N \subset M\) is a submodule,
but \(N\) is not free.

\subsection*{Proof}

\(R = \bbZ[x]\).
It's not a PID, because the ideal \((2, x)\) is not a principal ideal.
(Instead, it's a UFD because a polynomial ring of UFD is UFD.)

Let \(M = R\) be a \(R\)-module.
Since \(M = R \simeq R^1\), \(M\) is free of rank 1.

Take \(N = (2, x)\).
Since it's an finitely generated ideal of \(M\), \(N\) is a submodule of \(M\).

Let's show that \(N\) is not free.

Suppose that \(N\) is free.
Note that free modules should have a basis.
Let \(\calB\) be a basis of \(N\).
Since \(N \neq 0\), \(\calB \neq \emptyset\) trivially.

Suppose that \(\calB = \{b\}\).
If \(\deg b \ge 1\), there is no \(r \in \bbZ[x]\) such that \(rb = 2\)
since \(rb = 0\) or \(\deg rb = \deg r + \deg b \ge \deg b = 1 > 0 = \deg 2\).
If \(\deg b = 0\), \(b = 2n\) for some \(n \in \bbZ\).
However, in this case, there is no \(r \in \bbZ[x]\) such that \(rb = x\), because every coefficient of \(rb\) is multiplications by \(2\).
Therefore, \(|\calB| \ge 2\).

As above, if \(\calB\) contains only elements of degree greater than 0, or if \(\calB\) contains only elements of degree 0, \(\calB\) cannot generate \(N\).
Thus, there are \(b_1, b_2 \in \calB\) such that \(\deg b_1 = 0\), \(\deg b_2 > 0\).
However, in this case,
\(b_1 \cdot b_2 - b_2 \cdot b_1 = 0\).
This shows that \(\calB\) is not linearly independent.

Therefore, there is no basis \(\calB\) of \((2, x)\) as an \(\bbZ[x]\)-module.
It means, the ideal \((2, x)\) cannot be free.
\qedsq
