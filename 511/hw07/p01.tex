\section*{Problem 1}

Prove:
\begin{theorem} (PBW; Poincar\'e-Birkhoff-Witt)
  Let
  \(F\) be a field,
  \(L\) be an \(F\)-Lie algebra with a basis \(\calB\).
  Give a well-ordering on \(\calB\).

  A canonical monomial over \(\calB\) is a sequence \((x_1, \cdots, x_r)\)
  with \(x_1 \le \cdots \le x_r\), \(x_i \in \calB\).
  For the natural map \(i: L \to U(L)\), define
  \(i(x_1, \cdots, x_r) := i(x_1) \cdots i(x_r)\).
  (\(i()\) is an embedded image of \(1_F \in F = T^0(L)\) to \(U(L)\))

  Then \(i\) is injective on the set of all canonical monomials,
  and the images form an \(F\)-basis of \(U(L)\).
\end{theorem}

\subsection*{Proof}

Notation: overline means embedded image from some \(T^k(L)\) into the \(U(L)\).
\(i\)-th entry of \(U(L)\) means the part of \(T^i(L)\) of \(U(L)\).
Adjacent entry of \(i\)-th entry means the part of \(T^{i+1}(L)\) or \(T^{i-1}\).
Higher entry than \(i\)-th means the part of \(T^{k}(L)\) where \(k > i\).
Lower entry than \(i\)-th means the part of \(T^{k}(L)\) where \(k < i\).
Index of the \(i\)-th entry of \(U(L)\) is \(i\).
Tensor product term means an expression which has a form of
\(a_1 \otimes \cdots \otimes a_n\).
The \(l\)-th item of a tensor product term \(a_1 \otimes \cdots \otimes a_n\)
is \(a_l\).

Note, \(U(L) = T(L) / I(L)\)
where \(I(L)\) is an ideal generated by \(x \otimes y - y \otimes x - [x, y]\) for \(x, y \in L\).
In other words, in \(U(L)\), \(\overline{[x, y]} = \overline{x \otimes y - y \otimes x}\).
\(i: L \to U(L)\) is a map such that \(x \mapsto x \in T_F^1(L) = L\).
\br
\noindent
(1) \(i\) is injective.

Before the proof, for \(U(L)\),
\begin{itemize}
\item
  \(\bfx = \overline{(0, \cdots, 0, x_1 \otimes \cdots \otimes x_r, 0, \cdots)}\) is 0 if \(x_1 \otimes \cdots \otimes x_r\) is zero.
  Because non-zero elements of \(I(L)\) contain at least two non-zero entries,
  \((0, \cdots, 0, x_1 \otimes \cdots \otimes x_r, 0, \cdots)\)
  cannot be in \(I(L)\) if \(x_1 \otimes \cdots \otimes x_r \neq 0\).
  Thus, the only possible case to make \(\bfx\) zero is
  making \(x_1 \otimes \cdots \otimes x_r\) zero.
  Since it's a tensor product over a field \(F\),
  \(\bfx = 0\) iff at least one of \(x_k\) is zero.
\item
  Let \(\bfx = \overline{(0, \cdots, 0, x_1 \otimes \cdots \otimes x_r, 0, \cdots)}\),
  and \(\bfy = \overline{(0, \cdots, 0, y_1 \otimes \cdots \otimes y_r, 0, \cdots)}\).
  To make \(\bfx - \bfy\) be zero,
  \(r\)-th entry of \(\bfx - \bfy\),
  which is
  \(x_1 \otimes \cdots \otimes x_r - y_1 \otimes \cdots \otimes y_r\)
  should be zero.
  To satisfy this condition, there are only two possible cases,
  which come from the equivalence relation of tensor product:
  (1) One of \(x_j\) and one of \(y_k\) are zero. In this case,
  each \(\bfx\) and \(\bfy\) become zero and the difference is
  also zero;
  (2) \(x_k = y_k\) for every \(k\). Because,
  To merge two tensor products
  \(x_1 \otimes \cdots \otimes x_r\)
  and
  \(y_1 \otimes \cdots \otimes y_r\),
  one entry \(x_k\) and \(y_k\) should be equal.
  Then,
  \(x_1 \otimes \cdots \otimes x_r
  + y_1 \otimes \cdots \otimes y_r
  = (x_1 \otimes \cdots x_{k - 1} - y_1 \otimes \cdots y_{k - 1})
  \otimes x_k
  \otimes (x_{k + 1} \otimes \cdots x_{r} - y_{k + 1} \otimes \cdots y_{ra})\).
  If \(r - 2\) more terms are same, we can repeat above process and obtain:
  \(x_1 \otimes \cdots \otimes x_{l - 1} \otimes (x_l - y_l) \otimes x_{l + 1} \otimes \cdots \otimes x_r\).
  Then, if \(x_l - y_l = 0\), this become zero.
  Otherwise, that cannot be zero.
  Therefore, \(\bfx = \bfy\) iff \(x_k = y_k\) for every \(k\).
\item
  Let \(\bfx = \overline{(0, \cdots, 0, x_1 \otimes \cdots \otimes x_r, 0, \cdots)}\),
  and \(\bfy = \overline{(0, \cdots, 0, y_1 \otimes \cdots \otimes y_s, 0, \cdots)}\) where \(r \neq s\).
  Then, \(\bfx - \bfy\) looks like \(\overline{(0, \cdots, 0, y_1 \otimes \cdots \otimes y_s, 0, \cdots, 0, x_1 \otimes \cdots \otimes x_r, 0, \cdots)}\).
  If one of \(x_j\) and one of \(y_k\) are zero, it's a zero.
  If not, we need to note below for checking zero:
  (1) \(\bfx - \bfy\) contains two non-zero entries;
  (2) The only possible case to reduce \(\bfx - \bfy\) to zero is using the relation \(x \otimes y - y \otimes x - [x, y]\).
  However, because \(x \otimes y - y \otimes x\) and \([x, y]\)
  are in two different (but adjacent) entries.
  It means, there are two different conversion:
  subtraction of two tensor product terms into one term in the lower adjacent entry,
  and convert lower entry to the subtraction of two tensor product terms in the higher adjacent entry.
\end{itemize}

Let \(r, s \in \bbZ^{\ge 0}\),
\(x_1, \cdots, x_r, y_1, \cdots, y_s \in \calB\),
and \(x_1 \le x_2 \le \cdots \le x_r\),
\(y_1 \le y_2 \le \cdots \le y_s\).
Let \(\bfx\) be an embedded image of \((x_1, \cdots, x_r)\) into \(U(L)\),
and \(\bfy\) be an embedded image of \((y_1, \cdots, y_s)\) into \(U(L)\).

First, suppose that \(r = s\).
Then, \(\bfx - \bfy = 0\) iff for every \(x_k = y_k\) for each \(k\).
Thus \(\bfx = \bfy\) iff \((x_1, \cdots, x_r) = (y_1, \cdots, y_s)\).

Next, suppose that \(r \neq s\).
WLOG, let's assume that \(r > s\).
Then,
\[\bfx - \bfy
= \overline{(0, \cdots, 0, -y_1 \otimes \cdots \otimes y_s, 0, \cdots, 0, x_1 \otimes \cdots \otimes x_r, 0, \cdots)}\]
To convert \(r\)-th entry to \(r-1\)-th entry,
there must be some \(k \in \{1, \cdots, r - 1\}\) such that
\((x_1 \otimes \cdots \otimes x_r)\),
\(x_k \otimes x_{k + 1} = a \otimes b - b \otimes a\)
for some \(a, b \in L\). (If so, we can reduce it as \([a, b]\)).
However, it's impossible.
(\(\because\)
 To reduce a subtraction of two tensor product terms into a tensor product term,
 If \(a \otimes b - b \otimes a\) can be reduced to some \(c \otimes d\),
 \(a = fb\) or \(b = fa\) for some \(f \in F\).
 WLOG suppose that \(a = fb\). Then,
 \(a \otimes b - b \otimes a = f(b \otimes b) - f (b \otimes b) = 0\).
 However, \(x_k \otimes x_{k + 1} \neq 0\) since it's a tensor product of
 basis which are non-zero.)
Thus, we cannot convert \(\bfx\)'s \(r\)-th entry to some lower entry.
This show that \(\bfx \neq \bfy\).

Therefore, \(i\) maps different elements of canonical basis into different elements of \(U(L)\).
\br
\noindent
(2) The image of \(i\) generates \(U(L)\)

The element of \(U(L)\) is a finite sum of embedded image of
elements of \(T^k(L)\) into \(U(L)\).
In other words, every non-zero element \(\bfx\) of \(U(L)\)
can be represented as,
\[\bfx = \overline{f_0} + \sum_{j=1}^{n} \overline{ \bigotimes_{k=1}^{m^j} f_{j,k} x_{j,k} }\]
for \(n, m_j \in \bbZ^{\ge 0}\), \(f_{j,k} \in F\) and \(x_{j, k} \in L\).
Then, for \(f_j = \prod_{k=1}^{m^j} f_{j,k}\),
\[\bfx = \sum_{j=0}^{n}  \overline{ f_j \bigotimes_{k=1}^{m^j} x_{j,k} }\]
by the property of the tensor product.
(\(m_0 = 0\). \(\overline{\bigotimes_{k=1}^{0} x_{j,k}} = \overline{1_F} = (1_F, 0, \cdots)\))
Then, there may be some reverse ordered \(x_{j, k}\) (where \(j\) satisfies \(m_j \ge 2\)),
i.e. \(x_{j, i} > x_{j, i+1}\).
We know that,
\[
  \overline{
  x_{j,1} \otimes \cdots \otimes x_{j,i-1} \otimes
  (x_{j,i} \otimes x_{j,i+1} - x_{j,i+1} \otimes x_{j,i} - [x_{j,i}, x_{j,i+1}]) \otimes x_{j,i+2} \otimes \cdots \otimes x_{j,m_j}
  }
  \in I
\]
Thus,
\begin{align*}
  &\overline{
  x_{j,1} \otimes \cdots \otimes x_{j,i-1} \otimes
  x_{j,i} \otimes x_{j,i+1} \otimes x_{j,i+2} \otimes \cdots \otimes x_{j,m_j}
  }
  \\=& - \overline{
  x_{j,1} \otimes \cdots \otimes x_{j,i-1} \otimes
   x_{j,i+1} \otimes x_{j,i} \otimes x_{j,i+2} \otimes \cdots \otimes x_{j,m_j}
  }
  \\& - \overline{
  x_{j,1} \otimes \cdots \otimes x_{j,i-1} \otimes
  [x_{j,i}, x_{j,i+1}] \otimes x_{j,i+2} \otimes \cdots \otimes x_{j,m_j}
  }
\end{align*}
Using this relation,
we can change an order of some two adjacent tensor product terms
in the \(j\)-th summand.
Therefore, by repeating reordering from the highest entry,
we can assume that \(x_{j, i} \le x_{j, i+1}\) for every \(j, i\).
(This proecss halts because \(\bfx\) is a finite sum.)

Then,
\begin{align*}
  \bfx
  &= \sum_{j=0}^{n} \overline{ f_j \bigotimes_{k=1}^{m^j} x_{j,k}}
  \\&= \sum_{j=0}^{n} f_j \overline{\bigotimes_{k=1}^{m^j} x_{j,k} }
  \\&= \sum_{j=0}^{n} f_j i \left(x_{j,1}, \cdots, x_{j,m_j} \right)
\end{align*}
Therefore, the image of \(i\) generates \(U(L)\).
\br
\noindent
(3) The image of \(i\) is linearly independent

Let \(\bfx \in U(L)\).
Suppose that
\begin{align*}
  \bfx
  &= \sum_{j=1}^m a_j i(x_{j,1},\cdots,x_{j,m_j})
  \\&= \sum_{j=1}^n b_j i(y_{j,1},\cdots,y_{j,n_j})
\end{align*}
for \(m, m_j, n, n_j \in \bbZ^{\ge 0}\), \(a_j, b_j \in F \bs \{0_F\}\), \(x_{j, k}, y_{j, k} \in L\), \(x_{j,k} \le x_{j,k+1}\), \(y_{j,k} \le y_{j,k+1}\).

By reordering and inserting some summand with zero coefficients,
we can make \(x_{j,k} = y_{j,k}\) and \(a_j, b_j \in F\).
Thus, let's assume that
\begin{align*}
  \bfx
  &= \sum_{j=1}^n a_j i(x_{j,1},\cdots,x_{j,n_j})
  \\&= \sum_{j=1}^n b_j i(x_{j,1},\cdots,x_{j,n_j})
\end{align*}
where \(i(x_{j,1},\cdots,x_{j,n_j})\) are distinct.
Then,
\[0 = \sum_{j=1}^n (a_j - b_j) i(x_{j,1},\cdots,x_{j,n_j})\]
Suppose that \(a_j \neq b_j\) for some \(j\).
Let \(c_j = a_j - b_j\).
Then, some \(c_j\) may be zero.
By removing zero coefficient summands and reordering,
\[0 = \sum_{j=1}^{n'} c_j i(x_{j,1},\cdots,x_{j,n'_j}) \quad\cdots(*)\]
where \(c_j\) are non-zero.
Let \(p\) be the highest index of non-zero entry of \((*)\).
If \(p\)-th entry of \((*)\) is just a single non-zero coefficient tensor product term,
as we showed above,
it cannot be reduced to the lower entry.
It means, \((*)\) is non-zero.
Suppose that \(p\)-th entry of \((*)\) is a sum of multiple non-zero coefficient tensor product term.
However, if they don't have at least \(p - 1\) common items for tensor product, the sum cannot be reduced to a single tensor product term.
In this case, the sum cannot be zero.
Suppose that each summands of \(p\)-th entry of \((*)\) have
common items of tensor products except \(l\)-th item.
Since \((x_{j,1},\cdots,x_{j,n'_j})\) are distinct,
\(l\)-th items of each summands of \(p\)-th entry of \((*)\) are distinct.
Then, the \(p\)-th entry of \((*)\) is reduced into
\[x_1 \otimes \cdots \otimes x_{k-1} \otimes y \otimes x_{k+1} \otimes \cdots \otimes x_p\]
where each \(x_j\) are common items of each tensor product term in the \(p\)-th entry of \((*)\), and \(y\) is a linear combination of \(l\)-th items of each sumamnds of \(p\)-th entry of \((*)\).
Since \(y\) is a linear combination of distinct elements of \(\calB\) with non-zero coefficient, \(y\) is non-zero.
Also, each tensor product summand of the \(p\)-th entry of \((*)\)
cannot be reduced into the \(p - 1\)-th entry of \((*)\).
(\(\because\)
 If two distinct \((x_1, \cdots, x_i, x_{i+1}, \cdots, x_p)\)
 and \((x_1, \cdots, x_{i+1}, x_i, \cdots, x_p)\)
 are included in the linear combination of \(p\)-th entry,
 we may use \(a \otimes b - b \otimes a = [a, b]\)
 to reduce the \(p\)-th entry into \(p-1\)-th entry.
 However, in this case,
 since \(x_k\) must be ordered by \(\le\),
 in this case,
 \(x_i \le x_{i+1}\) and \(x_{i+1} \le x_i\).
 Then,
 \(x_i = x_{i+1}\).
 However, each entry is a linear combination of different tensor products,
 \((x_1, \cdots, x_i, x_{i+1}, \cdots, x_p) \neq (x_1, \cdots, x_{i+1}, x_i, \cdots, x_p)\).
 It's a contradiction.)
Therefore, \((*)\) cannot be zero,
and it's a contradiction because we assumed that \((*) = 0\).

Therefore, every element of \(U(L)\) can be represented
by the unique linear combination of images of \(i\).
\qedsq