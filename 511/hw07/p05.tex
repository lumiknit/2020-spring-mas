\section*{Problem 5}

Let \(R\) be commutative ring with 1.
Regard the group of units \(R^\times\) as a \(\bbZ\)-module.
Consider the tensor algebra \(T_\bbZ(R^\times)\),
and let \(I \subset T_\bbZ(R^\times)\) be the two-sided ideal
generated by elements of the form
\(a \otimes (1 - a)\)
(where \(a, 1-a \in R^\times\)).

Consider the graded ring \(K_*^M(R) := T_\bbZ(R^\times) / I\)
and the image of the degree \(n\) part of \(T(R^\times)\)
is written \(K_n^M(R)\).
The former \(K_*^M(R)\) is called
the Milnor \(K\)-ring of \(R\),
and the latter \(K_n^M(R)\)
is called the \(n\)-th Milnor \(K\)-group of \(R\).

Let \(n \ge 2\)
and let \(R\) be a finite field.
Prove that \(K_n^M(R) = 0\).
(Hint: Recall \(R^\times\) is cyclic.)

\subsection*{proof}

Let \(n = |R^\times|\) and \(\bbn = \{0, \cdots, n - 1\}\).

Since \(R\) is a finite field, every element except 0 are unit.
For \(a \in R \setminus \{0\}\), \(1 - a = 0\) iff \(a = 1\).
Thus, \(1 - a \) is a unit if \(a \in R \setminus \{0, 1\}\).

Since \(R^\times\) is finite, cyclic, let \(R^\times = \langle g \rangle\).
Then, every element of \(R^\times\) can be represented as \(g^k\) for \(k \in \bbZ\).
Let's denote \(k \cdot g = g^k\).
Note that \(1\) in \(R^\times\) is \(0 \cdot g = g^0\).

Then, for arbitrary \(a, b \in R^\times\),
\(a = \alpha \cdot g\),
\(b = \beta \cdot g\)
and
\(a \otimes b = (\alpha \cdot g) \otimes (\beta \cdot g) = \alpha\beta (g \otimes g)\).
It means, for every element \(\tau \in R^\times \otimes_\bbZ R^\times\),
\begin{align*}
  \tau
  &= \sum_{k=1}^{m} (\alpha_k \cdot g) \otimes (\beta_k \cdot g)
  \\&= \left(\sum_{k=1}^m \alpha_k \beta_k\right) (g \otimes g)
\end{align*}
Thus, every element of \(R^\times \otimes_\bbZ R^\times\) can be represented as
\(m \cdot (g \otimes g)\) for some \(m \in \bbZ\).

Therefore, to show \(K_2^M(R)\) is zero, 
it's enough to show that \(g \otimes g\) is zero in \(K_2^M(R)\).

Let's show \(g \otimes g \in I\).
\br
\noindent
Before the proof of above claim, we note some important facts.
First, let \(-^*: R \to R\) such that \(x^* = 1 - x\).
This is a dual operator (i.e. \((x^*)^* = x\)),
since \(1 - (1 - x) = x\) for every \(x \in R\).
It means, every element of \(R\) must be paired with exactly one element of \(R\) (it can be itself) by the operator.
Also, let's denote \(R' = R^\times \setminus \{0\}\).
Then, \(-^*\) is closed in \(R'\), because \(0^* = 1\).
\br
\noindent
The first case is \(R \simeq F_{2^m}\) for some \(m \in \bbN\).
It's the only case such that \(n\) is odd.

If \(m = 1\), \(R^\times = \{1\}\).
Then, \(g = 1\), which is an identity of \(R^\times\).
Thus, \(g \otimes g \in I\) since it's a zero in the tensor algebra.

Suppose that \(m = 2\).
Then, \(R^\times = \langle g \rangle = \{g^0, g^1, g^2\}\).
Then, \((g^1)^*\) should be \(g^1\) or \(g^2\).
Thus, \(g^1 \otimes (g^1)^* = g \otimes g \text{ or } 2(g \otimes g)\),
and one of them is contained in \(I\).
If \(g \otimes g \in I\), we are done.
Since \(2 (2 (g \otimes g)) = 4 (g \otimes g) = g \otimes g\),
\(g \otimes g \in I\) if \(2 (g \otimes g) \in I\).

Suppose that \(m > 2\).
Let \(S = \{k \in \bbn \mid k \cdot g = g^a (g^a)^* \text{ for some } g^a \in R'\}\).
Let's show that the GCD of \(S\) is 1 or 2.
Suppose that for the prime number \(p \ge 3\), \(p\) divides every elements of \(S\).
In other words, for every \(g^a \in R'\) and \(g^b = (g^a)^*\),
\(ab\) should be divides by \(p\).
That means, for every \(g^a \in R'\) and \(g^b = (g^a)^*\),
\(p \mid a\) or \(p \mid b\).
Therefore, at least \(\lceil \frac{n - 1}{2} \rceil\) elements of \(\bbn \setminus \{0\}\) should be multiplications of \(p\).
However, in \(\bbn\), there are at most \(\lfloor \frac{n - 1}{p} \rfloor\)
elements which are multiplications of \(p\).
Thus, it's a contradiction.
Also, if \(p = 4\), we can make the same argument and we obtain the fact that
4 cannot divide all elements of \(S\).
Therefore, only 1 or 2 may divides all elements of \(S\) and GCD of \(S\) is 1 or 2.
Note that for every \(g^a \in R'\) and \(g^b = (g^a)^*\),
\(g^a \otimes (g^a)^* = ab(g \otimes g)\).
In other words, \(S = \{k \in \bbn \mid k (g \otimes g) = g^a \otimes (g^a)^* \text{ for some } g^a \in R'\}\).
And, GCD of \(S\) is 1 or 2 implies that by taking linear combination of \(g^a (g^a)^* \in R'\), we obtain \(g \otimes g\) or \(2 (g \otimes g)\).
In other words, \(g \otimes g \in I\) or \(2 (g \otimes g) \in I\).
If \(g \otimes g \in I\), we are done.
If \(2 (g \otimes g) \in I\),
since \(n\) is odd, for \(k = \lceil n/2 \rceil = \frac{n + 1}{2}\),
\[2k(g \otimes g) = ((n + 1) \cdot g) \otimes g = g \otimes g + (n \cdot g) \otimes g = g \otimes g \in I\]
as \(R^\times\) is a group of order \(n\).
\br
\noindent
Next, let's consider the case of \(R \simeq F_{p^m}\) for some prime \(p \ge 3\) and \(m \in \bbN\).
Let
\begin{align*}
S &= \{k \in \bbn \mid k \cdot g = g^a (g^a)^* \text{ for some } g^a \in R'\}
\\&= \{k \in \bbn \mid k (g \otimes g) = g^a \otimes (g^a)^* \text{ for some } g^a \in R'\}
\end{align*}
as the previous case.
Note that \(n\) is even and \(|R'|\) is odd.
Let's show that the GCD of \(S\) is 1.
Suppose that there is a prime number \(p\) such that \(p\) divides every element of \(S\).
Then, for every \(g^a \in R'\) and \(g^b = (g^a)^*\),
\(p \mid a\) or \(p \mid b\).
Thus, at least \(\lceil\frac{|R'|}{2}\rceil = \lceil\frac{n - 1}{2}\rceil = \frac{n}{2}\)
elements of \(\bbn \setminus \{0\}\) must be divides by \(p\).
First, if \(p = 2\), it's impossible, because, \(\bbn \setminus \{0\} = \{1, 2, \cdots, n - 1\}\) contains only \(\frac{n}{2} - 1\) even elements,
but we need at least \(\lceil\frac{n - 1}{2}\rceil = n\) elements.
Also, if \(p \ge 3\), \(\bbn \setminus \{0\}\) contains at most
\(\lfloor\frac{n - 1}{p}\rfloor\) elements of multiplication of \(p\).
Thus, we cannot have \(n\) elements of multiplication of \(p\).
Therefore, any prime numbers cannot divide every elements of \(S\).
This shows that GCD of \(S\) is 1.
In other words, we obtain \(g \otimes g\) from
some linear combination of \(g^a \otimes (g^a)^*\) of \(R'\).
Thus, \(g \otimes g \in I\).
\br
\noindent
Then, since every element of \(K_2^M(R)\) is generated by \(g \otimes g\),
\(K_2^M(R) \subseteq I\).
It implies
\(K_2^M(R) = 0\).
\br
\noindent
Let \(l \in \bbZ^{\ge 3}\) and suppose that
\(K_k^M(R)= 0\)  for every \(2 \le k < l\).
By the definition of \(T\),
all elements of \(K_l^M(R)\) is generated by the tensor product
\(a \otimes b\)
where \(a \in K_i^M(R)\) and \(b \in K_{l - i}^M(R)\)
for some \(i \in \{1, \cdots, l - 1\}\).
WLOG, suppose that \(i \ge l - i\).
Then, \(i \ge \frac{l}{2} \ge \frac{3}{2}\).
Since \(i\) is an integer, \(i \ge 2\).
Then, by induction hypothesis,
\(a = 0\) because \(a \in K_i^M(R) = 0\).
Because tensor product with \(0\) is zero, \(a \otimes b = 0\).
Therefore, \(K_l^M(R)\) is generated by \(0\)
and it implies \(K_l^M(R) = 0\).

In conclusion, for every \(l \in \bbZ^{\ge 2}\), \(K_l^M(R) = 0\).
\qedsq