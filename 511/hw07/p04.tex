\section*{Problem 4}

Let \(L\) be a Lie algebra over a field \(F\).

\begin{enumerate}[label=(\arabic*)]
\item
  For an \(F\)-vector space \(M\),
  find the natural definition of an \(L\)-module over the Lie algebra \(L\).
  (Hint: you need something similar to that of the Jacobi identity.)
\item
  Prove that an \(L\)-module \(M\) is naturally an \(U(L)\)-module
  for the associative \(F\)-algebra \(U(L)\) in the ordinary sense.
\item
  Prove that there is a natural equivalence of categories
  \(L-\Mod \leftrightarrow U(L)-\Mod\)
\end{enumerate}

\subsection*{Proof of (1)}

Note: \(M\) is an abelian group, \(F\) acts of \(M\).
\(L\) is a \(F\)-vector space.

For a ring \(R\), we know that \(R\)-module \(M\)
with an action \(\cdot: R \times M \to M\) should satisfies,
for \(r, s \in R\), \(m, n \in M\),
\[r \cdot (s \cdot m) =  (rs) \cdot m\]
\[(r + s) \cdot m = r \cdot m + s \cdot m\]
\[r \cdot (m + n) = r \cdot m + r \cdot n\]

The below two are linearly, and the first one is associativity.

Let \(L\) be a Lie algebra over a field \(F\)
and \(M\) be an \(F\)-vector space.
Then, if \(M\) can be called as a \(L\)-module,
it may need to satisfy linearlity and associativity.

However, \(L\) is not associative.
More specifically, there is a Jacobi identity: for \(\bfj, \bfk, \bfl \in L\),
\[[\bfj, [\bfk, \bfl]] + [\bfk, [\bfl, \bfj]] + [\bfl, [\bfj, \bfk]] = 0\]
Suppose that \(\bfm \in M\) and \(\cdot: L \times M \to M\) is an action.
We want to replace \(\bfl\) to \(\bfm\).
To achieve this, make above terms into the form of \([-, \bfl]\).
\begin{align*}
  0 =
  [\bfj, [\bfk, \bfl]] + [\bfk, [\bfl, \bfj]] + [\bfl, [\bfj, \bfk]]
  &= [\bfj, [\bfk, \bfl]] + [\bfk, -[\bfj, \bfl]] - [[\bfj, \bfk], \bfl]
  \\&= [\bfj, [\bfk, \bfl]] - [\bfk, [\bfj, \bfl]] - [[\bfj, \bfk], \bfl]
\end{align*}
Then, change \(\bfl\) to \(\bfm \in M\)
and \([-, \bfn]\) into \(- \cdot \bfn\) for any \(\bfn \in M\):
\begin{align*}
  0
  &= [\bfj, [\bfk, \bfl]] - [\bfk, [\bfj, \bfl]] - [[\bfj, \bfk], \bfl]
  \\&= [\bfj, \bfk \cdot \bfm] - [\bfk, \bfj \cdot \bfm] - [\bfj, \bfk] \cdot \bfm
  \\&= \bfj \cdot (\bfk \cdot \bfm) - \bfk \cdot (\bfj \cdot \bfm) - [\bfj, \bfk] \cdot \bfm
\end{align*}
By add \([\bfj, \bfk] \cdot \bfm\) to the both side, we obtain
\[[\bfj, \bfk] \cdot \bfm = \bfj \cdot (\bfk \cdot \bfm) - \bfk \cdot (\bfj \cdot \bfm)\]
This is a property looks like associativity for \(L\)-module.

Linearity is almost same as the \(R\)-module.

Thus, we can say that
a \(F\)-vector space \(M\) is a \(L\)-module
for an action \(\cdot: L \times M \to M\)
if it satisfies
\[[\bfk, \bfl] \cdot \bfm = \bfk \cdot (\bfl \cdot \bfm) - \bfl \cdot (\bfk \cdot \bfm)\]
\[(\bfk + \bfl) \cdot \bfm = \bfk \cdot \bfm + \bfl \cdot \bfm\]
\[\bfl \cdot (\bfm + \bfn) = \bfl \cdot \bfm + \bfl \cdot \bfn\]
\[f (\bfl \cdot \bfm) = (f \bfl) \cdot \bfm = \bfl \cdot (f \bfm)\]
for every \(f \in F\), every \(\bfk, \bfl \in L\) and every \(\bfm, \bfn \in M\).

\subsection*{Proof of (2)}

Suppose that some structure \(A\) over a field \(F\) acts on some module \(M\) with \(\cdot: A \times M \to M\).
In this case, we want to extend the action to
\(*: (A \otimes_F A) \times M \to M\).
Maybe, the most natural way to defined \(*\) is,
\[(a \otimes b) * m = a \cdot (b \cdot m) \]
for each generator \((a \otimes b)\) of \(A \otimes_F A\).
Since \(\otimes\) and \(\cdot\) satisfies linearity, \(*\) also have
linearity.
But tensor product itself does not have any product between them,
thus we cannot say anything about associativity of \(*\).

However, in the tensor algebra \(T_F(A)\),
we have a concatenation product.
Let \(a_1 \otimes \cdots \otimes a_n \in T_F^n(A)\)
and \(\overline{a_1 \otimes \cdots \otimes a_n}\) be an embedded image of \(a_1 \otimes \cdots \otimes a_n\)
in \(T_F(A)\).
In this case, let 
\[\overline{a_1 \otimes \cdots \otimes a_n} * m = a_1 \cdot ( \cdots (a_n \cdot m))\]
Then,
for \(a_1 \otimes \cdots \otimes a_n \in T_F^n(A)\)
and \(b_1 \otimes \cdots \otimes b_l \in T_F^l(A)\),
\begin{align*}
  \overline{a_1 \otimes \cdots \otimes a_n} * (\overline{b_1 \otimes \cdots \otimes b_l} * m)
  &= a_1 * (a_2 * (\cdots (a_n * (b_1 * (\cdots (b_l * m))))))
  \\&=
  \overline{a_1 \otimes \cdots \otimes a_n \otimes b_1 \otimes \cdots \otimes b_l} * m
  \\&=
  \left(\overline{a_1 \otimes \cdots \otimes a_n} \cdot \overline{b_1 \otimes \cdots \otimes b_l} \right) * m
\end{align*}
Thus, this action is associative as an action of associative structure.
\br
\noindent
Let \(L\) be a Lie algebra, and \(M\) be a \(L\)-module (See (1)).
Note that each element of \(U(L)\) can be represented as:
\[\sum_{j=1}^{n} \overline{\bigotimes_{k=1}^{l_j} a_{j,k}}\]
for \(n, l_j \in \bbZ^{\ge 0}\), \(a_{j,k} \in L\).
(If \(l_j = 0\), \(\bigotimes_{k=1}^{l_j} a_{j,k}\) is some element of \(F = T^0(L)\).
Overline means embedded image into the \(U(L)\).)
Then, let's define an action of \(U(L)\) on \(M\) as,
\[\left(\sum_{j=1}^{n} \overline{\bigotimes_{k=1}^{l_j} a_{j,k}}\right) \cdot m
= \sum_{j=1}^{n} (a_{j,1} \cdot (\cdots ( a_{j,l_j} \cdot m)))\]

First, we should check this is well-defined.
Note that there are some equivalence relation of \(U(L)\):
\begin{itemize}
\item
  \(a \otimes b + a \otimes c = a \otimes (b + c)\).
  For \(m \in M\),
  \begin{align*}
    \overline{a \otimes b + a \otimes c} \cdot m
    &= \overline{a \otimes b} \cdot m + \overline{a \otimes c} \cdot m
    \\&= a \cdot (b \cdot m) + a \cdot (c \cdot m)
    \\&= a \cdot (b \cdot m + c \cdot m)
    \\&= a \cdot (b \cdot m + c \cdot m)
    \\&= a \cdot ((b + c) \cdot m)
    = \overline{a \otimes (b + c)} \cdot m
  \end{align*}
\item
  \(a \otimes c + b \otimes c = (a + b) \otimes c\).
  In the similar way above, we can show that
  \[\overline{a \otimes c + b \otimes c} \cdot m
  = \overline{(a + b) \otimes c} \cdot m\]
  for \(m \in M\).
\item
  \(f (a \otimes b) = (fa) \otimes b = a \otimes (fb)\) for \(f \in F\).
  Then, by linearity of action of \(L\) on \(M\), for any \(m \in M\),
  \begin{align*}
    \overline{(fa) \otimes b} \cdot m
    &= fa \cdot (b \cdot m)
    \\&= f (a \cdot (b \cdot m))
    = f (\overline{a \otimes b} \cdot m)
    \\&= a \cdot (fb \cdot m)
    \\&= \overline{a \otimes fb} \cdot m
  \end{align*}
\item
  \(\overline{a \otimes b - b \otimes a - [a, b]} = 0\).
  In this case, Jacobi-identity-like associativity of \(L\)-action helps us:
  for \(m \in M\),
  \begin{align*}
    \overline{a \otimes b - b \otimes a - [a, b]} \cdot m
    &= \overline{a \otimes b} \cdot m - \overline{b \otimes a} \cdot m - \overline{[a, b]} \cdot m
    \\&= a \cdot (b \cdot m) - b \cdot (a \cdot m) - [a, b] \cdot m
    \\&= [a, b] \cdot m - [a, b] \cdot m
    = 0
  \end{align*}
\end{itemize}
Therefore, by properties of tensor product and pseudo associativity(?)
of \(L\)-action, the \(U(L)\)-action is well-defined.

Then, it satisfies all properties of associative algebra action:
\begin{itemize}
\item
  \(\bfa \cdot (\bfb \cdot \bfm) = (\bfa \bfb) \cdot \bfm\)
  for \(\bfa, \bfb \in U(L)\), \(\bfm \in M\).
  It holds because as we shown at the top of the proof,
  for the concatenation product of \(U(L)\),
  this kind of associativity hold.
\item
  \((\bfa + \bfb) \cdot \bfm = \bfa \cdot \bfm + \bfb \cdot \bfm\) 
  for \(\bfa, \bfb \in U(L)\), \(\bfm \in M\).
  This is directly from the definition of the action.
  (Since the \(U(L)\)-action takes a sum of embedded images of tensor products
  into a sum of value of module which obtained from \(L\)-action.)
\item
  \(\bfa \cdot (\bfm + \bfn) = \bfa \cdot \bfm + \bfa \cdot \bfn\) 
  for \(\bfa \in U(L)\), \(\bfm, \bfn \in M\).
  Let \(\bfa = \sum_{j=1}^{n} \overline{\bigotimes_{k=1}^{l_j} a_{j,k}}\).
  Then,
  \begin{align*}
    \bfa \cdot (\bfm + \bfn)
    &= \left(\sum_{j=1}^{n} \overline{\bigotimes_{k=1}^{l_j} a_{j,k}}\right) \cdot (\bfm + \bfn)
    \\&= \sum_{j=1}^{n} \overline{\bigotimes_{k=1}^{l_j} a_{j,k}} \cdot (\bfm + \bfn)
    \\&= \sum_{j=1}^{n} a_{j,1} ( \cdots ( a_{j,l_j}\cdot (\bfm + \bfn))) 
    \\&= \sum_{j=1}^{n} a_{j,1} ( \cdots ( a_{j,l_j-1} \cdot ( a_{j,l_j} \cdot \bfm + a_{j,l_j} \cdot \bfn)))
    \\&= \cdots
    \\&= \sum_{j=1}^{n} \left( a_{j,1} ( \cdots ( a_{j,l_j}\cdot \bfm)) + a_{j,1} ( \cdots ( a_{j,l_j}\cdot \bfn))  \right)
    \\&= \sum_{j=1}^{n} \overline{\bigotimes_{k=1}^{l_j} a_{j,k}}  \cdot \bfm + \sum_{j=1}^{n} \overline{\bigotimes_{k=1}^{l_j} a_{j,k}} \cdot \bfn
    \\&= \bfa \cdot \bfm + \bfa \cdot \bfn
  \end{align*}
\item
  \(f(\bfa \cdot \bfm) = f \bfa \cdot \bfm = \bfa \cdot f\bfm\)
  for \(f \in F\), \(\bfa \in U(L)\), \(\bfm \in M\).
  It can be shown as:
  first, express \(f \bfa \cdot \bfm\) as a sum of form of
  \(f(a_1 \otimes \cdots \otimes a_p) \cdot \bfm\),
  then it's,
  \(f(a_1 \cdot (\cdots (a_p \cdot \bfm)))\) by the definition,
  use the property of \(L\)-action such that
  \(f \bfl \cdot \bfm = \bfl \cdot f \bfm\)
  \(p\)-times,
  then we obtain
  \((a_1 \cdot (\cdots (a_p \cdot f \bfm)))\)
  and it can be resoted into
  \(\bfa \cdot f\bfm\)
\end{itemize}

Thus, we can extend \(L\)-module \(M\) to the \(U(L)\)-module.

Note that this is a unique extension of \(L\)-module to the \(U(L)\)-module.
Because, if \(F: L-\Mod \to U(L)-\Mod\) is a extension,
\(F\) should preserve all objects
and \(F\) preserve the behavior of action for the elements of
\(T_F^1(L) = L\) on \(M\).
Then, because of associativity,
for \(a, b \in L\),
\(\overline{a} \overline{b} = \overline{a \otimes b} \in T_F^2(L)\) holds,
and \(\overline{a} \cdot (\overline{b} \cdot m) = (\overline{a} \overline{b}) \cdot m = \overline{a \otimes b} \cdot m\)
should holds.
Repeating this process, we obtain the above definition of action.

The proof of naturality is in the next part.

\subsection*{Proof of (3)}

Let \(F: L-\Mod \to U(L)-\Mod\) be a functor
such that \(F\) maps each \(L\)-module into the \(U(L)\)-module
preserving every object and morhpisms,
but just changing the action as in the proof of (2).

In other words,
for every \(M, N \in L-\Mod\)
and every homomorhpism \(\varphi: M \to N\),
\(F(M) = M\) in the sense of a set
and
\(F(\varphi) = \varphi\)
in the sense of a set map.

First, \(F\) is well-defeined as a map of objects,
because we showed that for every \(L\)-module \(M\),
\(M\) is a \(U(L)\)-module.

Also, \(F\) is essentially surjective.
Let \(M\) be an \(U(L)\)-module.
Then, we can define a \(L\)-module action as,
\[\bfl \cdot \bfm = \overline{\bfl} \cdot \bfm\]
where \(\bfl \in L\), \(\bfm \in M\) and overline means the embeded image into the \(U(L)\).
Then, we can easily show that this action satisfies all properties which 
\(L\)-module must satisfy:
linearity is trivial,
\([\bfk, \bfl] \cdot \bfm = \bfk \cdot (\bfl \cdot \bfm) - \bfl \cdot (\bfk \cdot \bfm)\) is hold because
\(\overline{[\bfk, \bfl]} = \overline{\bfk \otimes \bfl - \bfl \otimes \bfk}\)
and \(\overline{\bfk \otimes \bfl} \cdot \bfm = \bfk \cdot (\bfl \cdot \bfm)\).
Thus, \(M\) is in \(L-\Mod\).
And in this case, \(F(M)\) is the given \(M\), because we showed that
\(L\)-module is extended to the \(U(L)\)-module in the unique way.
This shows that \(F\) is essentially surjective.

\(F\) as a map of morphisms is well-defined.
It's because \(U(L)\)-action is just a sum of results of \(L\)-actions.
More specifically, let \(\varphi: M \to N\) be an \(L\)-module homomorphism.
Then, for \(m, n \in M\) \(\varphi(m + n) = \varphi(m) + \varphi(n)\) holds.
For \(m \in M\) and \(\sum_{j=1}^{n} \overline{\bigotimes_{k=1}^{l_j} a_{j,k}} \in U(L)\)
\begin{align*}
  \varphi\left(\left(\sum_{j=1}^{n} \overline{\bigotimes_{k=1}^{l_j} a_{j,k}}\right) \cdot m\right)
  &= \varphi\left(\sum_{j=1}^{n} a_{j, 1} \cdot ( \cdots ( a_{j,l_j} \cdot m ))\right)
  \\&= \sum_{j=1}^{n} a_{j, 1} \cdot ( \cdots ( a_{j,l_j} \cdot \varphi(m)))
  \\&= \left(\sum_{j=1}^{n} \overline{\bigotimes_{k=1}^{l_j} a_{j,k}}\right) \cdot \varphi(m)
\end{align*}
Thus, every \(L\)-module homomorphism is a \(U(L)\)-module homomorphism.
Suppsoe that \(\varphi: M \to N\) is an \(U(L)\)-module homomorphism.
Then, for \(m, n \in M\) \(\varphi(m + n) = \varphi(m) + \varphi(n)\) holds trivially.
For \(m \in M\) and \(l \in L\),
\[
  \varphi(l \cdot m) = \varphi(\overline{l} \cdot m)
  = \overline{l} \cdot \varphi(m) = l \cdot \varphi(m)
\]
Thus every \(U(L)\)-module homomorphism is a \(L\)-module homomorphism.

Also, the above shows that there is 1-1 correspondence between
\(L\)-module homomorphism and \(U(L)\)-module homomorphism.
Therefore, \(F\) is fully faithful.

Then, this \(F\) is a functor because,
\begin{itemize}
\item \(F(\Id_M) = \Id_{F(M)}\).
  Note \(F\) preserved every object and morphism.
  Thus, image of identity function by \(F\) is an identity function.
  (\(F(\Id_M) = \Id_M = \Id_{F(M)}\))
\item \(F(f \circ g) = F(f) \circ F(g)\).
  Composition of homomorphisms is same to the composition of set functions.
  Since \(F\) preserved morhpisms,
  \(F(f \circ g) = f \circ g = F(f) \circ F(g)\)
  holds.
\end{itemize}

Therefore, we find a fully faithful and essentially surjective functor
from \(L-\Mod\) to \(U(L)-\Mod\).
Therefore, \(F\) is an equivalenec between \(L-\Mod\) and \(U(L)-\Mod\).
\qedsq