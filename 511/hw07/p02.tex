\section*{Problem 2}

\begin{theorem}
  Let
  \(R\) be PID,
  \(M\) be finitely generated \(R\)-modules.
  Then
  \(M \simeq R^r \oplus R/(a_1) \oplus \cdots \oplus R/(a_m)\).
  for some \(a_i \in R\) such that \(a_1 \mid a_2 \mid \cdots \mid a_m\).
  The number \(r\) is unique and \(a_1, \cdots, a_m\) are uniquely
  decided up to units in \(R\).
\end{theorem}

Let \(x_1, \cdots, x_n \in M\) be a set of generators of \(M\) as an \(R\)-Mod.

Then, we can choose a generating set such that \(n\) to be minimum.

Define a surjective \(R\)-mod homomorphism \(\phi: R^n \to M\)
given by \((b_1, \cdots, b_n) \mapsto \sum_i b_ix_i\).
Here \(\ker \phi \subset R^n\) is a submodule of free module \(R^n\)
over the PID \(R\),
by the previous theorem,
it is free of rank \(\le n\), say \(m\).

Also by the previous theorem,
we can choose a basis \(y_1, \cdots, y_n \in R^n\)
and \(a_1 \mid \cdots \mid a_m \in R\)
such that \(a_1y_1, \cdots, a_my_m \in \ker \phi\)
and this set gives a basis for \(\ker \phi\).
Then
\begin{align*}
  M \simeq R^n / \ker\phi
  &= (Ry_1 \oplus \cdots \oplus Ry_n) / (Ra_1y_1 \oplus \cdots \oplus Ra_my_m \oplus 0 \cdots)
  \\&\simeq R/(a_1) \oplus \cdots \oplus R/(a_m) \oplus R^{n - m}
\end{align*}

Show that the uniqueness part.

\subsection*{Proof}

Let \(\Tor(M)\) be a torsion \(R\)-submodule of \(M\).
In other words,
\(\Tor(M) = \{m \in M \mid \exists r \in R \setminus \{0\}: rm = 0\}\).

Also, let's denote \(a \sim b\) is \(a = ub\) for some unit \(u \in R\).

Suppose that there are
\(r, s \in \bbZ^{\ge 0}\)
and \(a_1 \mid a_2 \mid \cdots \mid a_m \in R\),
\(b_1 \mid \cdots \mid b_n \in R\)
such that
\begin{align*}
  M &\simeq R^r \oplus R/(a_1) \oplus \cdots \oplus R/(a_m)
  \\&\simeq R^s \oplus R/(b_1) \oplus \cdots \oplus R/(b_n)
\end{align*}
For convinience, let's denote
\[A_1 = R^r \oplus R/(a_1) \oplus \cdots \oplus R/(a_m)\]
\[A_2 = R^s \oplus R/(b_1) \oplus \cdots \oplus R/(b_n)\]

Since \(M \simeq A_1 \simeq A_2\),
\(\Tor(M) \simeq \Tor(A_1) \simeq \Tor(A_2)\).
And, \(M/\Tor(M) \simeq A_1 / \Tor(A_1) \simeq A_2 / \Tor(A_2)\).

Since \(R\) is a PID (thus ID), there is no zero divisors in \(R\).
In other words, for every nonzero \(r \in R\), \(rs \neq 0\) for every \(s \in R \setminus \{0\}\).
Then, for \(\bfx \in A_1\) such that
\[\bfx = (x_1, \cdots, x_r, \overline{y_1}, \cdots, \overline{y_m})\]
by multiplicating by \(a_m\)
\begin{align*}
  a_m\bfx
  &= (a_m x_1, \cdots, a_m x_r, a_m\overline{y_1}, \cdots, a_m\overline{y_m})
  \\&= (a_m x_1, \cdots, a_m x_r, 0, \cdots, 0)
\end{align*}
since \(a_k \mid a_m\) for each \(k = 1, \cdots, m\).
Also, each \(r x_k\) is non-zero for \(r \in R \setminus \{0\}\) if \(x_k\) is non-zero.
Thus,
\[\Tor(A_1)
  = 0^r \oplus R/(a_1) \oplus \cdots \oplus R/(a_m)
  \simeq R/(a_1) \oplus \cdots \oplus R/(a_m)\]
\[A_1 / \Tor(A_1)
  = R^r \oplus 0 \oplus \cdots \oplus 0
  \simeq R^r\]
In the same way,
\[\Tor(A_2) \simeq R/(b_1) \oplus \cdots \oplus R/(b_n) \]
\[A_2 / \Tor(A_2) \simeq R^s\]

First, \(R^r \simeq A_1 / \Tor(A_1) \simeq A_2 / \Tor(A_2) \simeq R^s\).
Then, the free rank of \(R^r\), \(r\), should be equal to
the free rank of \(R^s\), \(s\).
Thus, \(r = s\).
\br
\noindent
Next,
\begin{align*}
  A'_1 = R/(a_1) \oplus \cdots \oplus R/(a_m)
  &\simeq \Tor(A_1)
  \\&\simeq \Tor(A_2)
  \simeq R/(b_1) \oplus \cdots \oplus R/(b_n) = A'_2
\end{align*}
Let \((x_1, \cdots, x_m) \in A'_1\).
Then, \(a_m x_k = 0\) for each \(k = 1, \cdots, m\), because \(x_k \in R/(a_k)\) and \(a_k \mid a_m\).
Thus, every element of \(A'_1\) becomes zero by multiplicating by \(a_m\).
Since \(A'_1 \simeq A'_2\), every element of \(A'_2\) becomes zero by multiplicating by \(a_m\) too.
It means, for every \(r \in R\),
\(a_m r + (b_n) = a_m(r + (b_n)) = (b_n)\)
and \(a_m r \in (b_n)\).
This shows \((b_n) \subseteq (a_m)\).

We can do this process from the \(A'_2\):
since every element of \(A'_2\) becomes zero by multiplication by \(b_n\),
every element of \(A'_1\) becomes zero by multiplication by \(b_n\),
and this implies \((a_m) \subseteq (b_n)\).

Then, we know that \((a_m) = (b_n)\). This implies \(a_m \sim b_n\).

Let \(j \in \{1, \cdots, m\}\) be the minimal integer
such that \(a_{j} \sim a_{j + 1} \sim \cdots \sim a_m\).
And let \(k \in \{1, \cdots, n\}\) be the minimal integer
such that \(b_{k} \sim b_{k + 1} \sim \cdots \sim b_n\).
Then, we known
\[(a_j) = (a_{j + 1}) = \cdots = (a_m) = (b_n) = (b_{n-1}) = \cdots = (b_k)\]

Suppose that \(j = 1\).
Then, every elements of \(A'_1\) and \(A'_2\) should be zero by multiplication by \(a_m\).
It means \(k\) should be 1 too.
Then,
\[A'_1 \simeq \oplus_{i=1}^m R/(a_m) \simeq \oplus_{i=1}^n R/(a_m) \simeq A'_2 \qquad\cdots(*)\].
In this case, \(m = n\) must hold.

If \(j > 1\),
since there is an element of \(A'_1\) which are not zero after the multiplication by \(a_m\),
there is an element of \(A'_2\) which are not zero after the multiplication by \(a_m\) too.
Thus, \(k > 1\).

Let's define \(D(r; M) = \{0\} \cup \{m \in M \mid sm = 0 \text{ iff } r \mid s \in R\}\) for \(r \in R\) and \(M \in \Ob(R-\Mod)\).
Then, \(D(r; M) \subseteq M\).

In this case,
\[D(a_m; A'_1) \simeq R/(a_j) \oplus \cdots \oplus R/(a_m)\]
\[A'_1 / D(a_m; A'_1) \simeq R/(a_1) \oplus \cdots \oplus R/(a_{j-1})\]
\[D(a_m; A'_2) \simeq R/(b_k) \oplus \cdots \oplus R/(b_n)\]
\[A'_2 / D(a_m; A'_2) \simeq R/(b_1) \oplus \cdots \oplus R/(b_{k-1})\]
Note that \(D(a_m; A'_1) \simeq D(a_m; A'_2)\) since \(A'_1 \simeq A'_2\).
Thus,
\[\oplus_{i=j}^m R/(a_m) \simeq \oplus_{i=k}^n R/(a_m) \qquad\cdots(*)\]
and \(m - j + 1 = n - k + 1\).
Also, \(A'_1 / D(a_m; A'_1) \simeq A'_2 / D(a_m; A'_2)\).
Thus, let's take \(A''_i = A'_i / D(a_m; A'_i)\) for each \(i = 1, 2\),
repeat the above process for \(A''_1\) and \(A''_2\).
Since the numbers of direct summands are strictly decreasing,
this process halt at some time.

After the repetition,
we know that,
\begin{itemize}
\item There are \(0 = j_0 < j_1 < \cdots < j_p = m\)
  such that \(a_{j_{i-1}} \not\sim a_{j_{i-1} + 1} \sim a_{j_{i-1} + 2} \sim \cdots \sim a_{j_i-1} \sim a_{j_i} \not\sim a_{j_i + 1}\)
  for each \(i\).
\item There are \(0 = k_0 < k_1 < \cdots < k_q = m\)
  such that \(b_{k_{i-1}} \not\sim b_{k_{i-1} + 1} \sim b_{k_{i-1} + 2} \sim \cdots \sim b_{k_i-1} \sim b_{k_i} \not\sim b_{k_i + 1}\)
  for each \(i\).
\item \(p = q\). It's because \(p\) and \(q\) are the number of repition of above process.
\item \(j_i - j_{i - 1} = k_i - k_{i - 1}\)  for every \(i\). It's because of \((*)\).
\item \(m = n\). Because of above properties.
\item \(a_i \sim b_i\) for every \(i\). It can be shown by above 5 properties and counting from the last element using the fact that for each process, the last element of \(\{a_k\}\) and \(\{b_k\}\) associate.
\end{itemize}

Therefore, it shows that \(m = n\) and \(a_i \sim b_i\) for each \(i = 1, \cdots, m\).
\br
\noindent
In conclusion, if
\begin{align*}
  M &\simeq R^r \oplus R/(a_1) \oplus \cdots \oplus R/(a_m)
  \\&\simeq R^s \oplus R/(b_1) \oplus \cdots \oplus R/(b_n)
\end{align*}
\(r = s\), \(m = n\) and \(a_i = b_i\) for each \(i = 1, \cdots, m\).
\qedsq