% MAS511 HW#04

\documentclass{article}

\usepackage{amssymb}
\usepackage{amsmath}
\usepackage{amsthm}
\usepackage{centernot}
\usepackage[a4paper, total={6in, 9in}]{geometry}
\usepackage{enumitem}
\usepackage{mathtools}
\usepackage{stmaryrd}
\usepackage{subfig}
\usepackage{kotex}
\usepackage{tikz-cd}
\usepackage{mathrsfs}

\newtheorem{definition}{Definition}
\newtheorem{theorem}{Theorem}
\newtheorem{principle}{Principle}
\newtheorem{lemma}{Lemma}
\newtheorem{corollary}{Corollary}
\newtheorem{axiom}{Axiom}

\newcommand{\bs}{\char`\\}
\newcommand{\br}{\vspace*{10px}\\}
\newcommand{\qedsq}{\hfill$\square$}

\newcommand{\triangleleftneq}{\mathrel{\ooalign{$\lneq$\cr\raise.22ex\hbox{$\lhd$}\cr}}}

\newcommand{\bbI}{\mathbb{I}}
\newcommand{\bbN}{\mathbb{N}}
\newcommand{\bbQ}{\mathbb{Q}}
\newcommand{\bbR}{\mathbb{R}}
\newcommand{\bbZ}{\mathbb{Z}}

\newcommand{\calA}{\mathcal{A}}
\newcommand{\calB}{\mathcal{B}}
\newcommand{\calC}{\mathcal{C}}
\newcommand{\calD}{\mathcal{D}}
\newcommand{\calF}{\mathcal{F}}
\newcommand{\calG}{\mathcal{G}}
\newcommand{\calH}{\mathcal{H}}
\newcommand{\calN}{\mathcal{N}}
\newcommand{\calP}{\mathcal{P}}
\newcommand{\calS}{\mathcal{S}}
\newcommand{\calf}{\mathcal{f}}
\newcommand{\calg}{\mathcal{g}}
\newcommand{\calh}{\mathcal{h}}

\newcommand{\scrP}{\mathscr{P}}

\newcommand{\Ab}{\textnormal{Ab}}
\newcommand{\Aut}{\textnormal{Aut}}
\newcommand{\Cod}{\textnormal{Cod}}
\newcommand{\Dom}{\textnormal{Dom}}
\newcommand{\Gal}{\textnormal{Gal}}
\newcommand{\Grp}{\textnormal{Grp}}
\newcommand{\Set}{\textnormal{Set}}
\newcommand{\Syl}{\textnormal{Syl}}
\newcommand{\op}{\textnormal{op}}
\newcommand{\Ob}{\textnormal{Ob}}
\newcommand{\Hom}{\textnormal{Hom}}
\newcommand{\Id}{\textnormal{Id}}
\newcommand{\im}{\textnormal{im}\,}

\title{MAS511 Spring 2020 Homework \#04}
\author{}

\begin{document}
\maketitle

% =============================================================================
\section*{Problem 1}

Determine whether \(\bbQ\) satisfies each of the following properties as a \(\bbZ\)-module:
(1) finitely generated,
(2) free,
(3) projective,
(4) flat,
(5) injective.

\subsection*{Solution}

(1) Not finitely generated.

Suppose that \(\bbQ\) is finitely generated. Then, there is \(a_1, \cdots, a_n \in \bbQ\) which satisfies that for any given \(q \in \bbQ\), there is \(z_1, \cdots, z_n \in \bbZ\) such that
\[q = \sum_{k=1}^n z_k a_k\]
Let's assume that each \(a_1, \cdots, a_n\) are non-zero.
(If \(a_i = 0\), \(z_i a_i = 0\) for any \(z_i \in \bbZ\). Thus, \(\sum_{k=1}^n z_k a_k\) does not change even if we ignore the term of \(a_i\).)
Because every non-zero elemtent of \(\bbQ\) can be expressed as \(\frac{p}{q}\) where \(p \in \bbZ\bs\{0\}\), \(q \in \bbN\) and \(p\) and \(q\) are relatively prime, let \(p_1, \cdots, p_n \in \bbZ\bs\{0\}\) and \(q_1, \cdots, q_n \in \bbN\) such that \((p_k, q_k) = 1\) and \(a_k = p_k/q_k\) for each \(k \in \{1, 2, \cdots, n\}\).

Let \[t = \frac{1}{2 \cdot q_1 \cdot q_2 \cdot \cdots \cdot q_n}\]
Trivially, it's a number obtained from \(1\) by dividing a natural number, and \(t \in \bbQ\).
Thus, there should be some \(z_1, \cdots, z_n \in \bbZ\) such that,
\[t
  = \frac{1}{2 \cdot q_1 \cdot q_2 \cdot \cdots \cdot q_n}
  = \sum_{k=1}^n z_k a_k
  = \sum_{k=1}^n \frac{z_k p_k}{q_k}
\]
By multiplicating \(2 \cdot q_1 q_2 \cdots q_n\) in both sides,
\[1 = \sum_{k=1}^{n} t_k \text{  where } t_k = 2 z_k p_k \prod_{j \in \{1, \cdots, n\} \bs \{k\}} q_j\]

Since \(z_k, p_k, q_j\) are integers, \(\sum_{k=1}^{n} t_k\) should be an integer, which can be divided by 2. But, \(\sum_{k=1}^{n} t_k = 1\) and it cannot be divided by 2. Thus, it's a contradiction.

Therefore, \(\bbQ\) cannot be finitely generated as a \(\bbZ\)-module.
\qedsq
\newline

\noindent
(2) Not free. It's because \(\bbQ\) is not projective as a \(\bbZ\)-module. See (3).
\qedsq
\newline

\noindent
(3) Not projective.

\[\begin{tikzcd}
  \bbZ[x_1, x_2, \cdots] \ar[->>, "f"]{r} & \bbQ \ar[->]{r} & 0
  \\
  & \bbQ \ar[->, swap, "\Id_\bbQ"]{u} \ar[->, dashed, "g"]{ul} &
\end{tikzcd}\]

Let \(f\) be a homomorphism such that \(f(z) = z\) for every \(z \in \bbZ\) and \(f(x_k) = \frac{1}{k}\) for every \(k \in \bbN\).
Then, \(f\) is surjective since for any \(q/p \in \bbQ\) where \(q, p \in \bbZ\), \(f(q x_p) = q/p\).

Suppose that there is a homomorphism \(g: \bbQ \to \bbZ[x_1, x_2, \cdots]\)
such that \(\Id_\bbQ = f \circ g\).
Let \(q \in \bbQ \bs \{0\}\).
Then, \(g(q) \neq 0\).
(If not, \(0 \neq q = \Id_\bbQ(q) = f(g(q)) = f(0) = 0\) and it's a contradiction.)
Then, \(g(q) = \sum_{j=0}^n \sum_{k=0}^{m_j} a_{j, k} x_j^k\)
for some \(n, m_k \in \bbZ^{\ge 0}, a_{j, k} \in \bbZ\)
where \(a_{n, m_n} \neq 0\).
(For convinience, let \(x_0 = 1\).)
And, \(f(\sum_{j=0}^n \sum_{k=0}^{m_j} a_{j,k} x_j^k) = q\).
In the similar way, there are \(p, q_k \in \bbZ^{\ge 0}, b_{j, k} \in \bbZ\) such that
\[g(\frac{q}{2a_{n, m_n}}) = \sum_{j=0}^p \sum_{k=0}^{q_j} b_{j, k} x_j^k \]
Then,
\[\sum_{j=0}^n \sum_{k=0}^{m_j} a_{j,k} x_j^k
  = g(q)
  = 2a_{n, m_n} g(\frac{q}{2a_{n, m_n}})
  = \sum_{j=0}^p \sum_{k=0}^{q_j} 2a_{n, m_n} b_{j, k} x_j^k 
\]
Let \(\alpha\) be the coefficient of \(x_j^k\)-term of \(g(q)\) and \(\beta\) be the coefficient of \(x_j^k\)-term of \(g(\frac{q}{2a_{n, m_n}})\).
Then,
\[\alpha = a_{n, m_n},
  \beta = \left\{ \begin{array}{ll} b_{n, m_n} & n \le p, m_n \le q_n \\ 0 & \text{Otherwise} \end{array} \right.\]
Note that \(\alpha, \beta \in \bbZ\).
Since \(g(q) = 2a_{n, m_n} g(\frac{q}{2a_{n, m_n}})\), \(\alpha = 2a_{n, m_n} \beta\).
Then,
\[\beta = \frac{\alpha}{2a_{n, m_n}} = \frac{a_{n, m_n}}{2a_{n, m_n}} = \frac{1}{2}\]
It's a contradiction since \(\beta \in \bbZ\) but \(\frac{1}{2} \not\in \bbZ\).

Therefore, there is no \(g\) such that \(\Id_\bbQ = f \circ g\), and it shows that \(\bbQ\) is not projective as a \(\bbZ\)-module.
\qedsq
\newline

\noindent
(4) Flat.

To show that \(\bbQ\) is flat, \(\bbQ \otimes_\bbZ -\) and \(- \otimes_\bbZ \bbQ\) are right exact fucntor.

Let's show that \(\bbQ \otimes_\bbZ -\) is right exact. Then, \(- \otimes_\bbZ \bbQ\) can be shown that it's also right exact in the similar method.

\[\begin{tikzcd}
  0 \ar[->]{r} & M \ar[>->, "f"]{r} \ar[Rightarrow, "\bbQ \otimes_\bbZ -"]{d} & N
  \\
  0 \ar[->]{r} & \bbQ \otimes_\bbZ M \ar[->, "\varphi"]{r} & \bbQ \otimes_\bbZ N
\end{tikzcd}\]

Let \(f: M \to N\) be an injective \(\bbZ\)-module homomorphism.
Let \(\varphi: \bbQ \otimes_\bbZ M \to \bbQ \otimes_\bbZ N\) such that \(\varphi = \bbQ \otimes_\bbZ f\) (note, \(\varphi(q \otimes m) = q \otimes f(m)\)).
Then, it's enough to show that \(\varphi\) is injective.

First, note that for arbitrary left \(\bbZ\)-module \(K\):
\begin{itemize}
\item \(0 \otimes 0\) is an (additive) identity of \(\bbQ \otimes_\bbZ K\).
\item \(0 \otimes k = 0 \otimes 0\).
  It's because \(0 \otimes k = (0 \cdot 0) \otimes k = 0 \otimes (0 \cdot k) = 0 \otimes 0\).
\item If \(k\) is an (additive) identity of \(K\), \(k = 0 \cdot k\) and \(q \otimes k = q \otimes (0 \cdot k) = (q \cdot 0) \otimes k = 0 \otimes k = 0 \otimes 0\) for any \(q \in \bbQ\).
\item For every \(q \in \bbQ \bs \{0\}\) and \(k \in K\), there is \(n \in \bbN\) such that \(q \otimes k = \frac{1}{n} \otimes k'\) for some \(k' \in K\). Because, let \(q = \frac{m}{n}\) for some \(n \in \bbN\) and \(m \in \bbZ\),
\(q \otimes k = (\frac{1}{n} \otimes m) \otimes k = \frac{1}{n} \otimes (m \cdot k)\)
\item Every non-zero element of \(\bbQ \otimes_\bbZ K\) can be expressed as \(\frac{1}{n} \otimes k\) for some \(n \in \bbN\) and \(k \in K\).

Let \(t \in \bbQ \otimes_\bbZ K\) is non-zero (i.e. not an additive identity). 
Then, there are \(q_1, \cdots, q_m \in \bbQ\) and \(k_1, \cdots, k_m \in K\)
such that
\[t = \sum_{i=1}^m q_i \otimes k_i\]
If some \(q_i \otimes k_i\) are an additive identity, the sum does not change even if we ignore the terms.
Thus, let's assume that every \(q_i \otimes k_i\) is non-zero.
(If every \(q_i \otimes k_i\) are zero, \(t\) should be zero. But as we assumed that \(t\) is non-zero, \(t\) must be a sum of at least one non-zero \(q_i \otimes k_i\).)
But as we showed above, there are \(n_1, \cdots, n_m \in \bbN\) and \(k'_1, \cdots, k'_m \in K\) such that \(q_i \otimes k_i = \frac{1}{n_i} \otimes k'_i\).
Also, let \(n = \prod_{i = 1}^m n_i\) and \(\kappa_i = \frac{n}{n_i} \cdot k'_i\).
Then,
\[\frac{1}{n_i} \otimes k_i = \frac{1}{n} \otimes (\frac{n}{n_i} \cdot k_i) = \frac{1}{n} \otimes \kappa_i\]
and
\[t = \sum_{i=1}^m q_i \otimes k_i = \sum_{i=1}^m \frac{1}{n} \otimes \kappa_i = \frac{1}{n} \otimes (\sum_{i=1}^m \kappa_i)\]
Let \(k = \sum_{i=1}^m \kappa_i\) then we obtain \(t = \frac{1}{n} \otimes k\).
\end{itemize}

Let \(t\) is a non-zero element of \(\bbQ \otimes_\bbZ M\),
and suppose that \(\varphi(t) = 0_{\bbQ \otimes_\bbZ N}\).
As we showed above, there is \(n \in \bbN\) and \(m \in M\) such that \(t = \frac{1}{n} \otimes m\).
Then, \(\varphi(t) = \varphi(\frac{1}{n} \otimes m) = \frac{1}{n} \otimes f(m) = 0_{\bbQ \otimes_\bbZ N}\).
Then, since there is no \(z \in \bbZ\) such that \(\frac{z}{n} = 0\), there should be some \(z \in \bbZ\) such that \(zf(m) = f(zm) = 0\).
Since \(f\) is injective, \(zm = 0\).
Then,
\[t = \frac{1}{n} \otimes m = \frac{1}{nz} \otimes zm = \frac{1}{nz} \otimes 0_M = 0_{\bbQ \otimes M}\]
It's a contradiction since we assumed that \(t\) is non-zero.

Thus, \(\ker \varphi = \{0_{\bbQ \otimes_\bbZ M}\}\) and \(\varphi = \bbQ \otimes_\bbZ f\) is injective. In the same way \(f \otimes_\bbZ \bbQ\) is injective for a \(\bbZ\)-module homomorphism \(f: M \to N\) where \(M\) and \(N\) are right \(\bbZ\)-module homomorphism. Thus, \(\bbQ\) is a flat \(\bbZ\)-module.
\qedsq
\newline

\noindent
(5) Injective

Use Baer Criterion.
Since \(\bbZ\) is a Euclidean domain, PID, every ideal of \(\bbZ\) is principal.
Let \(n \in \bbZ\) and take an ideal \((n)\).
Let \(g: (n) \to \bbQ\) be a \(\bbZ\)-module homomorphism.

If \(n = 0\), \(g(0) = 0\).
(Otherwise, \(2g(0) = g(2 \cdot 0) = g(0)\) should hold and \(0 \neq g(0) = 2g(0) - g(0) = 0\). It's a contradiction.)
Thus, if we define \(\tilde{g}: \bbZ \to \bbQ\) as \(\tilde{g} = 0\), it's a extension of \(g\) and a \(\bbZ\)-module homomorphism (more specifically, zero homomorphism).

For \(n \neq 0\), every element \(m\) of \(\bbZ\) can be expressed as \(\frac{mn}{n}\). Then, let's define \(\tilde{g}: \bbZ \to \bbQ\) as
\[\tilde{g}(m) = \frac{g(mn)}{n}\]

Then,
\begin{itemize}
\item \(\tilde g\) is well-defined.
  For \(m \in \bbZ\), \(mn \in n\bbZ = (n) = \Dom(g)\).
\item \(\tilde g\) is an extension of \(g\).
  Let \(m \in (n)\).
  Then, \(\tilde{g}(m) = \frac{g(mn)}{n} = \frac{ng(m)}{n} = g(m)\).
\item \(\tilde g\) is a \(\bbZ\)-module homomorphism.
  Let \(x, y \in \bbZ\), then,
  \[\tilde{g}(x + y) = \frac{g((x + y)n)}{n} = \frac{g(xn) + g(yn)}{n} = \tilde{g}(x) + \tilde{g}(y)\]
  \[\tilde{g}(xy) = \frac{g(xyn)}{n} = \frac{xg(yn)}{n} = x \tilde{g}(y)\]
\end{itemize}

Thus, by Baer's Criterion, \(\bbQ\) is a injective \(\bbZ\)-module.
\qedsq

\newpage
% =============================================================================
\section*{Problem 2}

Let \(R\) be a ring
and let
\[0 \rightarrow K_P \xrightarrow{f_1} P \xrightarrow{f_0} M \rightarrow 0\]
\[0 \rightarrow K_Q \xrightarrow{g_1} Q \xrightarrow{g_0} M \rightarrow 0\]
be exact sequences of left \(R\)-modules such that \(P\) and \(Q\) are projective.
Prove that there exists an isomorphism of left \(R\)-modules
\[K_P \oplus Q \simeq K_Q \oplus P\]

\subsection*{Solution}
\[\begin{tikzcd}
  \ker f_0 \simeq \im f_1 \simeq K_P
  \arrow[dd, dashed, shift left, "\exists h"]
  \ar[>->, "f_1"]{r} & P \ar[->>, "f_0"]{rd}
  \arrow[dd, dashed, shift left, "\exists\tilde{f_0}"]
  & \\
  & & M \\
  \ker g_0 \simeq \im g_1 \simeq K_Q
  \ar[>->, swap, "g_1"]{r} & Q \ar[->>, swap, "g_0"]{ru}
  \arrow[uu, dashed, shift left, "\exists\tilde{g_0}"]
  & 
\end{tikzcd}\]
Since \(P\) and \(Q\) are projective, there are liftings \(\tilde{f_0}: P \to Q\) and \(\tilde{g_0}: Q \to P\).
And, \(f_0 = g_0 \circ \tilde{f_0}\) and \(g_0 = f_0 \circ \tilde{g_0}\).

Since \(f_1\) and \(g_1\) are injective homomorphisms, \(K_P \simeq \im f_1\) with an isomorphism \(f_1\) and \(K_Q \simeq \im g_1\) with an isomorphism \(g_1\).
Note that \(f_1\) is an isomorphism by restricting its codomain to \(\im f_1 = \ker f_0\) and \(g_1\) is an isomorpihsm by restricting its codomain to \(\im g_1 = \ker g_0\).

Let \(x \in K_P\). Then, \(0 = f_0(f_1(x)) = g_0((\tilde{f_0} \circ f_1)(x))\).
Thus, \((\tilde{f_0} \circ f_1)(x) \in \ker g_0 = \im g_1\).
Let \(h: K_P \to K_Q\) be a map such that
\[h(x) = g_1^{-1}(\tilde{f_0}(f_1(x)))\]
Since \(\im (\tilde{f_0} \circ f_1) \subseteq \im g_1\), it's well-defined, and it's a homomorphism since \(g_1\) is an isomorphism and \(\tilde{f_0} \circ f_1\) is a homomorphism from \(K_P\) to \(\im g_1\).

Then, let \(\varphi: K_P \to K_Q \oplus P\) such that \(\varphi(k) = (h(k), f_1(k))\) for \(k \in K_P\),
and \(\psi: K_Q \oplus P \to Q\) such that \(\psi(k, p) = g_1(k) - \tilde{f_0}(p)\).
Then,
\begin{itemize}
\item \(\varphi\) is injective. If \(k \in \ker\varphi\), \(\varphi(k) = (0, 0)\). Then, \(f_1(k) = 0\) Since \(f_1\) is injective, \(k\) should be \(0\).
\item \(\psi\) is surjective.
  Let \(q \in Q\).
  Then, \(q - (\tilde{f_0} \circ \tilde{g_0})(q) \in \ker g_0 = \im g_1\),
  because
  \(g_0(q - (\tilde{f_0} \circ \tilde{g_0})(q))
    = g_0(q) - f_0(\tilde{g_0}(q))
    = g_0(q) - g_0(q) = 0\).
  Take \(k = g_1^{-1}(q - (\tilde{f_0} \circ \tilde{g_0})(q))\)
  and \(p = -\tilde{g_0}(q)\).
  Then,
  \[\psi(k, p) =
    g(g_1^{-1}(q - (\tilde{f_0} \circ \tilde{g_0})(q))) - (- \tilde{f_0}(\tilde{g_0}(q))) = q\]
\item \(\ker \psi = \im \varphi\).
  Let \(k \in K_P\).
  Then,
  \begin{align*}
    \psi(h(k), f_1(k))
    &= g_1(h(k)) - \tilde{f_0}(f_1(k))
    \\&= g_1(g_1^{-1}(\tilde{f_0}(f_1(k)))) - \tilde{f_0}(f_1(k))
    \\&= \tilde{f_0}(f_1(k)) - \tilde{f_0}(f_1(k))
    = 0
  \end{align*}
\end{itemize}
Thus, the below is a short exact sequence:
\[\begin{tikzcd}
  0 \ar{r} &
  K_P \ar["\varphi"]{r} &
  K_Q \oplus P \ar["\psi"]{r} &
  Q \ar{r} &
  0
\end{tikzcd}\]
Since \(Q\) is projective, above sequence is split.
Therefore, \(K_Q \oplus P \simeq K_P \oplus Q\).
\qedsq

\newpage
% =============================================================================
\section*{Problem 3}

Let \(n \ge 2\) be an integer.
We define two complexes of \(\bbZ\)-modules \(A_\bullet\), \(B_\bullet\) by
\[A_\bullet = \left(
  \cdots
  \longrightarrow A_3 = 0
  \longrightarrow A_2 = 0
  \longrightarrow A_1 = \bbZ
  \xrightarrow{\times n} A_0 = \bbZ
  \longrightarrow A_{-1} = 0
  \longrightarrow \cdots
  \right)\]
\[B_\bullet = \left(
  \cdots
  \longrightarrow B_3 = 0
  \longrightarrow B_2 = 0
  \longrightarrow B_1 = 0
  \longrightarrow B_0 = \bbZ/n\bbZ
  \longrightarrow B_{-1} = 0
  \longrightarrow \cdots
  \right)\]
Prove that \(A_\bullet\) and \(B_\bullet\) are quasi-isomorphic but \emph{not} chain homotopy equivalent.

\subsection*{Solution}

\[H_k(A_\bullet) = \frac{\ker f_k}{\im f_{k + 1}} \text{ where } f_k: A_k \to A_{k - 1}\]

Note that
\[\ker (0 \rightarrow M) = \im (M \rightarrow 0) =
  \im (0 \rightarrow M) = 
  \{0\}\]
\[\ker (M \rightarrow 0) = M\]
\[\ker (\bbZ \xrightarrow{\times n} \bbZ) = \{0\}\]
\[\im (\bbZ \xrightarrow{\times n} \bbZ) = n\bbZ\]
where \(M\) is a \(\bbZ\)-module. And,
\[H_0(A_\bullet) = H_0(B_\bullet) = \bbZ_n\]
\[H_k(A_\bullet) = H_k(B_\bullet) = 0 \text{ for } k \in \bbZ\bs\{0\}\]

\subsection*{Quasi-isomorphic}

\[\begin{tikzcd}
  A_\bullet = & \cdots \ar{r} & A_2 = 0 \ar{r} & A_1 = \bbZ \ar["\times n"]{r} & A_0 = \bbZ \ar{r} & A_{-1} = 0 \ar{r} & \cdots \\
  C_\bullet \ar["f_\bullet"]{u} \ar[swap, "g_\bullet"]{d} = & \cdots \ar{r} & C_2 = 0 \ar{r} \ar["f_2"]{u} \ar[swap, "g_2"]{d} & C_1 = \bbZ \ar["f_1"]{u} \ar[swap, "g_1"]{d} \ar["\times n"]{r} & C_0 = \bbZ \ar["f_0"]{u} \ar[swap, "g_0"]{d} \ar{r} & C_{-1} = 0 \ar["f_{-1}"]{u} \ar[swap, "g_{-1}"]{d} \ar{r} & \cdots \\
  B_\bullet = & \cdots \ar{r} & B_2 = 0 \ar{r} & B_1 = 0 \ar{r} & B_0 = \bbZ_n \ar{r} & B_{-1} = 0 \ar{r} & \cdots
\end{tikzcd}\]

Take \(f_k, g_k\) are identity maps except \(g_1\) which is a zero map and \(g_0\) which is a canonical injection (\(g_0: x \mapsto x + n\bbZ\)).

Since all of \(H_k(A_\bullet), H_k(B_\bullet), H_k(C_\bullet)\) are \(0\) for \(k \in \bbZ \bs \{0\}\), the homomorphism induced from \(f_k, g_k\) for this \(k\) is a zero map, which is an isomorphism between zero modules.

Since \(H_0(A_\bullet) = H_0(C_\bullet) = \bbZ / n\bbZ\) and \(f_0\) is an identity map, the homomorphism induced from \(f_0\) is an identity map between \(\bbZ / n\bbZ\). Thus it's an isomorphism between \(H_0(A_\bullet) = H_0(C_\bullet) = \bbZ / n\bbZ\).

Since \(g_0\) is a canonical injection from \(\bbZ\) to \(\bbZ / n\bbZ\), it induces an identity map from \(\bbZ / n\bbZ\) to itself.
Thus, \(g_0\) is also an isomorphism between \(H_0(B_\bullet) = H_0(C_\bullet) = \bbZ / n\bbZ\).

Therefore, there is a \(C_\bullet\) and chain maps from \(C_\bullet\) to \(A_\bullet\) and \(B_\bullet\) which induces isomorphisms betwene homology groups.
Thus, \(A_\bullet\) and \(B_\bullet\) are quasi-isomorphic.
\qedsq

\subsection*{Not chain homotopy equivalent}

Suppose that there are chain maps \(f_\bullet: A_\bullet \to B_\bullet\) and \(g_\bullet: B_\bullet \to A_\bullet\) such that \(f_\bullet \circ g_\bullet \sim \Id_{B_\bullet}\) and \(g_\bullet \circ f_\bullet \sim \Id_{A_\bullet}\).

\begin{tikzcd}
  A_\bullet \ar["f_\bullet", shift left]{d} = & \cdots \ar{r} & A_2 = 0 \ar{r} \ar["f_2", shift left]{d} & A_1 = \bbZ \ar{r} \ar["f_1", shift left]{d} & A_0 = \bbZ \ar{r} \ar["f_0", shift left]{d} & A_{-1} = 0 \ar{r} \ar["f_{-1}", shift left]{d} & \cdots \\
  B_\bullet \ar["g_\bullet", shift left]{u} = & \cdots \ar{r} & B_2 = 0 \ar{r} \ar["g_2", shift left]{u} & B_1 = 0 \ar{r} \ar["g_1", shift left]{u} & B_0 = \bbZ_n \ar{r} \ar["g_0", shift left]{u} & B_{-1} = 0 \ar{r} \ar["g_{-1}", shift left]{u} & \cdots
\end{tikzcd}

Note that there are only one homomorphism from some \(\bbZ\)-module to \(0\), which is a zero map, and a zero map is the only homomorphism from \(0\) to some \(\bbZ\)-module. Thus, \(f_k \equiv 0\) and \(g_k \equiv 0\) for \(k \in \bbZ \bs \{0\}\).

Also, \(g_0\) should be zero.
(Because the order of \(g_0(1)\) should be \(n\) since \(n g_0(1) = g_0(n) = g_0(0) = 0\), but every element of \(\bbZ\) has an infinite order.)

Then, \(f_0 \circ g_0\) is a zero map. Also, since every \(f_k, g_k\) for \(k \in \bbZ \bs \{0\}\) are zero maps, \(f \circ g\) is a zero chain map.

Suppose that \(f \circ g\) is homotopic to \(\Id_B\).
Then there is a maps \(s_k: B_k \to B_{k + 1}\) for each \(k \in \bbZ\) such that \(\Id_B - (f \circ g) = sd + ds\), where \(d_k: B_k \to B_{k - 1}\) are maps of chain complex \(B_\bullet\).
Let \(h = \Id_B - (f \circ g)\) (i.e. \(h_k = \Id_{B_k} - (f_k \circ g_k)\)). Then, \(h_k = \Id_{B_k}\) since \(f_k \circ g_k = 0\) for each \(k \in \bbZ\). And,
\[\begin{tikzcd}
  \cdots \ar{r} &
  B_{1} = 0 \ar["d_1"]{r} \ar[swap, "\Id_{B_{1}}"]{d} &
  B_{0} = \bbZ_n \ar["d_0"]{r} \ar["s_0"]{ld} \ar["\Id_{B_0}"]{d} &
  B_{-1} = 0 \ar{r} \ar["s_{-1}"]{ld} \ar["\Id_{B_{-1}}"]{d} &
  \cdots \\
  \cdots \ar{r} &
  B_{1} = 0 \ar[swap, "d_1"]{r} &
  B_{0} = \bbZ_n \ar[swap, "d_0"]{r} &
  B_{-1} = 0 \ar{r} &
  \cdots
\end{tikzcd}\]
Since \(s_0\) and \(d_0\) are homomorphisms to a zero module, \(s_0 = d_0 = 0\).
Since \(s_{-1}\) and \(d_1\) are homomorphisms from a zero module, \(s_{-1} = d_1 = 0\).
Thus, \(d_1s_0 + s_{-1}d_0 = 0\).
Since \(\Id_{B_0} = \Id_{\bbZ_n} \neq 0\), \(\Id_{\bbZ_n} = d_1s_0 + s_{-1}d_0 = 0\) cannot hold.
Thus, it's a contradiction.

Therefore, \(f \circ g\) cannot be chain homotopic to \(\Id_B\).
It implies that \(A\) and \(B\) cannot be chain homotopy equivalent.
\qedsq

\newpage
% =============================================================================
\section*{Lemmata}

\begin{lemma}\label{lem-char}
  Let \(G\) be an abelian group, and \(K \le G\) be a subgroup.
  Let \(\varphi: K \to \bbQ / \bbZ\) is a group homomorphism.
  Then, there is a group homomorphism \(\tilde{\varphi}: G \to \bbQ / \bbZ\)
  such that \(\left. \tilde{\varphi} \right|_K = f\).
\end{lemma}
\begin{proof}

Let
\[\calP = \{(H, f) \mid H \le G, \text{homomorphism} f: H \to \bbQ/ \bbZ\}\]
Note that \(\calP\) is non-empty since \((K, \varphi) \in \calP\).
Then, let \(\prec\) be a relation of \(\calP\) such that
\((H, f) \prec (H', f')\) if and only if \(H \le H'\) and \(\left. f' \right|_H = f\).
It's a partial order since,
\begin{itemize}
\item Reflexive. Because \(H \le H\) and \(\left. f \right|_H = f\).
\item Antisymmertic. If \((H, f) \prec (H', f')\) and \((H, f) \succ (H', f')\),
\(H \le H' \le H\) implies \(H = H'\), and \(f = \left. f' \right|_H = \left. f' \right|_{H'} = f'\) holds. Thus, \((H, f) = (H', f')\).
\item Transitive. If \((H, f) \prec (H', f')\) and \((H', f') \prec (H'', f'')\), \(H \le H' \le H''\) and \(f = \left. f' \right|_H = \left. \left. f'' \right|_{H'} \right|_{H} = \left. f'' \right|_{H}\). Thus \((H, f) \prec (H'', f'')\).
\end{itemize}

If a chain of the relation \(\prec\) is finite, there is a maximum element of the chain.
Let \(\{(H_\alpha, f_\alpha)\}_{\alpha \in \Lambda}\) be a infinitely long chain of the relation \(\prec\).
Then, we can construct a element \((M, g) \in \calP\):
\[M = \bigcup_{\alpha \in \Lambda} H_\alpha\]
\[g(x) = f_\alpha(x) \text{ for some } \alpha \text { such that } x \in H_\alpha\]
\(M \subseteq G\), because each \(H_\alpha \subseteq G\).
\(M\) is a group, because, (1) Since each \(H_\alpha\) contains \(0_G\), \(0_G \in M\); (2) for \(a, b \in M\), there is \(\alpha, \beta \in \Lambda\) such that \(a \in H_\alpha\) and \(b \in H_\beta\), WLOG if we supppose that \((H_\alpha, f_\alpha) \prec (H_\beta, f_\beta)\), \(a + b \in H_\beta \subseteq M\), thus \(M\) is closed under addition; (3) for \(a, b, c \in M\), \((a + b) + c = a + (b + c)\) because there is \(\alpha, \beta, \gamma \in \Lambda\) such that \(a \in H_\alpha\), \(b \in H_\beta\), and \(c \in H_\gamma\), and WLOG if we assumed that \((H_\beta, f_\beta), (H_\gamma, f_\gamma) \prec (H_\alpha, f_\alpha)\), \((a + b) + c\) and \(a + (b + c)\) are operations in \(H_\alpha\) and they are equal since \(H_\alpha\) is a group; (4) if \(a \in M\), there is \(\alpha \in \Lambda\) such that \(a \in H_\alpha\), and since \(a^{-1} \in H_\alpha\), \(a^{-1} \in M\).
Therefore, \(M \le G\).
\(g\) is well-defined because if \(x \in H_\alpha\) and \(x \in H_\beta\) for some \(\alpha, \beta \in \Lambda\), WLOG if we assumed that \((H_\alpha, f_\alpha) \prec (H_\beta, f_\beta)\), \(f_\beta(x) = \left. f_\beta\right|_{H_\alpha}(x) = f_\alpha(x)\) holds.
\(g\) is a homomorphism from \(M\) to \(\bbQ/\bbZ\), because if \(x_\alpha \in H_\alpha, x_\beta \in H_\beta\), WLOG if we assumed that \((H_\alpha, f_\alpha) \prec (H_\beta, f_\beta)\), \(g(x_\alpha + x_\beta) = f_\beta(x_\alpha + x_\beta) = f_\beta(x_\alpha) + f_\beta(x_\beta) = g(x_\alpha) + g(x_\beta)\).
And \((M, g)\) is the maximum of the chain since \(H_\alpha \le \bigcup_{\alpha \in \Lambda} H_\alpha = M\) for any \(\alpha \in \Lambda\) and \(g(x) = f_\alpha(x)\) for any \(x \in H_\alpha\).
It implies that \((M, g)\) is the maximum element of the chain, which is contained in \(\calP\).

Since \(\calP\) is non-empty,
by Zorn's Lemma, there is a maximal element of \(\calP\).
\newline

Let \((M, \tilde\varphi)\) be a maximal element of \((K, \varphi)\) in \(\calP\).

Suppose that \(M \lneq G\).
Then, there is \(x \in G \bs M\).
In this case, there may be some \(k \in \bbN\) such that \(k \cdot x \in M\).
If such \(k\) exists, let \(p\) be a minimum \(k\) and \(z = \frac{\alpha(p \cdot x)}{p}\).
If there is no such \(k\), let \(p = 0\) and \(z = 0 + \bbZ\).
And define \(M' = \langle x \rangle + M\) and \(\beta: M' \to \bbQ / \bbZ\) such that
\[\beta(k \cdot x + y) = kz + \tilde\varphi(y)\]
where \(k \in \bbZ\), \(y \in M\).

\(\beta\) is well-defined, because if \(k \cdot x + y = k' \cdot x + y'\) for \(k, k' \in \bbZ\) and \(y, y', \in M\), \((k - k') \cdot x = y' - y\).
Since \(y' - y \in M\), \((k - k') \cdot x \in M\).
Then, \(\tilde\varphi(y') - \tilde\varphi(y) = \tilde\varphi((k - k') \cdot x)\)
and \(\tilde\varphi(y') = \tilde\varphi(y) + \tilde\varphi((k - k') \cdot x)\).
\begin{itemize}
\item If \(p > 0\), \(p \mid k - k'\). Let \(q \in \bbZ\) such that \(k - k' = pq\). Then,
\begin{align*}
  \beta(k' \cdot x + y')
  = k'z + \tilde\varphi(y')
  &= k'z + \tilde\varphi(y) + \tilde\varphi((k - k') \cdot x)
  \\&= k'z + \tilde\varphi(y) + \tilde\varphi(qp \cdot x)
  \\&= k'z + \tilde\varphi(y) + q \tilde\varphi(p \cdot x)
  \\&= k'z + \tilde\varphi(y) + q pz
  \\&= k'z + \tilde\varphi(y) + (k - k')z
  = kz + \tilde\varphi(y)
  = \beta(k \cdot x + y)
\end{align*}
\item If \(p = 0\), \(k - k'\) should be \(0\). Thus \((k - k') \cdot x = 0\) and \(k = k'\). Then,
\begin{align*}
  \beta(k' \cdot x + y')
  &= k'z + \tilde\varphi(y')
  \\&= k'z + \tilde\varphi(y) + \tilde\varphi((k - k') \cdot x)
  \\&= kz + \tilde\varphi(y) + \tilde\varphi(0 \cdot x)
  = kz + \tilde\varphi(y)
  = \beta(k \cdot x + y)
\end{align*}
\end{itemize}

Also, \(\beta\) is a homomorphism, because for \(k, k' \in \bbZ\) and \(y, y' \in M\),
\begin{align*}
  \beta((k \cdot x + y) + (k' \cdot x + y'))
  &= \beta((k + k') \cdot x + (y + y'))
  \\&= (k + k')z + \tilde\varphi(y + y')
  \\&= kz + \tilde\varphi(y) + k'z + \tilde\varphi(y')
  = \beta(k \cdot x + y) + \beta(k' \cdot x + y')
\end{align*}

And, \(\left. \beta \right|_M = \tilde\varphi\) because for \(y \in M\),
\[\beta(y) = 0z + \tilde\varphi(y) = \tilde\varphi(y)\]

However, it's a contradiction since we assumed that \((M, \tilde\varphi)\) is maximal, but there is \((M', \beta)\) such that \((M, \tilde\varphi) \prec (M' \beta)\) but \((M, \tilde\varphi) \neq (M', \beta)\).

Therefore, \(M\) should be \(G\).
In this case, \(\tilde\varphi:G \to \bbQ / \bbZ\) is a homomorphism and, 
\[\tilde\varphi(k) = \left. \tilde\varphi \right|_{K} (k) = \varphi(k)\]
for every \(k \in K\).

Thus, there is a homomorphism from \(G\) to \(\bbQ / \bbZ\) which is extended from the given homomorphism \(\varphi\).
\end{proof}

\newpage
% =============================================================================
\section*{Problem 4}

Let \(R\) be a ring with unity
and \(M\) be a left \(R\)-module.
We equip the abelian group \(M^* := \Hom_\bbZ(M, \bbQ/\bbZ)\)
with right \(R\)-module structure via
\((\varphi r)(m) := \varphi(rm)\) for \(r \in R\), \(\varphi \in M^*\) and \(m \in M\).
\begin{enumerate}[label=(\arabic*)]
\item
  Let \(G\) be an abelian group.
  Prove that for any nonzero \(g \in G\), there is a group homomorphism \(\alpha: G \to \bbQ / \bbZ\) such that \(\alpha(g) \neq 0\).
\item
  Suppose that \(M^*\) is an injective right \(R\)-module.
  For any injective (in the set-theoretical sense) homomorphism of right \(R\)-modules \(f: A \to B\), prove that
  \[\Hom_\bbZ(B \otimes_R M, \bbQ/\bbZ)
    \xrightarrow{- \circ (f \otimes_R \Id_M)}
    \Hom_\bbZ(A \otimes_R M, \bbQ/\bbZ)
    \longrightarrow
    0\]
  is an exact sequence of abelian groups.
  Using this and (1), prove also that \(M\) is a flat left \(R\)-module.
\end{enumerate}

\subsection*{Solution of (1)}

Let \(G\) be an abelian group and \(g \in G\) be a non-zero element.

First of all, we can easily make a homomorphism from \(\langle g \rangle\)
to \(\bbQ/\bbZ\).
Let \(n = |\langle g \rangle|\) if \(|\langle g \rangle| < \infty\),
 \(n = 2\) otherwise.
Then, let \(f: \langle g \rangle \to \bbQ/\bbZ\) such that:
\[f(k \cdot g) = \frac{k}{n} + \bbZ\]
for every \(k \in \bbZ\).

\(f\) is well-defined. Because if \(k \cdot g = l \cdot g\) for \(k \neq l\), \((l - k) \cdot g = 0\).
If \(|\langle g \rangle| = \infty\), it's not possible.
Thus \(n = |\langle g \rangle| < \infty\) and \(n \mid l - k\).
Then, \(\frac{l - k}{n} \in \bbZ\) and
\[f(k \cdot g) = \frac{k}{n} + \bbZ = \frac{k}{n} + \frac{l - k}{n} + \bbZ
  = \frac{l}{n} + \bbZ = f(l \cdot g)\]

Also, \(f\) is a homomorphism since, for \(k, l \in \bbZ\),
\[f(k \cdot g + l \cdot g) = f((k + l) \cdot g) = \frac{k + l}{n} + \bbZ = \left(\frac{k}{n} + \bbZ\right) + \left(\frac{l}{n} + \bbZ\right) = f(k \cdot g) + f(l \cdot g)\]

In addition \(f(g) = \frac{1}{n} + \bbZ \neq 0 + \bbZ\). Thus, it maps \(g\) to some non-zero element.

Therefore, \(f\) is a homomorphism maps \(g\) to some non-zero element.
\newline

Since \(\langle g \rangle \le G\), by Lemma \ref{lem-char},
there is a group homomorphism \(\alpha: G \to \bbQ / \bbZ\)
such that \(\left. \alpha \right|_{\langle g \rangle} = f\).
It implies, \(\alpha\) is a homomorphism such that,
\[\alpha(g) = f(g) = \frac{1}{n} + \bbZ \neq 0 + \bbZ\]
Therefore, for an arbitrary abelian group \(G\) and some non-zero element \(g \in G\), there is a group homomorphism \(\alpha\) which maps \(g\) to a non-zero element.
\qedsq

\subsection*{Solution of (2)}

Suppose that \(M^*\) be an injective right \(R\)-module.
And let \(f: A \hookrightarrow B\) is an injective \(R\)-module homomorphism.

Then, we can construct an exact sequence:
\[\begin{tikzcd}
  0 \ar{r} &
  A \ar["f"]{r} &
  B
\end{tikzcd}\]
Since \(M^*\) is injective, the contravariant functor \(\Hom_R(-, M^*)\) is exact.
Note that \(\Hom_R(h, M) = - \circ h\) for a \(R\)-module homomorphism \(h\).
Thus,
\[\begin{tikzcd}
  \Hom_R(B, M^*) \ar["- \circ f"]{r} &
  \Hom_R(A, M^*) \ar{r} &
  0
\end{tikzcd}\]
is an exact sequence.

Since \(M^* = \Hom_\bbZ(M, \bbQ/\bbZ)\),
by adjuction formula,
for right \(R\)-module \(N\),
\[\Hom_\bbZ(N \otimes_R M, \bbQ/\bbZ)
  \simeq \Hom_R(N, \Hom_\bbZ(M, \bbQ/\bbZ))
  = \Hom_R(N, M^*)\]
Thus,
\[\begin{tikzcd}
  \Hom_\bbZ(B \otimes_R M, \bbQ/\bbZ) \ar["\varphi"]{r} &
  \Hom_\bbZ(A \otimes_R M, \bbQ/\bbZ) \ar{r} &
  0
\end{tikzcd}\]
is exact, where \(\varphi\) was induced from \(f\).

Let \(\eta_N: \Hom_R(N, \Hom_\bbZ(M, \bbQ/\bbZ)) \to \Hom_\bbZ(N \otimes_R M, \bbQ/\bbZ)\) be a natural isomorphism for a \(\bbR\)-module \(N\), such that
\[\eta(f)(n \otimes m) = f(n)(m)\]
Then, \(\varphi = \eta_A \circ (- \circ f) \circ \eta_B^{-1}\) and,
\begin{align*}
  \varphi(g)(a \otimes m)
  &= (\eta_A \circ (- \circ f) \circ \eta_B^{-1})(g)(a \otimes m)
  \\&= \eta_A(\eta_B^{-1}(g) \circ f)(a \otimes m)
  \\&= (\eta_B^{-1}(g) \circ f)(a)(m)
  \\&= \eta_B^{-1}(g)(f(a))(m)
  \\&= g(f(a) \otimes m)
\end{align*}
for \(g \in \Hom_\bbZ(B \otimes_R M, \bbQ/\bbZ)\), \(a \in A\) and \(m \in M\),  therefore,
\[\varphi = - \circ (f \otimes_R \Id_M)\]
(Since \(A \otimes_R M\) is generated by \(a \otimes m\) for \(a \in A\) and \(m \in M\), and \(\varphi\) is a homomorphism as a composition of homomorphisms, it's enough to check only for the case of generator \(a \otimes m\).)

Therefore,
\[
  \Hom_\bbZ(B \otimes_R M, \bbQ/\bbZ)
  \xrightarrow{- \circ (f \otimes_R \Id_M)}
  \Hom_\bbZ(A \otimes_R M, \bbQ/\bbZ)
  \rightarrow
  0
\]
is exact.
\qedsq

\subsection*{\(M\) is flat}

Suppose that \(M^* = \Hom_\bbZ(M, \bbQ/\bbZ)\) is injective.

To show that \(M\) is flat, it's enough to check that \(- \otimes_R M\) is left exact.

In other words, for right \(R\)-modules \(A, B, C\) and \(f: A \to B\)  such that
\[
  0
  \rightarrow
  A
  \xrightarrow{f}
  B
  \rightarrow
  C
  \rightarrow
  0
\]
the below should be exact:
\[
  0
  \rightarrow
  A \otimes_R M
  \xrightarrow{f \otimes_R M}
  B \otimes_R M
  \rightarrow
  C \otimes_R M
  \rightarrow
  0
\]

Suppose that \(\ker (f \otimes_R M) \neq \{0\}\).
In other words, there is a non-zero
\(a \in \ker (f \otimes_R M) \subseteq \Dom (f \otimes_R M) = A \otimes_R M\).
Since \(A \otimes_R M\) is an abelian group,
there is a group homomorphism \(\alpha: A \otimes_R M \to \bbQ / \bbZ\)
such that \(\alpha(a) \neq 0\).
Note that an abelian group is \(\bbZ\)-module and the group homomrphism \(\alpha\) is a \(\bbZ\)-module homomorphism such that:
\[\alpha(n \cdot x) = n \cdot \alpha(x)\]
where \(n \in \bbZ\), \(x \in A \otimes_R M\).
Thus, \(\alpha \in \Hom_\bbZ(A \otimes_R M, \bbQ / \bbZ)\).

As we showed above,
\[
  \Hom_\bbZ(B \otimes_R M, \bbQ/\bbZ)
  \xrightarrow{- \circ (f \otimes_R \Id_M)}
  \Hom_\bbZ(A \otimes_R M, \bbQ/\bbZ)
  \rightarrow
  0
\]
is exact.
In other words, \(- \circ (f \otimes_R \Id_M) = - \circ (f \otimes_R M)\)
is surjective.
Because \(\alpha \in \Hom_\bbZ(A \otimes_R M, \bbQ/\bbZ)\),
there is \(\beta \in \Hom_\bbZ(B \otimes_R M, \bbQ/\bbZ)\)
such that
\(\beta \circ (f \otimes_R \Id_M) = \alpha\).
Then,
\begin{align*}
  \alpha(a)
  &= \beta ((f \otimes_R \Id_M)(a))
  \\&= \beta(0) = 0
\end{align*}
since \(a \in \ker (f \otimes_R \Id_M)\).
However, it's a contradiction, because \(\alpha\) is a homomorphism such that \(\alpha(a) \neq 0\).

Thus, \(\ker (f \otimes_R M)\) must be \(\{0\}\),
and \(M\) is flat.
\qedsq

\newpage
% =============================================================================
\section*{Problem 5}

Let \(R\) be a ring with unity and \(M\) be a left \(R\)-module.
Prove that \(M\) is a flat left \(R\)-module
if and only if
for any right ideal \(I\) of \(R\),
\[I \otimes_R M \to IM: \sum_{j} a_j \otimes m_j \mapsto \sum_{j} a_j m_j\]
is an isomorphism of abelian groups.
(Hint: Try to use Baer's criterion.)

\subsection*{Solution}

Note that \(R \otimes_R M \simeq M\). because for every \(r \in R\) and \(m \in M\), \(r \otimes m = 0_R \otimes (rm)\),
and
\[\sum_j r_j \otimes m_j = \sum_j 1_R \otimes (r_jm_j) = 1_R \otimes \sum_j r_jm_j\]
It implies, every element of \(R \otimes_R M\) can be expressed as \(1_R \otimes m\) for some \(m \in M\).
Thus \(R \otimes_R M \simeq M\), with an isomorphism \(\psi: R \otimes_R M \to M\) such that \(\psi(1_R \otimes m) = m\).

\subsection*{(\(\Longrightarrow\))}

Suppose that \(M\) is a flat left \(R\)-module.

Let \(I\) be an ideal of \(R\).

Then, take a short exact sequence,
\[\begin{tikzcd}
  0 \ar[->]{r} &
  I \ar[->, "\iota"]{r} &
  R \ar[->, "\pi"]{r} &
  R/I \ar[->]{r} &
  0
\end{tikzcd}\]
where \(\iota: I \to R\) is an injection, and \(\pi: R \to R/I\) be a \(R\) homomorphism such that \(\pi: r \mapsto r + I\).

Since \(M\) is flat, the below is exact:
\[\begin{tikzcd}
  0 \ar[->]{r} &
  I \otimes_R M \ar[->, "\iota \otimes_R M"]{r} &
  R \otimes_R M \ar[->, "\pi \otimes_R M"]{r} &
  R/I \otimes_R M \ar[->]{r} &
  0
\end{tikzcd}\]

Since \(\iota \otimes_R M = \iota \otimes_R \Id_M\), it's a map from \(I \otimes_R M\) to \(R \otimes_R M \simeq M\).
And because of the exactness, \(\iota \otimes_R M\) is injective.

Since \(\psi\) we defined above is an isomorphism, \(\eta = \psi \circ (\iota \otimes_R M)\) is an injective homomorphism, such that
\[\eta(\sum_j a_j \otimes m_j) = \psi(\sum_j a_j \otimes m_j) = \psi(1_R \otimes \sum_j a_j m_j) = \sum_j a_j m_j\]
where \(a_j \in I\) and \(m_j \in M\).

Also,
because, for each \(a_j \in I\) and \(m_j \in M\),
\(a_jm_j \in IM\)
and
\(\sum_j a_j m_j \in IM\),
and \(\eta(\sum_j a_j \otimes m_j) = \sum_j a_j m_j \in IM\).
And, for each \(\sum_j a_j m_j \in I M\),
there is \(\sum_j a_j \otimes m_j \in I \otimes_R M\) such that
\(\eta(\sum_j a_j \otimes m_j) = \sum_j a_j m_j\).
It means, the image of \(\eta\) is \(IM\).

Therefore, \(\tilde{\eta}: I \otimes_R M \to IM\), which obtained from \(\eta\) by restricting the codomain to \(IM\), which given in the problem, is an isomorphism.
\qedsq

\subsection*{(\(\Longleftarrow\))}
Suppose that for every right idel \(I \subseteq R\),
\[I \otimes_R M \to IM: \sum_{j} a_j \otimes m_j \mapsto \sum_{j} a_j m_j\]
is an isomorphism of abelian groups.
It means, \(I \otimes_R M \simeq IM\) for every right ideal \(I \subseteq R\).

Let \(M^* = \Hom_\bbZ(M, \bbQ/\bbZ)\) be \(R\)-module.
Then, by adjunction formula and above isomorphism relations,
\[\Hom_R(I, M^*)
  = \Hom_R(I, \Hom_\bbZ(M, \bbQ / \bbZ))
  \simeq \Hom_\bbZ(I \otimes_R M, \bbQ / \bbZ)
  \simeq \Hom_\bbZ(I M, \bbQ / \bbZ)
\]
\[\Hom_R(R, M^*)
  = \Hom_R(R, \Hom_\bbZ(M, \bbQ / \bbZ))
  \simeq \Hom_\bbZ(R \otimes_R M, \bbQ / \bbZ)
  \simeq \Hom_\bbZ(M, \bbQ / \bbZ)
\]

(Note: in this part, for any abelian group \(A\), \(A\) is a \(\bbZ\)-module such that \(n \cdot a\) is sum of \(a\) \(n\)-times.
Then, group homomorphisms between abelian groups are \(\bbZ\)-module homomorphisms.
)

Let \(g \in \Hom_R(I, M^*)\).
Then, there is \(h \in \Hom_\bbZ(IM, \bbQ / \bbZ)\), which is the isomorphic image of \(g\).
Since \(h\) is a group homomorphism from \(IM \le M\) to \(\bbQ / \bbZ\),
by Lemma \ref{lem-char},
there is an extension \(\tilde{h} \in \Hom_\bbZ(M, \bbQ / \bbZ)\) of \(h\).
Note that since \(M\) and \(\bbQ / \bbZ\) are abelian groups, \(\tilde{h}\) is a \(\bbZ\)-module homomorphism.
And, let \(\tilde{g} \in \Hom_R(R, M^*)\) be an isomorphic image of \(\tilde{h}\).
Then, \(\tilde{g}\) is an extension of \(g\).

In other words, for every \(R\)-module homomorphism \(g: I \to M^*\),
there is a homomorphism \(\tilde{g}: R \to M^*\) which extends \(g\).
Thus, by Baer's Criterion,
\(M^*\) is an injective module.

As we proved in Problem 4, since \(M^*\) is injective, \(M\) is flat
\qedsq
\end{document}
