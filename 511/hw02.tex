% MAS511 HW#02

\documentclass{article}

\usepackage{amssymb}
\usepackage{amsmath}
\usepackage{amsthm}
\usepackage{centernot}
\usepackage[a4paper, total={6in, 9in}]{geometry}
\usepackage{enumitem}
\usepackage{mathtools}
\usepackage{stmaryrd}
\usepackage{subfig}
\usepackage{kotex}
\usepackage{tikz-cd}
\usepackage{mathrsfs}

\newtheorem{definition}{Definition}
\newtheorem{theorem}{Theorem}
\newtheorem{principle}{Principle}
\newtheorem{lemma}{Lemma}
\newtheorem{corollary}{Corollary}
\newtheorem{axiom}{Axiom}

\newcommand{\bs}{\char`\\}
\newcommand{\br}{\vspace*{10px}\\}
\newcommand{\qedsq}{\hfill$\square$}

\newcommand{\triangleleftneq}{\mathrel{\ooalign{$\lneq$\cr\raise.22ex\hbox{$\lhd$}\cr}}}

\newcommand{\bbI}{\mathbb{I}}
\newcommand{\bbN}{\mathbb{N}}
\newcommand{\bbQ}{\mathbb{Q}}
\newcommand{\bbR}{\mathbb{R}}
\newcommand{\bbZ}{\mathbb{Z}}

\newcommand{\calA}{\mathcal{A}}
\newcommand{\calB}{\mathcal{B}}
\newcommand{\calC}{\mathcal{C}}
\newcommand{\calD}{\mathcal{D}}
\newcommand{\calF}{\mathcal{F}}
\newcommand{\calG}{\mathcal{G}}
\newcommand{\calH}{\mathcal{H}}
\newcommand{\calN}{\mathcal{N}}
\newcommand{\calf}{\mathcal{f}}
\newcommand{\calg}{\mathcal{g}}
\newcommand{\calh}{\mathcal{h}}

\newcommand{\scrP}{\mathscr{P}}

\newcommand{\Ab}{\textnormal{Ab}}
\newcommand{\Aut}{\textnormal{Aut}}
\newcommand{\Cod}{\textnormal{Cod}}
\newcommand{\Dom}{\textnormal{Dom}}
\newcommand{\Gal}{\textnormal{Gal}}
\newcommand{\Grp}{\textnormal{Grp}}
\newcommand{\Set}{\textnormal{Set}}
\newcommand{\Syl}{\textnormal{Syl}}
\newcommand{\op}{\textnormal{op}}
\newcommand{\Ob}{\textnormal{Ob}}
\newcommand{\Hom}{\textnormal{Hom}}
\newcommand{\Id}{\textnormal{Id}}

\title{MAS511 Spring 2020 Homework \#02}
\author{}

\begin{document}
\maketitle

Any ring homomorphism in this homework is assumed to preserve \(+, \cdot\) and \(1\).

% =============================================================================
\section*{Problem 1}
Let \(G\) be a finite group and \(p\) be a prime number dividing \(|G|\). Assuming Cauchy's theorem for the case of abelian \(G\), deduce Cauchy's theorem for general \(G\) by using the class equation for \(G\). (Hint: Use induction on \(|G|\).)

\begin{theorem}[Cauchy's]
  Let \(G\) be a finite group.
  \(G\) has an element of order \(p\)
  for every prime number \(p\) which divides \(|G|\).
\end{theorem}

\subsection*{Solution}

\begin{lemma}\label{lem-cen-sub}
  Let \(G\) be a group and \(S \subseteq G\). Then, the centralizer \(C_G(S)\) is a subgroup of \(G\).
  In particular, \(Z(G) = C_G(G) \le G\).
\end{lemma}
\begin{proof}
  To show that \(C_G(S) \le G\), it's enough to show that (1) \(C_G(S) \subseteq S\); (2) \(C_G(S) \neq \emptyset\); (3) \(\forall a, b \in C_G(S): a^{-1}b \in C_G(S)\).

  First, \(C_G(S) \subseteq G\),
  because by the definition \(C_G(S) = \{g \in G \mid \forall s \in S: gs = sg\}\).

  Next, 
  because \(1_G\) satisfies \(g 1_G = g = 1_G g\) for every \(g \in G\),
  \(s 1_G = 1_G s\) holds for every \(s \in S\).
  Therefore, \(1_G \in C_G(S)\) and \(C_G(S) \neq \emptyset\).

  Suppose that \(a, b \in C_G(S)\). It means, for every \(s \in S\), \(as = sa\) and \(bs = sb\) hold.
  Then,
  \begin{align*}
    a^{-1} b s
    = a^{-1} s b
    &= a^{-1} s a a^{-1} b
    \\&= a^{-1} a s a^{-1} b
    = s a^{-1} b
  \end{align*}
  Thus, for every \(s \in S\), \((a^{-1}b)s = s(a^{-1}b)\) also hold.
  It implies that \(a^{-1}b \in C_G(S)\).

  Therefore, \(C_G(S) \le G\).
\end{proof}

\noindent
Let \(G\) be a finite group,
and suppose that Cauchy's Theorem hold for abelian groups.

Let's use induction on \(|G|\) to prove:

Hypothesis: For every finite group \(G\), Cauchy's Theorem holds for \(G\).
\newline

\noindent
(Base case 1) Suppose that \(G\) is trivial. Then, there are no prime \(p \in \bbN\) such that \(p \mid |G| = 1\). Therefore, Cauchy's Theorem is vacuously true for a trivial group.
\newline

\noindent
(Base case 2) Suppose that \(|G| = p\) for some prime number \(p \in \bbN\).
Then, \(|G|\) is a prime number \(\Rightarrow\) \(G\) is a cyclic group \(\Rightarrow\) \(G\) is an abelian group.
Since we assumed Cauchy's Theorem for abelian groups, Cauchy's Theorem holds for \(G\), which is one of abelian groups.
\newline

\noindent
(Inductive step)
Let \(n \in \bbN\) be a non-prime number and \(G\) be a group of order \(n\).
Let's assume that Cauchy's Theorem holds for every group of order less than \(n\) as an induction hypothesis,
and show that Cauchy's Theorem holds for \(G\) using the hypothesis.

Let \(p\) be an arbitrary prime number such that \(p \mid n\).
Then, there is \(\alpha, m \in \bbN\) such that \(p^{\alpha}m = n\) and \(p \nmid m\).
Let's show that \(G\) contains an element of order \(p\) for this arbitrary \(p\).
\newline

First of all, suppose that \(|G| = |Z(G)|\).
Since \(Z(G) \le G\), \(G = Z(G) = \{g \in G \mid \forall h \in G: gh = hg\}\).
It means, for every \(g \in G\), \(g\) commutes every \(h \in G\).
Thus, \(G\) is abelian.
In this case, \(G\) has an element of order \(p\) by Cauchy Theorem for abelian groups.
\newline

Let's assume that \(|Z(G)| \neq |G|\). With the fact that \(Z(G) \le G\), \(|Z(G)| < |G|\) and \(Z(G) \lneq G\) hold.
Then, for some \(x \in G \bs Z(G)\), there is \(y \in G\) such that \(xy \neq yx\).
Then, \(\{g \in G \mid gxg^{-1}\}\) contains at least two elements \(x = 1x1^{-1}\) and \(yxy^{-1}\).
Therefore, \(G\) has at least one conjugation orbit of length greater than 1.
Let \(O_1, O_2, \cdots, O_L\) are distinct conjugation orbits.
And let \(x_1, x_2, \cdots, x_L\) such that \(x_i \in O_i\) for each \(i \in \{1, 2, \cdots, L\}\).
Then, the class equation holds:
\[|G| = |Z(G)| + \sum_{i=1}^{L} [G : C_G(x_i)]\]

Suppose that \(p \mid |Z(G)|\).
Then, because \(Z(G)\) is a group, \(p \mid |Z(G)|\), and \(|Z(G)| < |G| = n\), \(Z(G)\) has an element \(y\) of order \(p\) by the induction hypothesis.
And since \(Z(G) \lneq G\), \(y \in Z(G) \subsetneq G\).
It indicates that \(G\) has an element \(y\), which has order \(p\).
\newline

Suppose that \(p \nmid |Z(G)|\).
In other words, \(|Z(G)| \not\equiv 0 \pmod{p}\).
Then, because \(|G| \equiv 0 \pmod{p}\), \(|G| - |Z(G)| \not\equiv 0 \pmod{p}\).
Then, let's take a modulo by \(p\) on the class equation:
\begin{gather*}
  |Z(G)| + \sum_{i=1}^{L} [G : C_G(x_i)] = |G|
  \\\sum_{i=1}^{L} [G : C_G(x_i)] = |G| - |Z(G)|
  \\\sum_{i=1}^{L} [G : C_G(x_i)] \equiv |G| - |Z(G)| \not\equiv 0 \pmod{p}
\end{gather*}
Thus, \(\sum_{i=1}^{L} [G : C_G(x_i)] \not\equiv 0 \pmod{p}\).

In this case, \([G : C_G(x_i)] \not\equiv 0 \pmod{p}\) for at least one \(i \in \{1, 2, \cdots, L\}\). Becuase, if not, \(\forall i \in \{1, 2, \cdots, L\}: [G : C_G(x_i)] \equiv 0 \pmod{p}\) holds, and
\begin{align*}
  \sum_{i=1}^{L} [G : C_G(x_i)]
  &\equiv [G : C_G(x_1)] + \sum_{i=2}^{L} [G : C_G(x_i)]
  \equiv 0 + \sum_{i=2}^{L} [G : C_G(x_i)]
  \\&\equiv [G : C_G(x_2)] + \sum_{i=3}^{L} [G : C_G(x_i)]
  \equiv 0 + \sum_{i=3}^{L} [G : C_G(x_i)]
  \\&\equiv \cdots
  \equiv 0 + [G : C_G(x_L)]
  \equiv 0 + 0
  \equiv 0 \pmod{p}
\end{align*}
But it's a contradiction because \(\sum_{i=1}^{L} [G : C_G(x_i)] \not\equiv 0 \pmod{p}\).

Let \(k \in \{1, 2, \cdots, L\}\) such that \([G : C_G(x_k)] \not\equiv 0 \pmod{p}\).
Then, \([G : C_G(x_k)] = \frac{|G|}{|C_G(x_k)|}\) does not contain \(p\) as a factor.
However, because \(|G| = p^{\alpha}m\), \(|C_G(x_k)| = p^{\alpha}l\)
where \(l \le m\) and \(p \nmid l\).
Thus, \(p \mid |C_G(x_k)|\).

Also, \(|C_G(x_k)| < |G|\).
It's because, as we chose \(x_k\) to be a representative of some conjugation orbit of length greater than \(1\),
\begin{gather*}
  [G : C_G(x_k)] > 1
  \\ \frac{|G|}{|C_G(x_k)|} > 1
  \\ |G| > |C_G(x_k)|
\end{gather*}

Therefore, \(p \mid |C_G(x_k)|\) and \(|C_G(x_k)| < |G| = n\).
It means, \(C_G(x_k)\) contains an element \(y\) of order \(p\) by the induction hypothesis.
And, by Lemma \ref{lem-cen-sub}, \(C_G(x_k) \le G\).
Thus, \(y \in C_G(x_k) \subsetneq G\).
Therefore, \(G\) has \(y\), which is an element of order \(p\).
\newline

\noindent
In conclusion, if Cauchy's Theorem holds for every groups of order less than some non-prime number \(n\), Cauchy's Theorem also holds for every group of order \(n\).
\newline

\noindent
Thus, by induction, for every group \(G\) such that \(|G| \in \bbN\), Cauchy's Theorem holds for \(G\).
\qedsq

\newpage
% =============================================================================
\section*{Problem 2}
Let \(p\) and \(q\) be a prime numbers with \(p < q\). Prove that any group \(G\) of order \(pq^n\) for some integer \(n \ge 1\) is solvable.

\subsection*{Solution}

\begin{theorem}[Sylow Theorem]
  Let \(G\) be a group of order \(p^\alpha m\), where \(p\) is a prime not dividing \(m\) and \(\alpha \in \bbN\).
\begin{enumerate}[label=(\arabic*)]
\item \(\Syl_p(G) \neq \emptyset\).
\item Let \(P \in \Syl_p(G)\) and \(Q \le G\) be a \(p\)-subgroup. There exists \(g \in G\) such that \(Q \le gPg^{-1}\).
\item \(|\Syl_p(G)| = n_p \equiv 1 \pmod{p}\) and \(n_p \mid m\).
\end{enumerate}
\end{theorem}

\begin{lemma}\label{lem-orb-fix}
  Let \(G\) be a finite group acts on a set \(X\).
  Let \(x \in X\), \(G \cdot x = \{g \cdot x \mid g \in G\}\) be the orbit of \(x\), and \(G_x = \{g \in G \mid g \cdot x = x\}\) be the stabilizer of \(x\).
  Then,
  \begin{enumerate}
  \item \(G_x \le G\).
  \item \(|G \cdot x| = [G : G_x]\)
  \item \(|G \cdot x|\) and \(|G_x|\) divides \(|G|\)
  \end{enumerate}
\end{lemma}
\begin{proof}
1.
  By definition, \(G_x \subseteq G\). And \(1_G \in G_x\) because an identity of \(G\) fixes every element of \(X\). Lastly, if \(a, b \in G_x\), they satisfies \(a \cdot x = x\) and \(b \cdot x = x\), and,
  \[ a^{-1}b \cdot x = a^{-1} \cdot x = a^{-1} \cdot a \cdot x = x\]
  Thus, \(a^{-1}b \in G_x\). These results indicate that \(G_x \le G\).

2. Let \(L\) be a set of cosets of \(G_x\) in \(G\).
  Let's define \(\varphi: G \cdot x \to L\) such as
  \(\varphi: y \mapsto g G_x\) where \(g \in G\) satisfies \(g \cdot x = y\).
  Then,
  \begin{itemize}
  \item \(\varphi\) is well-defined.
    If \(y \in G \cdot x\), it means there is \(g \in G\) such that \(g \cdot x = y\) holds. Also, if \(g, h \in G\) satisfies \(g \cdot x = h \cdot x = y\), then \(h^{-1}g \cdot x = x\). Since \(h^{-1}g\) fixes \(x\), \(h^{-1}g \in G_x\). Thus \(h^{-1}g G_x = G_x\) and \(g G_x = h G_x\).
  \item \(\varphi\) is injective.
    Let \(y, z \in G \cdot x\) satisfies \(\varphi(y) = \varphi(z)\).
    It means, \(g_1 \cdot x = y\), \(g_2 \cdot x = z\), and \(g_1 G_x = g_2 G_x\).
    Since \(g_1 G_x = g_2 G_x\) implies \(g_2^{-1}g_1 \in G_x\) (i.e. \(g_2^{-1}g_1\) fixes \(x\)),
    \[y = g_1 \cdot x = g_2 \cdot g_2^{-1} g_1 \cdot x = g_2 \cdot x = z\]
  \item \(\varphi\) is surjective.
    Let \(g G_x \in L\). If we take \(y = g \cdot x\), \(y \in G \cdot x\) and \(\varphi(y) = g G_x\).
  \end{itemize}
  Thus, \(\varphi\) is a well-defined bijective map.
  Because \(G\) is a finite group, \(|L| = [G : G_x]\) is also finite.
  And because there is a bijective map between \(L\) and \(G \cdot x\),
  \(|G \cdot x| = |L| = [G : G_x]\) holds.

3. Since \(|G \cdot x| = [G : G_x] = \frac{|G|}{|G_x|}\), \(|G \cdot x| \cdot |G_x| = |G|\). Thus, \(|G \cdot x|\) and \(|G_x|\) divides \(|G|\).
\end{proof}

\begin{lemma}[Sylow Theorem (1)\('\)]\label{lem-syl-1}
  Let \(G\) be a group of order \(p^\alpha m\), where \(p\) is a prime not dividing \(m\) and \(\alpha \in \bbN\). Then, there exists \(H_1, H_2, \cdots, H_\alpha\) such that
  \[\{1\} = H_0 \triangleleftneq H_1 \triangleleftneq H_2 \triangleleftneq \cdots \triangleleftneq H_\alpha\]
  where \([H_i : H_{i - 1}] = p\) for all \(i \in \{1, 2, \cdots, \alpha\}\) and \(H_\alpha \in \Syl_p(G)\).
\end{lemma}
\begin{proof}
  Let's construct \(H_1, \cdots, H_\alpha\) in two steps:
  (1) constructing \(H_1\),
  (2) constructing \(H_{i + 1}\) when \(H_i\) was constructed for \(i \in \{1, 2, \cdots, \alpha - 1\}\).

  (1) We can easily find \(H_1\). By Cauchy Theorem, \(G\) contains an element \(x\) of order \(p\). Let \(H_1 = \langle x \rangle\). Then \(\{1\} \triangleleftneq H_1 = \langle x \rangle\), and \([H_1: \{1\}] = |H_1| = p\).

  (2) Suppose that we already constructed \(H_1, \cdots, H_i\) for some \(i \in \{1, 2, \cdots, \alpha - 1\}\), such that
  \(\{1\} = H_0 \triangleleftneq H_1 \triangleleftneq H_2 \triangleleftneq \cdots \triangleleftneq H_i\), and \([H_{k} : H_{k - 1}] = p\) for every \(k \in \{1, 2, \cdots, i\}\).
  Note that
  \[|H_i| = |H_0|\prod_{k = 1}^{i} \frac{|H_k|}{|H_{k - 1}|}
  = 1 \cdot \prod_{k = 1}^{i} [H_k : H_{k - 1}] = \prod_{k = 1}^{i} p = p^i\]

  Let \(L = \{gH_i \mid g \in G\}\), the set of left cosets of \(H_i\).
  Then, \(|L| = [G : H_i]\).
  Also, we will consider that \(H_i\) acts on \(L\) by left multiplication.

  Let \(L_{H_i} = \{S \in L \mid \forall h \in H_i : hS = S\}\).
  Then, for any \(a \in G\),
  \begin{align*}
    aH_i \in L_{H_i}
    &\Leftrightarrow \forall h \in H_i: haH_i = aH_i
    \\&\Leftrightarrow \forall h \in H_i: a^{-1}haH_i = H_i
    \\&\Leftrightarrow \forall h \in H_i: a^{-1}ha \in H_i
    \\&\Leftrightarrow a^{-1}H_ia \subseteq H_i
    \\&\Leftrightarrow a^{-1}H_ia = H_i
    \;\;(\because H_i\text{ is a finite group, and } |H_i| = |a^{-1}H_ia| \text{ holds})
    \\&\Leftrightarrow a \in N_G(H_i)
    \\&\Leftrightarrow aH_i \in N_G(H_i) / H_i
  \end{align*}
  holds.
  This shows that \(L_{H_i} = N_G(H_i) / H_i\), and \(|L_{H_i}| = [N_G(H_i) : H_i]\).

  Let \(O_{S} = \{h S \mid h \in H_i\}\), the orbit of \(S \in L\).
  Let \(S_1, \cdots, S_n \in L\) are the representatives of distinct orbits; i.e. \(O_{S_1}, \cdots, O_{S_n}\) are disjoint pairwisely and \(\bigcup_{k = 1}^{n} O_{S_k} = L\).
  Then,
  \[|L| = \sum_{k=1}^{n} |O_{S_k}|\]
  holds.
  
  Suppose that \(|O_{S_k}| = 1\) for some \(k \in \{1, \cdots, n\}\).
  It means \(h S_k = S_k\) for all \(h \in H_i\).
  Then \(S_k \in L_{H_i}\) by the definition of \(L_{H_i}\).
  Conversely, if \(S_k \in L_{H_i}\), then \(\forall h \in H_i: h S_k = S_k\), \(O_{S_k} = \{S_k\}\), and \(|O_{S_k}| = 1\).
  Thus, for \(I = \{k \in \{1, 2, \cdots, n\} \mid |O_{S_k}| = 1\}\), \(L_{H_i} = \{S_i \mid i \in I\}\) and \(|I| = |L_{H_i}|\).

  Let \(k \in \{1, 2, \cdots, n\} \bs I\). Then, \(|O_{S_k}| \neq 0\) because \(S_k \in O_{S_k}\), and \(|O_{S_k}| \neq 1\) because \(k \not\in I\). Thus, \(|O_{S_k}| \ge 2\). But by Lemma \ref{lem-orb-fix}, \(|O_{S_k}| \mid |H_i| = p^i\). This two facts show that \(p \mid |O_{S_k}|\) must hold for \(k \in \{1, \cdots, n\} \bs I\).

  So, we obtain:
  \begin{align*}
    |L| = \sum_{k=1}^{n} |O_{S_k}|
    &= \sum_{k \in I} |O_{S_k}| + \sum_{k \in \{1, \cdots, n\} \bs I} |O_{S_k}|
    \\&= \sum_{k \in I} 1 + \sum_{k \in \{1, \cdots, n\} \bs I} |O_{S_k}|
    \\&= |I| + \sum_{k \in \{1, \cdots, n\} \bs I} |O_{S_k}|
    \\&= |L_{H_i}| + \sum_{k \in \{1, \cdots, n\} \bs I} |O_{S_k}|
  \end{align*}
  and
  \begin{align*}
    [G : H_i] = |L|
    &\equiv |L_{H_i}| + \sum_{k \in \{1, \cdots, n\} \bs I} |O_{S_k}|
    \\&\equiv |L_{H_i}| = [N_G(H_i) : H_i] \pmod{p}
  \end{align*}
  Also, \(p \mid [G : H_i] = p^{\alpha - i} m\) holds since \(|H_i| = p^i\) and \(i < \alpha\). Thus \([N_G(H_i) : H_i] \equiv [G : H_i] \equiv 0 \pmod{p}\).
  But because \([N_G(H_i) : H_i] > 0\), \([N_G(H_i) : H_i] \ge p > 1\).
  This indicates that \(H_i \triangleleftneq N_G(H_i)\).
  \newline

  Let \(Q = N_G(H_i) / H_i\), and the canonical surjective homomorphism \(\gamma: N_G(H_i) \to Q\) such that \(\gamma: h \mapsto h H_i\).
  Because \(p \mid |Q|\), by Cauchy Theorem, there exists \(x \in Q\) of order \(p\).
  Trivially, we know that \(\{H_i\} \triangleleftneq \langle x \rangle\).
  Let's define \(H_{i + 1} = \gamma^{-1}(\langle x \rangle)\).
  Then, \(H_{i} = \gamma^{-1}(\{H_i\}) \triangleleftneq \gamma^{-1}(\langle x \rangle) = H_{i + 1}\) because of the fourth isomorphism theorem (lattice theorem).
  In addition,
  \[[H_{i + 1} : H_i] = |\langle q \rangle : \{H_i\}| = |\langle x \rangle| = p\]

  Therefore, this \(H_{i + 1}\) satisfies \(H_i \triangleleftneq H_{i + 1}\) and \([H_{i + 1} : H_i] = p\).
  \newline

  By following (1) and repeating (2) for \(\alpha - 1\) times, we obtain \(H_1, H_2, \cdots, H_\alpha\) which satisfies
  \[\{1\} = H_0 \triangleleftneq H_1 \triangleleftneq H_2 \triangleleftneq \cdots \triangleleftneq H_\alpha\]
  and \([H_i : H_{i - 1}] = p\) for every \(i \in \{1, 2, \cdots, \alpha\}\).
  Also, because
  \[|H_\alpha| = |H_0|\prod_{i = 1}^{\alpha} \frac{|H_i|}{|H_{i - 1}|}
  = 1 \cdot \prod_{i = 1}^{\alpha} [H_i : H_{i - 1}] = \prod_{i = 1}^{\alpha} p
  = p^\alpha\]
  \(H_\alpha \in \Syl_p(G)\) holds.
\end{proof}

\begin{lemma}[Sylow Theorem (2)\('\)]\label{lem-syl-2}
  Let \(G\) be a group of order \(p^\alpha m\), where \(p\) is a prime not dividing \(m\). \(\Syl_p(G) = \{H\}\) implies \(H \trianglelefteq G\).
\end{lemma}
\begin{proof}
  Suppose that \(\Syl_p(G) = \{H\}\). Then, for any \(g \in G\), \(gHg^{-1} \in \Syl_p(G)\).

  (\(\because\)
  Let \(\phi_g: G \to G\) such that \(\phi_g: x \mapsto gxg^{-1}\) for \(g \in G\).
  \(\phi_g\) is surjective because for any \(h \in G\), \(g^{-1}hg \in G\) and \(\phi_g(g^{-1}hg) = gg^{-1}hgg^{-1} = h\).
  \(\phi_g\) is injective because for any \(h_1, h_2 \in G\), \(\phi_g(h_1) = \phi_g(h_2)\) implies \(gh_1g^{-1} = gh_2g^{-1}\) and \(h_1 = h_2\) by multiplicating \(g^{-1}\) on left and \(g\) on right.
  Thus, \(\phi_g\) is bijective and \(|gHg^{-1}| = |\phi_g(H)| = |H| = p^\alpha\).
  Therefore, \(gHg^{-1}\) is also a subgroup of \(G\) of order \(p^\alpha\).
  )

  But since \(\Syl_p(G)\) is a singleton containing \(H\), \(H \in \Syl_p(G)\) implies \(H = gHg^{-1}\) for arbitrary \(g \in H\). Thus, \(H\) is a normal subgroup of \(G\).
\end{proof}

Let's think about Sylow \(q\)-group of \(G\).

First of all, by Sylow Theorem (3), for \(n_q = |\Syl_q(G)|\), \(n_q \equiv 1 \pmod{p}\) and \(n_q \mid p\). It implies there is \(k \in \bbZ^{\ge 0}\) such that \(n_q = 1 + kq\) and \(1 + kq \mid p\). But if \(k \ge 1\), then \(1 + kq \ge 1 + q > q > p\) and \(n_q = 1 + kq \nmid p\). Therefore, \(k\) must be 0 and \(n_q = 1\). Let \(\Syl_q(G) = \{H\}\).

In this case, by Lemma \ref{lem-syl-2}, \(H \trianglelefteq G\) and \(|H| = q^n\).

And, by Lemma \ref{lem-syl-1}, there is a subnormal series
\[\{1\} = H_0 \triangleleftneq H_1 \triangleleftneq H_2 \triangleleftneq \cdots \triangleleftneq H_n\]
where \([H_{i} : H_{i - 1}] = q\) for all \(i \in \{1, 2, \cdots, n\}\) and \(H_n \in \Syl_q(G)\).
However, since \(\Syl_q(G)\) is a singleton of \(H\), \(H_n = H\).
Thus, \(\{1\} = H_0, H_1, \cdots, H_n = H, G\) form a subnormal series:
\[\{1\} = H_0 \triangleleftneq H_1 \triangleleftneq H_2 \triangleleftneq \cdots \triangleleftneq H_n = H \triangleleftneq G\]
where \([H_{i} : H_{i - 1}] = q\) for all \(i \in \{1, 2, \cdots, n\}\) and \([G : H] = p\).
Because \(p\) and \(q\) are prime numbers, \(H_i / H_{i - 1}\) for \(i \in \{1, 2, \cdots, n\}\) and \(G / H\) are cyclic groups, which are abelian.
This shows that \(G\) is solvable.
\qedsq

\newpage
% =============================================================================
\section*{Problem 3}
Let \(X\) be a set of cardinality at least \(2\). Prove that \(\Aut(F(X))\), the group of all group automorphisms of the free group \(F(X)\), is not a nilpotent group. (Hint: Compute \(Z(F(X))\) by looking at the number of non-\(e\)-letters of a reduced word in it.)

\subsection*{Solution}

\begin{lemma}\label{lem-conj-aut}
  Let \(G\) be a group and \(g \in G\).
  Then, for any \(g \in G\), a conjugation \(f_g: G \to G\) defined as
  \(f_g: x \mapsto gxg^{-1}\) is an automorphism of \(G\).
\end{lemma}
\begin{proof}
  \(f_g\) is an endomorphism by the definition.

  \(f_g\) is a homomorphism. Because, for any \(x, y \in G\),
  \[f_g(x) f_g(y) = gxg^{-1}gyg^{-1} = gxyg^{-1} = f_g(xy)\]

  \(f_g\) is surjective. Because, for any \(x \in G\), \(g^{-1}xg \in G\) since \(G\) is closed under \(\cdot\) and \(f_g(g^{-1}xg) = gg^{-1}xgg^{-1} = x\).

  \(f_g\) is injective. Because \(\ker f_g = \{1_G\}\).
  (Let \(x\) be an arbitrary element of \(G\).
  If \(x \in \ker f_g\), \(f_g(x) = gxg^{-1} = 1_G\) must hold.
  Then, \(x = g^{-1} 1_G g = g^{-1} g = 1_G\).
  Therefore, \(1_G\) is the only element of \(\ker f_g\).)
\end{proof}

\begin{lemma}\label{lem-cen-aut}
  Let \(G\) be a group. If \(Z(G) = \{1_G\}\), \(Z(\Aut(G)) = \{\Id_G\}\).
\end{lemma}
\begin{proof}
  Suppose that \(Z(G) = \{1_G\}\), and let \(\varphi \in Z(\Aut(G))\).
  Then, for every \(\psi \in \Aut(G)\), \(\psi \circ \varphi = \varphi \circ \psi\) must hold.
  Because a conjugation \(f_g: G \to G\) such that \(x \mapsto gxg^{-1}\) for \(g \in G\) is an automorphism,
  \(f_g \circ \varphi = \varphi \circ f_g\) must hold for every \(g \in G\).
  Then, for every \(h \in G\),
  \begin{align*}
    g\varphi(h)g^{-1} &= f_g(\varphi(h))
    \\&= \varphi(f_g(h)) = \varphi(ghg^{-1})
    = \varphi(g)\varphi(h)\varphi(g^{-1}) = \varphi(g)\varphi(h)\varphi(g)^{-1}
  \end{align*}
  must holds. And if the above holds, the below equation also holds:
  \begin{align*}
    \varphi(g)^{-1}g \varphi(h) = \varphi(h) \varphi(g)^{-1}g
  \end{align*}
  Because \(\varphi\) is an automorphism, \(G = \varphi(G)\).
  Thus, \[\forall h \in G: (\varphi(g)^{-1}g) \varphi(h) = \varphi(h) (\varphi(g)^{-1}g)\]
  is equivalent to \[\forall h \in G: (\varphi(g)^{-1}g) h = h (\varphi(g)^{-1}g)\]
  Then,
  \(\varphi(g)^{-1}g \in Z(G)\)
  since the definition of \(Z(G)\) is \(\{x \in G \mid \forall h \in G : xh = hx \}\).
  However, as \(Z(G) = \{1_G\}\), \(\varphi(g)^{-1}g = 1_G\) and \(g = \varphi(g)\).
  In summary, if \(\varphi \in Z(\Aut(G))\), then \(g = \varphi(g)\) for all \(g \in G\) at the least.
  However, if \(\forall g \in G: g = \varphi(g)\), then \(\varphi = \Id_G\) since \(G\) is the domain of \(\varphi\).
  Therefore, \(\varphi = \Id_G\) if \(\varphi \in Z(\Aut(G))\).
  This indicates that \(Z(\Aut(G)) \subseteq \{\Id_G\}\).

  Also, \(\Id_G \in Z(\Aut(G))\),
  because for every \(\psi \in \Aut(G)\), \(\psi \circ \Id_G = \psi = \Id_G \circ \psi\).

  Therefore, \(Z(\Aut(G)) = \{\Id_G\}\).
\end{proof}

\begin{lemma}\label{lem-nil-z-1}
  Let \(G\) be a non-trivial group. If \(Z(G) = \{1_G\}\), \(G\) is not nilpotent.
\end{lemma}
\begin{proof}
  Let \(Z_0(G) = \{1_G\}\), \(Z_1(G) = Z(G)\), \(Z_{i}(G) \le G\) such that \(Z_i / Z_{i - 1} \simeq Z(G / Z_{i - 1})\) for integer \(i \ge 2\).
  Then,
  \[\{1_G\} = Z_0(G) \trianglelefteq Z_1(G) \trianglelefteq Z_2(G) \trianglelefteq \cdots \trianglelefteq Z_n(G) \trianglelefteq \cdots\]
  If \(G\) is nilpotent, then there exists \(n \in \bbN\) such that \(Z_n(G) = G\).

  Suppose that \(Z(G) = \{1_G\}\).
  Then, \(Z_1(G) = Z(G) = \{1_G\}\).

  For \(i \in \bbN\), suppose that \(Z_i(G) = \{1_G\}\).
  Then, \(G / Z_i(G) = G / \{1_G\} \simeq G\),
  and \(Z(G / Z_i(G)) \simeq Z(G) = \{1_G\}\).
  Thus, \(Z_{i + 1}(G) / Z_i(G) \simeq Z(G / Z_i(G)) \simeq \{1_G\}\).
  It implies that \(Z_{i + 1}(G) / Z_i(G)\) is a trivial group, and \(Z_{i + 1}(G) = Z_i(G)\).
  Therefore, \(Z_{i + 1}(G) = Z_i(G) = \{1_G\}\).

  In this case, \(\forall n \in \bbN: Z_n(G) = \{1_G\}\) by induction with \(Z_1(G) = \{1_G\}\) (base case) and \(\forall i \in \bbN: Z_i(G) = \{1_G\} \Rightarrow Z_{i + 1}(G) = \{1_G\}\) (inductive step).
  Also, because \(G\) is non-trivial, there is no \(n \in \bbN\) such that \(Z_n(G) = G\), becuase \(Z_n(G) = \{1_G\} \neq G\) for every \(n \in \bbN\).
  Therefore, \(G\) cannot be nilpotent.
\end{proof}

Let's consider \(F(X)\) as a group of reduced words generated by \(X\). \(e \in F(X)\) is an identity (empty word). And the length of \(w \in F(X)\) is \(n\) for a reduced word \(w = x_1 x_2 \cdots x_n\) of \(F(X)\). The length of empty word \(e\) is 0. Also, note that:
\[Z(F(X)) = \{w \in F(X) \mid \forall x \in F(X): wx = xw\}\]

Since \(|X| \ge 2\), there exists at least two distinct \(x, y \in X\).

Suppose that \(w \in F(X)\) such that the length of \(w\) is greater than 0.

Suppose that the first character of reduced \(w\) is \(x\), there is \(w' \in F(X)\) such that \(w = xw'\). Then, for \(u = yw \in F(X)\),
\[ uw = (yxw')(xw') \neq (xw')(yxw') = wu \]
because the first characters are different.
It indicates that there is a word \(u\) in \(F(X)\) such that \(uw \neq wu\).

Suppose that the first character of reduced \(w\) is \(x^{-1}\), there is \(w' \in F(X)\) such that \(w = x^{-1}w'\). Then, for \(u = yw \in F(X)\),
\[ uw = (yx^{-1}w')(x^{-1}w') \neq (x^{-1}w')(yx^{-1}w') = wu \]
because the first characters are different.
It indicates that there is a word \(u\) in \(F(X)\) such that \(uw \neq wu\).

Suppose that the first character of reduced \(w\) is neither \(x\) nor \(x^{-1}\). Then, by taking \(u = xw \in F(X)\),
\[ uw = (xw)w \neq w(xw) = wu \]
because the first characters are different.
It also indicates that there is a word \(u\) in \(F(X)\) such that \(uw \neq wu\).

Therefore, if the length of \(w\) is greater than 0, there is a word \(u \in F(X)\) such that \(uw \neq wu\).
It implies that every element of \(Z(F(X))\) has 0-length.

The 0-length word of \(F(X)\), \(e\), satisfies \(\forall u \in F(X) : ue = u = eu\), because \(e\) is an identity.

In conclusion,
\[Z(F(X)) = \{e\}\]
and \(F(X)\) cannot be nilpotent.
\newline

\noindent
By Lemma \ref{lem-cen-aut}, \(Z(\Aut(F(X))) = \{\Id_{F(X)}\}\) because \(Z(F(X)) = \{e\}\).
Note that \(\Aut(F(X))\) is non-trivial.
(\(\because\) The conjugation \(f_x\) of \(x\) is an automorphism by Lemma \ref{lem-conj-aut},
and \(f_x\) is not an identity because \(f_x(y) = xyx^{-1}\) is already reduced and \(f_x(y) = xyx^{-1} \neq y\).
Thus, \(\Aut(F(X))\) has at least two automorphisms: \(\Id_{F(X)}\) and \(f_x\).)
By Lemma \ref{lem-nil-z-1}, \(\Aut(F(X))\) is not nilpotent because \(\Aut(F(X))\) is non-trivial and \(Z(\Aut(F(X)))\) is trivial.
\qedsq

\newpage
% =============================================================================
\section*{Problem 4}
Let \(R\) be a nonzero integral domain and \(D\) be the set of all nonzero elements of \(R\). Let \(i: R \to D^{-1}R\) be the injective ring homomorphism defined by \(i(r) = r/1\). Prove \(D^{-1}R\) is the `smallest' field containing \(R\) in the following sense: \(D^{-1}R\) is a field, and for any injective ring homomorphism \(f: R \to K\) for some field \(K\), there exists a unique injective ring homomorphism \(g: D^{-1}R \to K\) such that \(g \circ i = f\).

\subsection*{Solution}

Let \(R\) be a non-zero integral domain, and \(D = R \bs \{0\}\).
Let \(i: R \to D^{-1}R\) be a injective ring homomorphism such that \(\forall r \in R: i(r) = r/1\).

Suppose that \(K\) be a field, and \(f: R \to K\) be an injective ring homomorphism.

Let \(g: D^{-1}R \to K\) be a function such that \(g\left(\frac{r}{d}\right) = f(d)^{-1}f(r)\) for \(r \in R\), \(d \in D\). Then,
\begin{enumerate}[label=(\roman*)]
\item \(g\) is well-defined.
  
  Suppose that \(r_1, r_2 \in R\) and \(d_1, d_2 \in D\) satisfies \(\frac{r_1}{d_1} = \frac{r_2}{d_2}\).
  It means, \(r_1d_2 = r_2d_1\) as \(R\) is an integral domain.
  Then, \(f(r_1d_2) = f(r_2d_1)\).
  Since \(f\) is a ring homomorphism,
  \(f(r_1)f(d_2) = f(r_2)f(d_1)\).
  Also, since \(f\) is injective and \(f(0) = 0_K\), \(f(r) \neq 0_K\) for every non-zero \(r \in R\).
  So \(f(d_1), f(d_2)\) are non-zero, thus units, and
  \(f(d_1)^{-1}f(r_1) = f(d_2)^{-1}f(r_2)\) holds.
  By the definition of \(g\), \(g(r_1 / d_1) = g(r_2 / d_2)\).

  Therefore, \(g\) is well-defined.

\item \(g(d / d) = f(1)\) for \(d \in D\).

   By definition, \(g(d / d) = f(d)^{-1}f(d) = f(d)^{-1}f(d)f(1) = f(1)\) holds.

\item \(f(1) g(1 / d)^{-1} = g(d / 1)\) for \(d \in D\).

  \(f(1) = g(d / d) = g((d / 1) (1 / d)) = g(d / 1) g(1 / d)\).
  Thus \(f(1) g(1 / d)^{-1} = g(d / 1)\).

\item \(g\) is a ring homomorphism.

  Let \(r_1, r_2 \in R\), \(d_1, d_2 \in D\), and \(q_1 = r_1 / d_1\), \(q_2 = r_2 / d_2\).

  \begin{align*}
    g(q_1 + q_2)
    = g(\frac{r_1}{d_1} + \frac{r_2}{d_2})
    &= g(\frac{d_2 r_1 + d_1 r_2}{d_1 d_2})
    \\&= f(d_1 d_2)^{-1} f(d_2 r_1 + d_1 r_2)
    \\&= f(d_1)^{-1} f(d_2)^{-1} (f(d_2)f(r_1) + f(d_1)f(r_2))
    \\&= f(d_1)^{-1} f(d_2)^{-1} f(d_2)f(r_1) + f(d_1)^{-1} f(d_2)^{-1} f(d_1)f(r_2)
    \\&= f(d_1)^{-1} f(r_1) + f(d_2)^{-1} f(r_2)
    = g(\frac{r_1}{d_1}) + g(\frac{r_2}{d_2})
    = g(q_1) + g(q_2)
  \end{align*}

  \begin{align*}
    g(q_1 q_2)
    = g(\frac{r_1}{d_1} \frac{r_2}{d_2})
    &= g(\frac{r_1 r_2}{d_1 d_2})
    \\&= f(d_1 d_2)^{-1} f(r_1 r_2)
    \\&= f(d_1)^{-1} f(d_2)^{-1} f(r_1) f(r_2)
    \\&= (f(d_1)^{-1} f(r_1)) (f(d_2)^{-1} f(r_2))
    = g(\frac{d_1}{r_1}) g(\frac{d_2}{r_2})
    = g(q_1) g(q_2)
  \end{align*}

  Also, since \(i\) and \(f\) are ring homomorphisms, \(i(1) = 1/1\) and \(f(1)\) are unities. And, \(g(1 / 1) = g(i(1)) = f(1)\) holds. Thus \(g\) also preserve unity.

  Thus, \(g\) is a ring homomorphism.

\item \(g\) is injective.

  It's enough to show that \(\ker g = \{0\}\).

  Let \(r \in R\) and \(d \in D\) such that \(g(r / d) = 0_K\).
  Then, \(g(r/d) = f(d)^{-1} f(r) = 0_K\),
  and \(f(r) = 0_K\) and \(f(r) \in \ker f\) because \(f(d)\) is non-zero in the field \(K\).
  Since \(f\) is injective, \(\ker f = \{0\}\).
  Thus, \(f(r) = 0_K\) implies \(r = 0\).
  Therefore, \(\ker g = \{0\}\).

\item \(g \circ i = f\).

  Let \(r \in R\). Then,
  \begin{align*}
    g(i(r)) &= g(r / 1) = f(1)^{-1} f(r) = f(1)^{-1} f(r \cdot 1) = f(1)^{-1} f(r) f(1) = f(r)
  \end{align*}

\end{enumerate}

(i), (iv), (v), (vi) implies that \(g\) is a well-defined injective ring homomorphism such that \(g \circ i = f\).
\newline

\noindent
Suppose that \(g': D^{-1}R \to K\) is an injective ring homomorphism such that \(g' \circ i = f\).
Let \(r/d \in D^{-1}R\) such that \(r \in R\), \(d \in D\).

\begin{align*}
  g(r/d) = g(r/1) g(1/d)
  &= g(r/1) g(d/1)^{-1} g(d/1) g(1/d)
  \\&= g(r/1) g(d/1)^{-1} g(d/d)
  \\&= g(r/1) g(d/1)^{-1} g(1/1)
  \\&= g(i(r)) g(i(d))^{-1} g(i(1))
  \\&= f(r) f(d)^{-1} f(1)
  \\&= f(r \cdot 1) f(d)^{-1}
  \\&= f(r) f(d)^{-1}
\end{align*}

\begin{align*}
  g'(r/d) = g'(r/1) g'(1/d)
  &= g'(r/1) g'(d/1)^{-1} g'(d/1) g'(1/d)
  \\&= g'(r/1) g'(d/1)^{-1} g'(1/1)
  \\&= g'(i(r)) g'(i(d))^{-1} g'(i(1))
  \\&= f(r) f(d)^{-1} f(1)
  \\&= f(r) f(d)^{-1}
\end{align*}

Therefore,
\[g(r/d) = f(r) f(d)^{-1} = g'(r/d)\]
holds for every \(r/d \in D^{-1}R\). Thus, \(g = g'\).
\newline

\noindent
In conclusion, for any injective ring homomorphism \(f: R \to K\) for some field \(K\), there is an injective ring homomorphism \(g: D^{-1}R \to K\) such that \(g \circ i = f\), and \(g\) satisfies UMP in the sense that if \(g': D^{-1}R \to K\) is an injective ring homomorphism such that \(g' \circ i = f\), \(g = g'\) holds.
\qedsq

\newpage
% =============================================================================
\section*{Problem 5}
Let \(R\) be a principle ideal domain. Prove that there exists only one maximal ideal in \(R\) if and only if there exists \(r \in R\) such that any nonzero proper ideal of \(R\) is equal to \(\left(r^n\right)\) for some integer \(n \ge 1\).

\subsection*{Solution}

Let \(R\) be a PID. Because \(R\) is a PID \(\Rightarrow\) UFD \(\Rightarrow\) a commutative ring with unity, every ideal in \(R\) can be represented as \((a) = aR\) for some \(a \in R\), and every element of \(R\) has a unique factorization up to association.

\subsubsection*{(\(\Longrightarrow\))}

Suppose that \(R\) has only one maximal ideal, and let \(M = (r)\) be the maximal ideal of \(R\).

Note that, \(r\) is prime iff \(r\) is irreducible iff an ideal \((r)\) is prime iff an ideal \((r)\) is maximal for \(r \in R\) because \(R\) is a PID.
Because we picked \(r\) such that \((r)\) is the maximal ideal of \(R\),
\(r\) is an irreducible element of \(R\).
Also, suppose that \(x\) is an arbitrary irreducible element of \(R\).
Because \(x\) is irreducible, \((x)\) is a maximal ideal.
But because there is only one maximal ideal \((r)\) in \(R\),
\((x) = (r)\) should hold.
Thus, \(x\) and \(r\) must associate.
This shows that every irreducible element of \(R\) associate with each other.

Let \(I \subsetneq R\) be an arbitrary non-zero proper ideal of \(R\).
Then, there is \(a \in R\) such that \(I = (a)\) because \(R\) is a PID.
Trivially, \(a\) cannot be a zero (If not, \((a) = \{0\}\) but we assumed \(I\) is a non-zero ideal), and cannot be a unit (If not, \((a) = R\) but we assumed \(I\) is a proper ideal).
Let \(a = u x_1 \cdots x_m\) be a factorization of \(R\)
where \(u \in R\) is a unit
and \(x_1, \cdots, x_m\) are irreducible elements of \(R\).
As \(R\) is a UFD,
the factorization exists and unique up to association
and
because \(a\) is non-zero non-unit, \(m \ge 1\).
As we showed that every irreducible elements of \(R\) associates,
\(x_1, \cdots, x_m\) associate to \(r\).
Let \(u_1, \cdots, u_m\) be units of \(R\) such that \(x_1 = u_1r, \cdots, x_m = u_m r\).
Let \(u' = u u_1 \cdots u_m\).
Then, \(u'\) is a unit because there is \(u^* = u_m^{-1} \cdots u_1^{-1} u^{-1} \in R\) which is the multiplicative inverse of \(u'\).
Since \(R\) is commutative,
\[ a
  = u x_1 \cdots x_m
  = u (u_1 r) \cdots (u_m r)
  = (u u_1 \cdots u_m) r^m
  = u' r^m
\]
Thus, \(a\) associates to \(r^m\),
and \(I = (a) = (r^m)\) holds for such \(m \ge 1\).
\qedsq

\subsubsection*{(\(\Longleftarrow\))}

Suppose that there is \(r \in R\), which satisfies that there exists \(n \in \bbN\) such that \(I = (r^n)\) for any nonzero proper ideal \(I\) of \(R\).
\newline

\noindent
Suppose that \(R\) has no nonzero proper ideal \(I\) of \(R\).
It means, \(R\) has only two ideals: trivial ideal \(\{0\}\) and \(R\).
Thus, \(R\) is a field.
And because the field \(R\) contains only one proper ideals \(\{0\}\), \(\{0\}\) is the unique maximal ideal of \(R\).
\newline

\noindent
Let's assume that \(R\) has nonzero proper ideals.

First, if \(r\) is a unit, \(r^n\) is also a unit for any \(n \in \bbN\).
(\(\because (r^n)^{-1} = (r^{-1})^n\))
But in this case, \((r^n) = R\).
(\(\because \forall x \in R: x = r^n((r^n)^{-1}x) \in r^nR = (r^n)\))
Then, for every \(n \in \bbN\), \(\{0\} \neq (r^n) = R\).
However it's a contradiction because we assumed that \(\exists n \in \bbN: I = (r^n)\) for any proper ideal \(I\) of \(R\). Thus, \(r\) cannot be a unit.

For any \(n \in \bbN\), \((r^n) \subseteq (r)\).
Because if \(a \in (r^n)\),
then there exists \(b \in R\) such that \(a = r^nb = r (r^{n - 1}b) \in rR = (r)\).

Let \(M\) be a maximal ideal of \(R\).
Then, there is \(m \in \bbN\) such that \(M = (r^m)\).
But as we showen above, \(M = (r^m) \subseteq (r) \subsetneq R\).
Because \(M\) is maximal, \(M = (r)\) or \((r) = R\).
Since \(r\) is a non-unit, \((r) \neq R\), and \(M = (r)\) must hold.
Therefore, if \(M\) is a maximal ideal of \(R\), it must be \((r)\).
Therefore, \(R\) has only one maximal ideal, \((r)\).
\qedsq

\end{document}
