\section*{Problem 2}

Give the proof of the below theorem:
\begin{theorem}
  Let \(\phi: G \to \GL(V)\) be a representation for a finite dimensoinal
  vector space \(V\).
  Let \(\{v_1, \cdots, v_n\}\) be a basis of \(V\)
  and let \(\{v_1^*, \cdots, v_n^*\}\) be its dual basis.

  Then \(\Tr \phi(g) = \sum_{i=1}^n v_i^*(g \cdot v_i)\).
\end{theorem}

\subsection*{Proof}

Let \(V\) is a vector space over a field \(F\),
and let \(n = \dim_F V\).

Note, the action of \(G\) on \(V\) is the representation \(\phi\).
i.e. \(g \cdot v_i = \phi(g)(v_i)\).

Let \(\calB = \{v_1, \cdots, v_n\}\).
And let \([A]_\calB\) be a matrix of a linear transformation \(A \in \End(V)\) with respect to \(\calB\), and let \([v]_\calB\) be a column matrix of \(v \in V\) with respect to \(\calB\).

Let \((a_{i,j}) = [\phi(g)]_\calB\).
Note that \([v_i]_\calB = e_i\),
where \(e_i\) is an element of \(V\) such that all entries are zero except
the \(i\)-th entry which is valued by \(1_F\).
In this case,
\[[\phi(g)(v_i)]_\calB = [\phi(g)]_\calB e_i =
\mMat{c}{a_{1,i} \\ \vdots \\ a_{n,i}}\]
Thus,
\[\phi(g)(v_i) = \sum_{j=1}^{n} a_{j, i} v_j\]
Then, by the definition of dual basis,
\begin{align*}
  v_i^*(\phi(g)(v_i))
  &= v_i^*(\sum_{j=1}^{n} a_{j, i} v_j)
  \\&= \sum_{j=1}^{n} a_{j, i} \delta_{i, j}
  = a_{i, i}
\end{align*}
Thus,
\[\sum_{i=1}^{n} v_i^*(\phi(g)(v_i)) = \sum_{i=1}^n a_{i,i} = \Tr [\phi(g)]_\calB = \Tr \phi(g)\]
since trace is not changed by change of basis
(because change of basis give a similar linear transformation,
and two similar linear transformations have same trace).
\qedsq
