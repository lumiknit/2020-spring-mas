\section*{Problem 6}

Note:
\begin{definition}
For two functions \(f_1, f_2: G \to \bbC\),
define the \emph{convolution} to be a function \(f_1 * f_2: G \to \bbC\)
given by,
\[(f_1 * f_2)(g) = \sum_{h \in G} f_1(gh^{-1})f_2(h).\]
\end{definition}

\begin{definition}
Let \(f \in F(G, \bbC)\) and let \(\varphi: G \to \GL(V)\) be a representation.
Then the \emph{Fourier transform of \(f\) at \(\varphi\)} is defined to be
\[\widehat{f}(\varphi) := \sum_{g \in G} f(g) \varphi(g).\]
\end{definition}

Let \(f_1, f_2 \in F(G, \bbC)\).
Then prove that
\[\widehat{f_1 * f_2} = \widehat{f_1} \widehat{f_2}.\]

\subsection*{Proof}

For any representation \(\varphi: G \to \GL(V)\),
\begin{align*}
  \widehat{f_1 * f_2}(\varphi)
  &= \sum_{g \in G} (f_1 * f_2)(g) \varphi(g)
  \\&= \sum_{g \in G} \left(\sum_{h \in G} f_1(gh^{-1}) f_2(h)\right) \varphi(g)
  \\&= \sum_{g \in G} \left(\sum_{h \in G} f_1(gh^{-1}) f_2(h)\right) \varphi(gh^{-1}) \varphi(h)
  \\&= \sum_{g' \in G h^{-1}} \left(\sum_{h \in G} f_1(g') f_2(h)\right) \varphi(g') \varphi(h)
  \\&= \sum_{g' \in G} \left(\sum_{h \in G} f_1(g') f_2(h)\right) \varphi(g') \varphi(h)
  \\&= \sum_{g' \in G} \sum_{h \in G} f_1(g') \varphi(g') f_2(h) \varphi(h)
  \\&= \sum_{g' \in G} f_1(g') \varphi(g') \sum_{h \in G} f_2(h) \varphi(h)
  \\&= \left(\sum_{g' \in G} f_1(g') \varphi(g') \right) \left( \sum_{h \in G} f_2(h) \varphi(h) \right)
  \\&= \widehat{f_1}(\varphi) \widehat{f_2}(\varphi)
\end{align*}
because \(Ga = G\) for every group \(G\) and its element \(a\).
Note that beacuse \(f_k\) maps elements of \(G\) into the \(\bbC\),
the set of scalar values,
it can commute to other scalar functions and linear transformations.
However, \(\varphi(gh^{-1})\) and \(\varphi(h)\) may not commute
each other, because their image is in \(\GL(V)\) which is a set of
some linear transformations.

Therefore, \(\widehat{f_1 * f_2} = \widehat{f_1} \widehat{f_2}\) holds.
\qedsq