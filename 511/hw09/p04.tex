\section*{Problem 4}

Prove below:
\begin{theorem}
  For a representation \(V\) of \(G\),
  let \(\chi\) be its character.
  Then the character for the dual representation \(V^*\)
  is the complex conjugate \(\overline{\chi}\).
\end{theorem}

\subsection*{Proof}

(Since trace is well-defined only for finite dimension matrix,
I'll assume that \(V\) is finite-dimensional.)

Let \(V\) be a \(n\)-dimensional vector space over \(\bbC\).
Let \(\calB = \{v_1, \cdots, v_n\}\) be a basis of \(V\).
Then, the dual space \(V^*\) exists and one of its basis is
\(\calB^* = \{v_1^*, \cdots, v_n^*\}\), the dual basis of \(\calB\).

Let \(\varphi: G \to \GL(V)\) be a representation \(V\) of \(G\)
and \(\chi\) be a character of the representation \(\varphi\).
Then, as the Problem 2,
\[\chi(g) = \Tr \varphi(g) = \sum_{i=1}^{n} v_i^*(\varphi(g)(v_i))\]
Also,
\[\chi^*(g) = \sum_{i=1}^{n} v_i^{**}(\varphi^*(g)(v_i^*))\]
where \(\varphi^*\) be the dual representation of \(\varphi\),
and \(\chi^*\) is the character of \(\varphi^*\).
By the definition of \(\varphi^*\),
\[\varphi^*(g)(f)(v) = (g \cdot f)(v) = f(g^{-1} \cdot v) = f(\varphi(g^{-1})(v))\]
for \(f \in V^*\).

\[\chi^*(g)
= \sum_{i=1}^{n} v_i^{**}(\varphi^*(g)(v_i^*))
= \sum_{i=1}^{n} v_i^{**}(v_i^* \circ \varphi(g^{-1}))\]

Let \((a_{i,j}) = [\varphi(g^{-1})]_\calB\).
Then, \(\varphi(g^{-1})(v_i) = \sum_{k=1}^n a_{k, i} v_k\).
Thus, \(v_i^* \circ \varphi(g^{-1})\) maps \(v_j\) to \(a_{i, j}\).
In this case, we can denote \(v_i^* \circ \varphi(g^{-1})\) as
\[v_i^* \circ \varphi(g^{-1}) = \sum_{j=1}^n a_{i, j} v_j^*\]
Then, by the linearity of dual basis,
\begin{align*}
  \chi^*(g)
  = \sum_{i=1}^{n} v_i^{**}(v_i^* \circ \varphi(g^{-1}))
  &= \sum_{i=1}^{n} v_i^{**}(\sum_{j=1}^n a_{i, j} v_j^*)
  \\&= \sum_{i=1}^{n} \sum_{j=1}^n v_i^{**}( a_{i, j} v_j^*)
  \\&= \sum_{i=1}^{n} \sum_{j=1}^n a_{i, j} \delta_{i, j}
  \\&= \sum_{i=1}^{n} a_{i, i} = \Tr\varphi(g^{-1}) = \chi(g^{-1})
\end{align*}

We proved the below lemma in the lecture (Lemma (B)):
\begin{lemma}
  For a character \(\psi: G \to \bbC\),
  \(\psi(x^{-1}) = \overline{\psi(x)}\) for all \(x \in G\).
\end{lemma}

Thus,
\[\chi^*(g) = \chi(g^{-1}) = \overline{\chi(g)}\]
\qedsq