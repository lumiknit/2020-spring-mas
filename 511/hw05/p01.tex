\section*{Problem 1}

(Horseshoe Lemma)
Suppose
\[0 \rightarrow M' \rightarrow M \rightarrow M'' \rightarrow 0\]
is a short exact sequence of \(R\)-modules.
Suppose there are projective resolutions
\(P'_\bullet \to M'\)
and
\(P''_\bullet \to M''\).

Prove that there is a projective resolution \(P_\bullet \to M\)
such that it fits into a short exact sequence of complexes:
\[0 \rightarrow P'_\bullet \rightarrow P_\bullet \rightarrow P''_\bullet \rightarrow 0\]

\subsection*{Lemmata}

\begin{lemma}\label{lem-prod-free}
  Let \(F_1, F_2\) be free \(R\)-modules. \(F_1 \oplus F_2\) is free.
\end{lemma}
\begin{proof}
  Note:
  \(F_1 \oplus F_2\) is a product and a coproduct of \(F_1\) and \(F_2\).
  Disjoint union is a coproduct in the Sets category \(\bfSets\).

  Let \(F_1\) is a free \(R\)-module on \(X\) and \(F_2\) is a free \(R\)-module on \(Y\).

  Let \(M\) be an arbitrary \(R\)-module and there is an injective map \(e: X_1 \sqcup X_2 \to M\).

  \[\begin{tikzcd}
    & M &
    \\ F_1 \ar[<->]{r} \ar[dashed, "\exists! f_1"]{ur}
    &  F_1 \oplus F_2 \ar[dashed, "f"]{u}
    &  F_2 \ar[<->]{l} \ar[dashed, swap, "\exists! f_2"]{ul}
    \\ X \ar["i_1"]{u} \ar[swap, "\iota_1"]{r} \ar[dashed, bend left=120, "e_1"]{uur}
    &  X \sqcup Y \ar[dashed, "\exists! h"]{u}
    &  Y \ar["i_2"]{u} \ar["\iota_2"]{l} \ar[dashed, bend right=120, swap, "e_2"]{uul}
  \end{tikzcd}\]

  Every element of \(X \sqcup Y\) is in \(X\) or \(Y\).
  Each \(e_1 = \blkWhere{e}_X\) and \(e_2 = \blkWhere{e}_Y\) are injective maps.
  Since \(F_1\) and \(F_2\) are free, there is unique \(R\)-module homomorrphisms \(f_1: F_1 \to M\) and \(f_2: F_2 \to M\)
  such that \(e_1 = f_1 \circ i_1\) and \(e_2 = f_2 \circ i_2\).

  Let \(j_1: F_1 \to F_1 \oplus F_2\) and \(j_2: F_2 \to F_1 \oplus F_2\)
  be injective \(R\)-module homomorphisms (which consist of the coproduct).
  Then, take \(h_1 = j_1 \circ \iota_1\) and \(h_2 = j_2 \circ \iota_2\).
  In this case, there is a unique \(h: X \sqcup Y \to F_1 \oplus F_2\)
  such that \(h \circ \iota_1 = h_1 = j_1 \circ i_1\)
  and \(h \circ \iota_2 = h_2 = j_2 \circ i_2\).
  This \(h\) is injective.
  (\(\because\) Note that the only element of images of \(j_1\) and \(j_2\) is \((0, 0)\), because \(j_1(x) = (x, 0)\) and \(j_2(y) = (0, y)\).
  Suppose that \(h\) is not injective.
  Then, there is \(a, b \in X \sqcup Y\)
  such that \(h(a) = h(b)\).
  If \(a, b \in X\), it's impossible since \(h_1\) is injective.
  If \(a, b \in Y\), it's impossible since \(h_2\) is injective.
  WLOG, let's assume \(a \in X\) and \(B \in Y\).
  Then, \(h_1(x) = (i_1(x), 0) = (0, i_2(y)) = h_2(y)\).
  It means \(i_1(x) = 0 = i_2(y)\).
  However, as we shown in the Homework \#3 Problem 2, the generator of free modules must no contain 0.
  Thus, It's a contradiction.)

  Also, since \(F_1 \oplus F_2\) is a coproduct and there is a function \(f_1: F_1 \to M\) and \(f_2: F_2 \to M\), there is a unique map \(f: F_1 \oplus F_2 \to M\) such that \(f_1 = f \circ j_1\) and \(f_2 = f \circ j_2\). Because \(f_1, f_2\) exists uniquely when \(e\) is determined, \(f\) exists uniquely for \(e\).
  In addition, as the above diagram, \(e = f \circ h\).

  Therefore, for \(X \sqcup Y\), there exists \(h: X \sqcup Y \to F_1 \oplus F_2\),
  and if \(e: X \sqcup Y \to M\) was given,
  there is a unique map \(f: F_1 \oplus F_2 \to M\) such that \(e = f \circ h\).

  Therefore, \(F_1 \oplus F_2\) is free over \(X \sqcup Y\).
\end{proof}

\subsection*{Proof}

\[
\begin{tikzcd}
 & \vdots \ar{d} & \vdots \ar[dashed]{d} & \vdots \ar{d} & \\
0 \ar{r} & P'_1 \ar["\iota_1"]{r} \ar["g_1"]{d} & P_1 = P'_1 \oplus P''_1 \ar["\pi_1"]{r} \ar[dashed, "f_1"]{d} & P''_1 \ar{r} \ar["h_1"]{d} & 0 \\
0 \ar{r} & P'_0 \ar["\iota_0"]{r} \ar["g_0"]{d} & P_0 = P'_0 \oplus P''_0 \ar["\pi_0"]{r} \ar[dashed,"f_0"]{d} & P''_0 \ar{r} \ar["h_0"]{d} & 0 \\
0 \ar{r} & M' \ar["i"]{r} \ar{d} & M \ar["p"]{r} \ar{d} & M'' \ar{r} \ar{d} & 0 \\
& 0 & 0 & 0 &
\end{tikzcd}
\]

Let \(M, M', M'', P'_\bullet, P''_\bullet, g_\bullet, h_\bullet, i, p\) are given as above.
\(P'_\bullet \to M'\), \(P''_\bullet \to M''\) are projective resolutions, \(0 \to M' \to M \to M'' \to 0\) is exact.

Note that \(0 \to P'_k \to P'_k \oplus P''_k \to P''_k \to 0\) is a split exact sequence with natural homomorphisms for every \(k \in \bbZ^{\ge 0}\).
Let \(P_k = P'_k \oplus P''_k\) for each \(k \in \bbZ^{\ge 0}\).

Note that each \(P_k\) are projective.
Because, since \(P'_k\) and \(P''_k\) are projective, there are \(R\)-modules \(Q'\) and \(Q''\) such that \(P'_k \oplus Q'\) and \(P''_k \oplus Q''\) are free.
By Lemma \ref{lem-prod-free},
\[P'_k \oplus Q' \oplus P''_k \oplus Q''
  = (P'_k \oplus P''_k) \oplus (Q' \oplus Q'')
  = P_k \oplus (Q' \oplus Q'')\]
is free.
Therefore, \(P_k\) is projective.

Thus, it's enough to find \(f_k\) which make \(P_\bullet \to M\) as a projective resolution.

First, because \(p\) is surjective and \(P''_0\) is projective,
there is \(h'_0: P''_0 \to M\) such that \(p \circ h'_0 = h_0\).
Also, we have \(g'_0: P'_0 \to M\) such that \(g'_0 = i \circ g_0\).
In this case, we have a homomorphism \(f_0: P_0 \to M\) such that
\[f_0(x, y) = g'_0(x) + h'_0(y)\].
This is surjective, because,
for \(m \in M\), since \(h_0 \circ \pi_0\) is surjective,
there is \(m' \in P_0\) such that \(h_0(\pi_0(m')) = p(m)\)
Then, \(p(m - f_0(m')) = p(m) - p(f_0(m')) = p(m) - h_0(\pi_0(m')) = 0\).
Since \(m - f_0(m') \in \ker p = \im i\),
Then, there is \(a \in M'\) such that \(i(a) = m - f_0(m')\).
Since \(g_0\) is surjective, there is \(a' \in P'_0\) such that \(i(g_0(a')) = m - f_0(m')\).
Let \(n = \iota_0(a') + m'\), then, \(f_0(n) = f_0(\iota_0(a')) + f_0(m') = m\).
\br
\noindent
Then, by Snake Lemma (See Problem 7),
\(\ker g_0 \to \ker f_0 \to \ker h_0\) is exact.
Also, since \(\iota_0\) is injective, natural homomorphism from \(\ker g_0\) to \(\ker f_0\) induced from \(\iota_0\) is also injective.
Also, since \(\coker g_0 = M' / \im g_0 = 0\) as \(\im g_0 = M'\), \(0 \to \ker g_0 \to \ker f_0 \to \ker h_0 \to 0\) is exact.
Also, note that the image of \(g_1\) and \(h_1\) are \(\ker g_0\) and \(\ker h_0\). Then, we can build the below diagram:

\[
\begin{tikzcd}
 & \vdots \ar{d} & & \vdots \ar{d} & \\
0 \ar{r} & P'_1 \ar["\iota_1"]{r} \ar["g_1"]{d} & P_1 = P'_1 \oplus P''_1 \ar["\pi_1"]{r} \ar[dashed, "f_1"]{d} & P''_1 \ar{r} \ar["h_1"]{d} & 0 \\
0 \ar{r} & \ker g_0 \ar["\tilde \iota_0"]{r} \ar{d} & \ker f_0 \ar["\tilde \pi_0"]{r} \ar{d} & \ker h_0 \ar{r} \ar{d} & 0 \\
& 0 & 0 & 0 &
\end{tikzcd}
\]
Each rows are exact, the first and the third columns also exact.

In this case, we can construct \(f_1\) repeating the above steps by considering each \(\ker f_0\), \(\ker g_0\), \(\ker h_0\) as \(M\), \(M'\), \(M''\) and using the projectivity of \(P''_1\).
In the same way we did above, \(f_1\) is also surjective (\(\because\) \(\coker g_1 = \ker f_0 / \im g_1 = 0\) as we consider the codomain of \(g_1\) is restricted to \(\ker f_0\)), \(0 \to \ker g_1 \to \ker f_1 \to \ker h_1 \to 0\) is exact, then, construct \(f_2\) which is surjective, \(\cdots\).

Then, because each \(\im f_k\) is \(\ker f_{k - 1}\) for \(k \in \bbN\), if we consider each \(f_k\) is a homomorphism from \(P_k\) to \(P_{k - 1}\), \(P_\bullet\) be a projective resolution of \(M\).
\qedsq

%-------------------------------------------------------------------------------
\subsection*{Failed Proof 1}

Let the below is a short exact sequence:
\[\begin{tikzcd}
  0 \ar{r}
  & M' \ar["i"]{r}
  & M \ar["p"]{r}
  & M'' \ar{r}
  & 0 \\
  \end{tikzcd}\]
and there are projective resolutions \(P'_\bullet \to M'\) and \(P''_\bullet \to M''\).

Let's take \(P_k = P'_k \oplus P''_k\) for \(k \in \bbZ^{\ge 0}\).

First, note that:
\[\begin{tikzcd}
  0 \ar{r}
  & P'_k \ar["\iota_k"]{r}
  & P_k = P'_k \oplus P''_k \ar["\pi_k"]{r}
  & P''_k \ar{r}
  & 0 \\
  \end{tikzcd}\]
is a split short exact sequence, where \(\iota_k\) is an embedding such tha \(\iota_k: x \mapsto (x, 0)\) and \(\pi_k\) is a projection such that \(\pi_k: (x, y) \mapsto y\).

Therefore, if we show that \(P_0, P_1, \cdots\) make a projective resolution of \(M\), \(M\) has a projective resolution \(P_\bullet\) which makes each row and column of the below diagram exact.
\[
\begin{tikzcd}
 & \vdots \ar{d} & \vdots \ar{d} & \vdots \ar{d} & \\
0 \ar{r} & P'_1 \ar["\iota_1"]{r} \ar["g_1"]{d} & P_1 = P'_1 \oplus P''_1 \ar["\pi_1"]{r} \ar["f_1"]{d} & P''_1 \ar{r} \ar["h_1"]{d} & 0 \\
0 \ar{r} & P'_0 \ar["\iota_0"]{r} \ar["g_0"]{d} & P_0 = P'_0 \oplus P''_0 \ar["\pi_0"]{r} \ar["f_0"]{d} & P''_0 \ar{r} \ar["h_0"]{d} & 0 \\
0 \ar{r} & M' \ar["i"]{r} \ar{d} & M \ar["p"]{r} \ar{d} & M'' \ar{r} \ar{d} & 0 \\
& 0 & 0 & 0 &
\end{tikzcd}
\]

\br
First, \(P_k = P'_k \oplus P''_k\) is projective.
Because, since \(P'_k\) and \(P''_k\) are projective, there are \(R\)-modules \(Q'\) and \(Q''\) such that \(P'_k \oplus Q'\) and \(P''_k \oplus Q''\) are free.
By Lemma \ref{lem-prod-free},
\[P'_k \oplus Q' \oplus P''_k \oplus Q''
  = (P'_k \oplus P''_k) \oplus (Q' \oplus Q'')
  = P_k \oplus (Q' \oplus Q'')\]
is free.
Therefore, \(P_k\) is projective.

%--------------------------
Suppose that, for \(k \in \bbN\),
\(0 \to P'_{k - 1} \to P_{k - 1} \to P''_{k - 1} \to 0\) is exact and split,
\(0 \to P'_{k} \to P_{k} = P'_{k} \oplus P''_{k} \to P''_{k} \to 0\) is exact,
and there is \(R\)-module homomorphisms \(g_k: P'_k \to P'_{k - 1}\) and \(h_k: P''_k \to P''_{k - 1}\). 
\[\begin{tikzcd}
  0 \ar{r} & P'_k \ar["\iota_k"]{r} \ar["g_k"]{d} & P_k = P'_k \oplus P''_k \ar["\pi_k"]{r} \ar[dashed, "\exists f_k"]{d} & P''_k \ar{r} \ar["h_k"]{d} & 0 \\
  0 \ar{r} & P'_{k - 1} \ar["\iota_{k - 1}"]{r} \ar["g_{k - 1}"]{d} & P_{k - 1} \simeq P'_{k - 1} \oplus P''_{k - 1} \ar["\pi_{k - 1}"]{r} \ar[dashed, "\exists f_{k - 1}"]{d} & P''_{k - 1} \ar{r} \ar["h_{k - 1}"]{d} & 0 \\
  0 \ar{r} & P'_{k - 2} \ar["\iota_{k - 2}"]{r} & P_{k - 2} \simeq P'_{k - 2} \oplus P''_{k - 2} \ar["\pi_{k - 2}"]{r} & P''_{k - 2} \ar{r} & 0
\end{tikzcd}\]
Because the 2nd row is split, there is \(\varphi: P''_{k - 1} \to P'_{k - 1} \oplus P''_{k - 1}\) such that \(\pi_{k - 1} \circ \varphi = \Id_{P''_{k - 1}}\).
Then, let \(f_k: P_k \to P_{k - 1}\) be a \(R\)-module homomorphism such that
\[f_k(x, y) = \iota_{k - 1}(g_k(x)) + \varphi(h_k(y))\]
Note that \(f_k\) is a \(R\)-module homomorphism, because each \(\iota_{k - 1}, \varphi, g_k, h_k\) are \(R\)-module homomorphisms.

In this way, we can construct \(R\)-module homomorphisms \(f_1: P_1 \to P_0\), \(\cdots\).

Let's show \(P_k \xrightarrow{f_k} P_{k - 1} \xrightarrow{f_{k - 1}} P_{k - 2}\) is exact for \(k \in \bbZ^{\ge 2}\).

Let \((x, y) \in \im f_k \subseteq P_{k - 1}\).
Then, there is \((x', y') = f_k(x, y)\) for some \(x' \in P'_k\) and \(y' \in P''_k\).
Then, \(x = g_k(x')\) and \(y = h_k(y')\).
Thus, \(x \in \im g_k = \ker g_{k - 1}\) and \(y \in \im h_k = \ker h_{k - 1}\).
Then, \(f_{k - 1}(x, y) = (g_{k - 1}(x), h_{k - 1}(y)) = (0, 0)\).
Therefore, since \(0 \to P'_{k - 2} \to P_{k - 2} \to P''_{k - 2} \to 0\) is split, there is \(\varphi: P''_{k - 2} \to P_{k - 2}\) such that \(\pi_{k - 2} \circ \varphi = \Id_{P''_{k - 2}}\).
\begin{align*}
  f_{k - 1}(n_1, n_2)
  &= \iota_{k - 2}(g_{k - 1}(n_1)) + \varphi(h_{k - 1}(n_2))
  = \iota_{k - 2}(0) + \varphi(0) = 0
\end{align*}
Thus, \((n_1, n_2) \in \ker f_{k - 1}\) and \(\im f_k \subseteq \ker f_{k - 1}\).

Let \((x, y) \in \ker f_{k - 1} \subseteq P_{k - 1}\).
Since \(f_{k - 1}(x, y) = 0\), its isomorphic image in \(P'_{k - 2} \oplus P''_{k - 2}\) is \((0, 0)\) and each isomorphic images of \(g_{k - 1}(x)\) and \(h_{k - 1}(y)\) are zero.
Thus, \(x \in \ker g_{k - 1} = \im g_k\) and \(y \in \ker h_{k - 1} = \im h_k\).
Then, there is \(x' \in P'_k\) and \(y' \in P''_k\) such that \(g_k(x') = x\) and \(h_k(y') = y\).
Then, \(f_k(x', y') = (x, y)\) and \((x, y) \in \im f_k\).
Therefore, \(\ker f_{k - 1} \subseteq \im f_k\).

This shows \(\cdots \to P_2 \to P_1 \to P_0\) is exact.
\br
%------------------------

Let \(p_1: P_0 \to P'_0\) and \(p_2: P''_0 \to P_0\) such that
\(p_1 \circ \iota_0 = \Id_{P'_0}\) and \(\pi_0 \circ p_2 = \Id_{P''_0}\).
Then, let \(\alpha_1: P_0 \to M\) such that \(\alpha_1 = i \circ g_0 \circ p_1\).
Since \(h_0\) is surjective (because of exact sequence),
and \(p: M \to M''\) is surjective because of an exact sequence
and \(P_0\) is projective,
there is \(\gamma: P''_0 \to M\) such that \(p \circ \gamma = h_0\).
Let \(f_0(x, y) = i(g_0(x)) + \gamma(y)\).
Since \(f_0\) is a sum of compositions of homomorphisms,
\(f_0\) is a homomorphism.

\(f_0\) is surjective.
Let \(m \in M\).
Since \(h_0 \circ \pi_0\) is surjective, there is \(m' \in P_0\) such that \(h_0(\pi_0(m')) = p(m)\).
Then, \(p(m - f_0(m')) = p(m) - p(f_0(m')) = p(m) - h_0(\pi_0(m')) = 0\).
Since \(m - f_0(m') \in \ker p = \im i\),
Then, there is \(a \in M'\) such that \(i(a) = m - f_0(m')\).
Since \(g_0\) is surjective, there is \(a' \in P'_0\) such that \(i(g_0(a')) = m - f_0(m')\).
Let \(n = \iota_0(a') + m'\).
Then,
\begin{align*}
  f_0(n)
  &= f_0(\iota_0(a')) + f_0(m')
  \\&= m - f_0(m') + f_0(m') = m
\end{align*}
Thus, \(f_0\) is surjective.

\(\im f_1 \supseteq \ker f_0\).
Let \((x, y) \in \ker f_0\).
Then, \(f_0(x, y) = 0\).
\(h_0(y) = p(f_0(x, y)) = p(0) = 0\).
Then, \(y \in \ker h_0 = \im h_1\).
And \(\gamma(y) \in \ker p = \im i\).
Then, \(i(g_0(x)) = f_0(x, 0) = f_0(x, y) - \gamma(y) = \gamma(-y)\).
\(g_0(x) = i^{-1}\gamma(-y)\).

Since \(i\) is injective, \(g_0(x) = 0\).
Thus, \(x \in \ker g_0 = \im g_1\).
Therefore, there is \(x' \in P'_1\) and \(y' \in P''_1\)
such that \(g_0(x') = x\) and \(h_0(y') = y\).
And, \(f_1(x', y') = (x, y)\).
Thus, \((x, y) \in \im f_1\)
and \(\ker f_0 \subseteq \im f_1\).

\(\im f_1 \subseteq \ker f_0\).
Let \((x, y) \in \im f_1\).
Then, there is \((x', y') \in P_1\), such that \(f_1(x', y') = (x, y)\).
Then, \(g_1(x') = x\) and \(h_1(y') = y\).
Since \(x \in \im g_1 = \ker g_0\) and \(y \in \im h_1 = \ker h_0\), 
\(g_0(x) = 0 = h_0(y)\).
Then, \(i(g_0(x)) = 0\).
And, \(\gamma(y) \in \ker p = \im i\).
Let \(z \in P'_0\) such that
\(\gamma(y) = i(g_0(z)) = f(z, 0)\).
Then, \(f(-z, y) = -i(g_0(z)) + \gamma(y) = 0\).
It means \((-z, y) \in \ker f_0\).
As we shown \(\ker f_0 \subseteq \im f_1\) above, \((-z, y) \in \im f_1\).
Then, there is \(z' \in P'_1\) such that \(g_1(z') = z\).
It means, \(z \in \im g_1 = \ker g_0\).
Therefore, \(\gamma(y) = i(g_0(z)) = i(0) = 0\).
Therefore, \(f(x, y) = 0\).
It shows \((x, y) \in \ker f\) and \(\im f_1 \subseteq \ker f_0\).

\br
\noindent
Therefore, \(P_\bullet\) is a projective resolution of \(M\).
\qedsq