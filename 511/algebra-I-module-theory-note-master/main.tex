% lumiknit's flavored latex presettings
% 20200526
% lumiknit

% === Main file ===

% Document descriptions
\title{Note for Algebra I}
\author{lumi}
\date{} % Commenting out this if you want to get a date

% Document style
\documentclass{article}
\usepackage[a4paper, total={6in, 9in}]{geometry}
\usepackage[nodisplayskipstretch]{setspace}
\setstretch{1.15}

% Load packages and settings
\input{packages}
\input{settings}

% --- Document ---
\begin{document}
\maketitle
\vspace{1pt}
% --- Put all contents below, using \input{<FILE>} or \include{<FILE>}

%-------------------------------------------------------------------------------
\section*{General Term}

\begin{definition}
  Let \(A\) be some algebraic structure.

  \(A\) is \emph{simple}
  if there is no proper non-trivial normal substructure.
\end{definition}

%-------------------------------------------------------------------------------
\section*{Module Theory}

\begin{lemma}
  If \(R\) is commutative, \(\Hom_R(M, N)\) is an \(R\)-module.
\end{lemma}

\begin{definition}
  \(\End_R(M) = \Hom_R(M, M)\) is an \emph{endomorphism ring}.
\end{definition}

\begin{definition}
  Annihilator
  \(\ann(m) = \{r \in R \mid rm = 0\}\)
\end{definition}

\begin{lemma}
  \(R/\ann(m) \simeq Rm\)
\end{lemma}

\begin{theorem}
  There is an abelian group denoted by \(M \otimes_R N\)
  with an\(R\)-balanced map \(\iota: M \times N \to M \otimes_R N\)
  with the following universal property:
  for any \(R\)-balanced map \(\varphi: M \times N \to L\)
  for some \(L \in (\Ab)\),
  there is a unique group homomorphism \(\tilde\varphi: M \otimes_R N \to L\)
  such that
  \[\begin{tikzcd}
    M \times N \ar[swap,"\iota"]{rd} \ar["\varphi"]{rr} & & L \\
    & M \otimes_R N \ar[swap, "\tilde\varphi"]{ru}
  \end{tikzcd}\]
  commutes
\end{theorem}

\begin{theorem}
  \(M \otimes_R N \simeq M \times N / Q\)
  such that \(Q\) is generated by
  \begin{itemize}
  \item \((m_1 + m_2, n) - (m_1, n) - (m_2, n)\)
  \item \((m, n_1 + n_2) - (m, n_1) - (m, n_2)\)
  \item \((mr, n) - (m, rn)\)
  \end{itemize}
\end{theorem}

\begin{example}
  \(R/I \otimes_R R/J = R/(I + j)\)
\end{example}
\begin{example}
  \(R[t_1, \cdots, t_r] = R \otimes_k k[t_1, \cdots, t_r]\)
\end{example}
\begin{example}
  \(\bbQ/\bbZ \otimes_\bbZ \bbQ/\bbZ = 0\)
\end{example}

\begin{theorem}
  For a commutative rings with unity, the tensor product is a push-out such that:
  \[\begin{tikzcd}
    R \ar{d} \ar{r} & R_1 \ar{d} \\
    R_2 \ar{r} & R_1 \otimes_R R_2
  \end{tikzcd}\]
\end{theorem}

\begin{theorem}
  For \(\phi: M \to M'\) and \(\psi: N \to N'\),
  there is a unique homomorphism (of groups)
  \[\phi \otimes \psi: M \otimes_R N \to M' \otimes_R N'\]
  such that
  \((\phi \otimes \psi)(m \otimes n) = \phi(m) \otimes \psi(n)\).
\end{theorem}

\begin{theorem}
  There is a natural isomorphism:
  \[(M \otimes_R N) \otimes_T L \simeq M \otimes_R (N \otimes_T L)\]
\end{theorem}

\begin{theorem}
  \[(\bigoplus_i M_i) \otimes_R N \simeq \bigoplus_i(M_i \otimes_R N)\]
\end{theorem}

\begin{theorem}
  Given an \(R\)-algebra \(R \to S\),
  \[S \otimes_R R^{\otimes n} \simeq S^{\oplus n}\]
\end{theorem}

\begin{theorem}
  \[R^m \otimes_R R^n = R^{mn}\]
\end{theorem}

\begin{definition}
  Let \(R\) be a ring with unity,
  and \(M\) be a non-zero \(R\)-module.
  \begin{enumerate}
  \item \(M\) is irreducible (or simple),
    if there are no proper non-trivial submodule of \(M\).
    Otherwise, \(M\) is reducible.
  \item \(M\) is indecomposable
    if \(M\) is not of the form \(M_1 \oplus M_2\) for non-zero submodules
    \(M_1, M_2 \subseteq M\).
    Otherwise, \(M\) is called decomposable.
  \item \(M\) is completely reducible
    if \(M\) is a direct sum of irreducible submodules.
  \item Each direct summand of irreducible decomposition of \(M\)
    is called a constituent of \(M\).
  \end{enumerate}
\end{definition}

\begin{theorem}
  If \(M\) is irreducible, it's indecomposable and completely reducible.
\end{theorem}

%-------------------------------------------------------------------------------
\section*{Homology}

\begin{definition}
  \(A \xrightarrow{\varphi} B \xrightarrow{\psi} C\)
  of \(R\)-modules is called exact at \(B\) if \(\ker \psi = \im \varphi\).
\end{definition}

\begin{definition}
  \(A_{i+1} \xrightarrow{\varphi_{i+1}} A_{i} \xrightarrow{\varphi_{i}} A_{i-1}\)
  is called a complex of \(R\)-modules, if \(\varphi_i \circ \varphi_{i+1} = 0\).
  The \(i\)-th homology is
  \(H_i(A_\bullet) = \ker \varphi_i / \im \varphi_{i+1}\).
  \(A_\bullet\) is exact if all homology is zero.
\end{definition}

\begin{theorem}
  \(0 \to A \xrightarrow{\varphi} B \xrightarrow{\psi} C \to 0\)
  is
  (1) exact at \(A\) iff \(\varphi\) is injective,
  (2) exact at \(B\) iff \(\psi\) is surjective.
\end{theorem}

\begin{definition}
  For \(C \simeq A \oplus B\),
  \[0 \to A \to C \to B \to 0\]
  is a short exact sequence.
\end{definition}

\begin{theorem}
  For \(0 \to A \xrightarrow{\varphi} B \xrightarrow{\psi} C \to 0\),
  TFAE:
  (1) it's split;
  (2) there is a homomorphism \(s: C \to B\) such that \(\psi \circ s = \Id_C\);
  (3) there is a homomorphism \(p: B \to A\) such that \(p \circ \varphi = \Id_A\);
\end{theorem}

\begin{theorem}(Short Five Lemma)
  For a commutative diagram with exact rows:
  \[\begin{tikzcd}
    0 \ar{r} & A \ar["f"]{d} \ar{r} & B \ar["g"]{d} \ar{r} & C \ar["h"]{d} \ar{r} & 0 \\
    0 \ar{r} & A \ar{r} & B \ar{r} & C \ar{r} & 0 \\
  \end{tikzcd}\]
  If \(f\) and \(h\) are injective/surjective/isomorphisms,
  then so is \(g\).
\end{theorem}

\begin{theorem}(Horseshoe Lemma for Projective Resolution)
  For given s.e.s \(0 \to M' \to M \to M''\),
  projective resolutions \(P'_\bullet \to M'\) and \(P''_\bullet \to M''\),
  there is a projective resolution \(P_\bullet \to M\) and a double complex
  \[0 \to P'_\bullet \to P_\bullet \to P''_\bullet \to 0\]
  such that each rows and columns are exact.
\end{theorem}

\begin{theorem}(Horseshoe Lemma for InjectiveResolution)
  For given s.e.s \(0 \to M' \to M \to M''\),
  projective resolutions \(M' \to I'^\bullet\) and \(M'' \to I''^\bullet\),
  there is a projective resolution \(M \to I^\bullet\) and a double complex
  \[0 \to I'^\bullet \to I^\bullet \to I''^\bullet \to 0\]
  such that each rows and columns are exact.
\end{theorem}

\begin{theorem}(Snake Lemma)
  Suppose we have a commutative diagram with exact rows:
  \[\begin{tikzcd}
    & A_1 \ar["f"]{d} \ar["\alpha"]{r} & A_2 \ar["g"]{d} \ar["\beta"]{r} & A_3 \ar["h"]{d} \ar{r} & 0 \\
    0 \ar{r} & B_1 \ar["\gamma"]{r} & B_2 \ar["\delta"]{r} & B_3
  \end{tikzcd}\]
  Then, there is a natural homomorphism \(\partial\)
  and an exact sequence
  \[\ker f \to \ker g \to \ker h \xrightarrow{\partial} \coker f \to \coker g \to \coker h\]

  If \(\alpha\) is injective, so is \(\ker f \to \ker g\).
  If \(\delta\) is injective, so is \(\coker g \to \coker h\).
\end{theorem}

\begin{theorem}
  Let \(0 \to A_\bullet \to B_\bullet \to C_\bullet \to 0\)
  be a s.e.s of complexes of \(R\)-modules.
  Then there is a long exact sequence
  \[\begin{tikzcd}
    \cdots \ar{r} & H_n(A) \ar{r} & H_n(B) \ar{r} & H_n(C) \ar[swap, "\partial"]{lld} \\
    & H_{n-1}(A) \ar{r} & H_{n-1}(B) \ar{r} & H_{n-1}(C) \ar{r} & \cdots \\
  \end{tikzcd}\]
\end{theorem}

\begin{theorem}
  \(\Hom_R(D, -)\) and \(\Hom_R(-, D)\) are left exact.
\end{theorem}
\begin{example}
  \(\Hom_R(D, -)\) is not exact, beacuse of
  \(0 \to \bbZ \xrightarrow{\times n} \bbZ \to \bbZ/n \to 0\).
\end{example}

\begin{theorem}
  For a ring \(R\) and an \(R\)-module \(P\), TFAE,
  \begin{itemize}
  \item \(\Hom_R(P, -)\) is exact,
  \item \(P\) is a projective \(R\)-module,
  \item For every s.e.s \(0 \to L \to M \to P \to 0\), this splits,
  \item \(P\) is a direct summand of a free \(R\)-module.
  \end{itemize}
\end{theorem}

\begin{definition}
  For projective \(P_\bullet\),
  exact \(P_\bullet \xrightarrow{\epsilon} M\) is a projective resolution of \(M\).
\end{definition}

\begin{theorem}
  For a ring \(R\) and an \(R\)-module \(Q\), TFAE,
  \begin{itemize}
  \item \(\Hom_R(-, Q)\) is exact,
  \item \(Q\) is an injective \(R\)-module,
  \item For any s.e.s \(0 \to Q \to M \to N \to 0\) is split.
  \end{itemize}
\end{theorem}

\begin{theorem}(Theorem (B))
  Let \(Q\) be an \(R\)-module.
  Then there is an injective homomorphism \(Q \rightarrowtail I\)
  such that \(I\) is an injective \(R\)-module.
\end{theorem}

\begin{theorem}(Theorem (C))
  \(\prod_{i} Q_i\) is injective iff each \(Q_i\) is injective.
\end{theorem}

\begin{theorem}(Baer Criterion)
  \(Q\) is injective, iff, for all each ideal \(I \subseteq R\)
  and each \(R\)-module homomorphism \(G: I \to Q\),
  it extends to \(\tilde g: R \to Q\).
\end{theorem}

\begin{corollary}
  Let \(R\) be a PID.
  Then \(Q\) is injective
  iff
  for each \(r \in R \bs \{0\}\),
  we have \(rQ = Q\).
\end{corollary}

\begin{lemma}
  Let \(J\) be an abelian group
  and \(R\) be a ring with unity.
  Then, \(\Hom_\bbZ(R, J)\) is a left \(R\)-module.
\end{lemma}

\begin{theorem}(Theorem (D))
  Let \(J\) be a divisible abelian group,
  and \(R\) be a ring with unity.
  Then, \(\Hom_\bbZ(R, J)\) is an injective \(R\)-module.
\end{theorem}

\begin{theorem}
  A tensor functor, \(D \otimes_R -\), is right exact.
  In the same sense, \(- \otimes_R D\) is right exact.
\end{theorem}
\begin{example}
  A tensor product is not exact, because for \(D = \bbZ / n\),
  and
  \(0 \to \bbZ \xrightarrow{\times n} \bbZ \to \bbZ/n \to 0\).
\end{example}

\begin{definition}
  \(D\) is \emph{flat} right \(R\)-module,
  if a tensor functor \(D \otimes_R -\) is exact.

  \(D\) is \emph{flat} left \(R\)-module,
  if a tensor functor \(- \otimes_R D\) is exact.
\end{definition}

\begin{theorem}(Adjunction)
  Let \(R, S\) be rings with unity,
  \(A\) be a right \(R\)-module,
  \(B\) be \((R, S)\)-bimodule,
  \(C\) be a right \(S\)-module.
  Then, there is a natural isomorphism of abelian groups:
  \[\Hom_S(A \otimes_R B, C) \to \Hom_R(A, \Hom_S(B, C))\]
  where \(A \otimes_R B\) is seen as a right \(S\)-module and \(\Hom_S(B, C)\)
  which is a right \(R\)-module.
\end{theorem}
\begin{note}
  \(\varphi: A \otimes_R B \to C\) induces
  \(\psi: A \to \Hom_S(B, C)\) such that
  \(\psi(a) = \psi_a\)
  and
  \(\psi_a = \varphi(a \otimes -)\).
\end{note}
\begin{note}
  \(\psi: A \to \Hom_S(B, C)\) defines
  \(\varphi \in \Hom_S(A \otimes_R B, C)\)
  given by
  \(\varphi(a \otimes b) = \psi(a)(b)\).
\end{note}

\begin{theorem}
  For a commutative ring with unity \(R\), in \(R-\Mod\),
  \[\text{Free} \Rightarrow \text{Projective} \Rightarrow \text{Flat} \Rightarrow \text{Torsion-Free}\]
\end{theorem}
\begin{example}
  \(0\) is free. Thus, it's projective, flat, and torsion-free.
\end{example}

\begin{theorem}
  For a commutative ring \(R\) with unity,
  \(P, Q\) are projective \(R\)-modules,
  then \(P \otimes_R Q\) is also projective.
\end{theorem}

\begin{theorem}
  \(R\)-modules are enough projective and enough injective.
\end{theorem}

\begin{theorem}
  For complexes of \(R\)-modules \(A_\bullet\)  and \(B_\bullet\),
  \(f_\bullet: A_\bullet \to B_\bullet\) is a chain map
  such that the following diagram commutes:
  \[\begin{tikzcd}
  \cdots \ar{r} & A_{n + 1} \ar["f_{n+1}"]{d} \ar{r} & A_{n} \ar["f_{n}"]{d} \ar{r} & A_{n - 1} \ar["f_{n-1}"]{d} \ar{r} & \cdots \\
  \cdots \ar{r} & B_{n + 1} \ar{r} & B_{n} \ar{r} & B_{n - 1} \ar{r} & \cdots
  \end{tikzcd}\]
\end{theorem}

\begin{theorem}
  \(f_n\) induces an induced homomorphism \(f_n: H_n(A_\bullet) \to H_n(B_\bullet)\).
\end{theorem}

\begin{definition}
  \(f_\bullet\) is a quasi isomorphism
  if induced \(f_\bullet\) into homologies is isomorphisms.
\end{definition}

\begin{definition}
  For two complexes \(A_\bullet\) and \(B_\bullet\),
  they are \emph{quasi-isomorphic},
  if there is a zig-zag of complexes and quasi-isomorphism
  \(A_\bullet \leftarrow C^0 \rightarrow C^1 \leftarrow C^2 \rightarrow \cdots \rightarrow B_\bullet\).
\end{definition}

\begin{definition}
  \(\Kom(R)\) is the category of all complexes of \(R\)-modules,
  where objects are complexes and morphisms are chain maps.
  \(\Kom^+(R)\) is bounded below,
  \(\Kom^-(R)\) is bounded below,
  and \(\Kom^b(R)\) is bounded below.
\end{definition}

\begin{definition}
  Let \(R\) be a ring with unity,
  \(f_\bullet, g_\bullet: A_\bullet \to B_\bullet\)
  be two chain maps.
  If there are \(R\)-module homomorphisms
  \(s = \{s_n \mid A_n \to B_{n+1}\}\)
  such that
  \(f - g = sd + ds\).

  If there is a (chain) homotopy between \(f\)and \(g\),
  they are (chain) homotopic. \(f \sim^s g\).
\end{definition}

\begin{theorem}
  If \(f, g\) are chain homotopic,
  then, \(f_n, g_n\) induced into homologies are equal.
\end{theorem}

\begin{definition}
  If chain map is homotopic to the zero map, \(f\) is null homotopic.
\end{definition}

\begin{definition}
  \(f\) is a chain homotopy equivalence
  if there is a chain map \(g\) such that \(g \circ f\) and \(f \circ g\) 
  are identity maps.
\end{definition}

\begin{theorem}
  If \(f\) is a chain homotomy equivalence, it's a quasi-isomorphism.
\end{theorem}

\begin{theorem}
  If \(f_1, f_2: A_\bullet \to B_\bullet\) and \(g_1, g_2: B_\bullet \to C_\bullet\) are chain homotopic chain maps,
  \(g_1 \circ f_1\) and \(g_2 \circ f_2\) are chain homotopic.
\end{theorem}

\begin{definition}
  \(K(R)\) is the category such that:
  \(\Ob(K(R))\) is the complexes of \(R\)-modules,
  \(\Hom(K(R))\) is \(\Hom_{K(R)}(A_\bullet, B_\bullet) = \Hom_{\Kom(R)}(A_\bullet, B_\bullet) / \sim\),
  where \(\sim\) is the chain homotopy.
\end{definition}

\begin{definition}
  \(D(R)\) is the \emph{derived category} such that:
  \(\Ob(K(R))\) is the complexes of \(R\)-modules,
  \(\Hom(K(R))\) is \(\Hom_{D(R)}(A_\bullet, B_\bullet) = \Hom_{\Kom(R)}(A_\bullet, B_\bullet) / \sim_{q.iso}\),
  where \(\sim\) is the chain homotopy.
\end{definition}

\begin{theorem}(Pseudo-universal Property)
  Let \(R\) be a ring with unity,
  \(M, N\) be \(R\)-modules.
  Let \(P_\bullet \to M\) and \(Q_\bullet \to N\) be projective resolutions,
  and let \(f_{-1}: M \to N\) be any \(R\)-module homomorphism.
  Then, there is a chain map \(f_\bullet: P_\bullet \to Q_\bullet\)
  which lifts \(f_{-1}\).
  \(f_\bullet\) is unique up to chain homotopy.
\end{theorem}

\begin{theorem}(Pseudo-universal Property)
  Two projective resolutions of \(R\)-module \(M\)
  are unique un to chain homotopy equivalence.
\end{theorem}

\begin{definition}
  Projective resolution of \(M_\bullet\) is  a projective chain complex \(P_\bullet\)
  if there is a quasi-isomorphism between \(M_\bullet\) and \(P_\bullet\).
\end{definition}

\begin{theorem}(HW5-Problem 2, 5)
  For bounded above complex \(M_\bullet\), there is a projective resolution of \(M_\bullet\).

  For bounded below complex \(M^\bullet\), there is an injective resolution of \(M^\bullet\).
\end{theorem}

\begin{definition}
  Let \(P_\bullet \to M\) be a projective resolution.
  \(\Tor^R_i(N, M) := H_i(N \otimes_R P_\bullet)\).

  Let \(N \to I^\bullet\) be a projective resolution.
  \(\tor^R_i(N, M) := H^i(I^\bullet \otimes_R M)\).
\end{definition}

\begin{lemma}
  \(\Tor^R_i(N, M)\)
  is independent of the choice of a projective resolution
  \(P_\bullet \to M\) up to isomorphism.
  \(\tor^R_i(N, M)\)
  is independent of the choice of a injective resolution
  \(N \to I^\bullet\) up to isomorphism.
\end{lemma}

\begin{theorem}
  If \(N, M\) are projective, \(\Tor^R_i\) are 0 for \(i \ge 1\).
  In the same way, if \(N, M\) are injective, \(\tor^R_i\) are 0 for \(i \ge 1\).
\end{theorem}

\begin{theorem}
  \(\Tor^R_i(N, M) \simeq \tor^R_i(N, M)\)
\end{theorem}

\begin{definition}
  Let \(A_\bullet\) and \(B_\bullet\) be complexes.
  \(\Tor(A \otimes B) = T_\bullet = \{T_n\}\)
  is a total complex of \(A\) and \(B\)
  where \(T_n = \bigoplus_{i+j=n} A_i \otimes B_j\).
\end{definition}

\begin{theorem}(HW5-P6)
  For a projective resolution \(Q_\bullet \to N\)
  and \(P_\bullet \to M\),
  there is a natural quasi-isomorphism
  \(\Tot(P \otimes Q)_\bullet \to Q_\bullet \otimes_R M\).
\end{theorem}

\begin{theorem}
  Let \(0 \to N_1 \to N_2 \to N_3 \to 0\) be a s.e.s of \(R\)-modules,
  and \(M \in \Ob(R-\Mod)\).
  Then, there is a long exact sequence
  \[\begin{tikzcd}
    \cdots \ar{r} & \tor_n(N_1, M) \ar{r} & \tor_n(N_2, M) \ar{r} & \tor_n(N_3, M) \ar[swap, "\partial"]{lld} \\
    & \tor_{n-1}(N_1, M) \ar{r} & \tor_{n-1}(N_2, M) \ar{r} & \tor_{n-1}(N_3, M) \ar{r} & \cdots
  \end{tikzcd}\]
\end{theorem}

\begin{definition}
  Let \(R\) be a ring with unity, \(M, N\) be left \(R\)-modules.
  Let \(N \to I^\bullet\) be an injective resolution.
  \(\Ext_R^n(M, N) = H^n(\Hom_R(M, I^\bullet))\).

  Let \(I_\bullet \to M\) be a projective resolution.
  \(\ext_R^n(M, N) = H^n(\Hom_R(P_\bullet, N))\).
\end{definition}

\begin{theorem}
  \(\Ext_R^n\) is independent of the choice of an injective resolution
  up to isomorphism.
  \(\ext_R^n\) is independent of the choice of a projective resolution
  up to isomorphism.
\end{theorem}

\begin{theorem}
  \(\Ext_R^0(M, N) \simeq \Hom_R(M, N) \simeq \ext_R^0(M, N)\).
\end{theorem}

\begin{theorem}
  If \(N\) is injective and \(n > 0\),
  then \(\Ext_R^n(M, N) = 0\).

  If \(M\) is projective and \(n > 0\),
  then \(\ext_R^n(M, N) = 0\).
\end{theorem}

\begin{theorem}
  Let \(0 \to N_1 \to N_2 \to N_3 \to 0\) be a s.e.s of \(R\)-modules,
  and \(M \in \Ob(R-\Mod)\).
  Then, there is a long exact sequence
  \[\begin{tikzcd}
    \cdots \ar{r} & \Ext^n(M, N_1) \ar{r} & \Ext^n(M, N_2) \ar{r} & \Ext^n(M, N_3) \ar[swap, "\partial"]{lld} \\
    & \Ext^{n+1}(M, N_1) \ar{r} & \Ext^{n+1}(M, N_2) \ar{r} & \Ext^{n+1}(M, N_3) \ar{r} & \cdots
  \end{tikzcd}\]
\end{theorem}

\begin{theorem}
  Let \(0 \to M_1 \to M_2 \to M_3 \to 0\) be a s.e.s of \(R\)-modules,
  and \(M \in \Ob(R-\Mod)\).
  Then, there is a long exact sequence
  \[\begin{tikzcd}
    \cdots \ar{r} & \ext^n(M_3, N) \ar{r} & \ext^n(M_2, N) \ar{r} & \ext^n(M_1, N) \ar[swap, "\partial"]{lld} \\
    & \ext^{n+1}(M_3, N) \ar{r} & \ext^{n+1}(M_2, N) \ar{r} & \ext^{n+1}(M_1, N) \ar{r} & \cdots
  \end{tikzcd}\]
\end{theorem}

\begin{theorem}
  \(\Ext_R^n(M, N) \simeq \ext_R^n(M, N)\).
\end{theorem}

\begin{definition}
  An \emph{extension} of \(M\) by \(N\) is a s.e.s of \(R\)-modules
  \[0 \to N \to T \to M \to 0\].
  If the s.e.s splits, it's \emph{the trivial extension}.
\end{definition}

\begin{definition}
  If \(T_1, T_2\) are two extensions of \(M\) by \(N\)
  and there is a homomorhpism \(T_1 \to T_2\),
  it's an isomorphism by the Short Five Lemma,
  and \(T_1\) and \(T_2\) are said to be \emph{equivalent}.
\end{definition}

\begin{definition}
  \(\Ext_R(M, N)\) be the set of equivalence classes of extensions of \(M\) by \(N\).
\end{definition}

\begin{lemma}
  Let \(e := [0 \to N \to T \to M \to 0] \in \Ext_R(M, N)\).
  Then there is a well-defined class \(\delta(e) \in \Ext_R^1(M, N)\).
\end{lemma}

\begin{lemma}
  Let \(e = [0 \to N \to T \to M \to 0] \in \Ext_R(M, N)\).
  \(e\) is a split exact sequence iff \(\delta(e) = 0\).
\end{lemma}

\begin{theorem}
  The map \(\delta: \Ext_R(M, N) \to \Ext_R^1(M, N)\) is bijective.
\end{theorem}

\begin{definition}
  Let \(e_i: 0 \to N \to T_i \to M \to 0\) for \(i = 1, 2\).
  Consider the pull-back \(T'\) of \(T_1 \to M \leftarrow T_2\),
  i.e. \(T' \subseteq T_1 \times T_2\)
  consisting of \((t_1, t_2)\) whose images in \(M\) coincide.
  Let \(D \subseteq T'\)
  be generated by \((-n, n)\) for \(n \in N\),
  and let \(T := T'/D\).
  This gies a s.e.s \(e: 0 \to N \to T \to M \to 0\),
  which is called \emph{the Baer sum of \(e_1\) and \(e_2\)}.
\end{definition}

\begin{theorem}
  \(\Ext_R(M, N)\) and the Baer Sum give a group structure.
  Furthermore, \(\delta: \Ext_R(M, N) \to \Ext_R^1(M, N)\)
  is a group homomorphism.
\end{theorem}

\begin{definition}
  An \emph{\(n\)-extension of \(M\) by \(N\)} is an exact sequence of the form
  \[0 \to N \to T_n \to \cdots \to T_1 \to M\]
\end{definition}

\begin{definition}
  If there is chain map \(f_\bullet\) between extension \(T_\bullet\), \(T'_\bullet\) of \(M\) by \(N\),
  it's an equivalence.
  Also, the class of this equivalence give \(\Ext_R^{(n)}(M, N)\), which is
  called \emph{the Yoneda \(n\)-Extension}
\end{definition}

\begin{theorem}
  The Yoneda \(n\)-Extension with a higher Baer sum is isomorphic to \(\Ext_R^n\).
\end{theorem}

\begin{definition}
  Let \(P_\bullet \to M\) be a projective resolution.

  Then, if there is some \(N \ge 0\) such that \(P_n = 0\) for every \(n > N\),
  \emph{the length of the projective resolution is less or equal to \(N\)}.

  If no such \(N\), the length is infinite.

  The projective dimension \(\pd_R M\) is the smallest length of such projective resolutions.
\end{definition}

\begin{theorem}(HW8-P1)

  \begin{itemize}
  \item \(\pd_k V = 0\) for a field \(k\).
  \item \(\pd_R M \le 1\) for PID \(R\) and a finitely generated \(R\)-module \(M\).
  \end{itemize}
\end{theorem}

%-------------------------------------------------------------------------------
\section*{Tensor Algebra}

\begin{definition}
  Let \(T_R^0(M) := R\) and for \(k \ge 1\), let
  \[T_R^k(M) = T^k(M) := M \otimes_R \cdots \otimes_R M\]
  Let \(T_R(M) = T(M) := \bigoplus_{k \ge 0} T^k(M)\).
  We have the associative \(\otimes: T^r(M) \otimes_R T^s(M) \to T^{r + s}(M)\).
  This \((T_R(M), +, \otimes, \cdot)\) is called \emph{the tensor algebra of \(M\) over \(R\)}.
\end{definition}

\begin{theorem}(Universal Property of Tensor Algebra)\
  Let \(R\) be a commutative ring with 1,
  \(A\) be any \(R\)-algebra with a given \(\varphi: M \to A\)
  where \(A\) is an \(R\)-module homomorphism.
  Then, there exists a unique \(R\)-algebra homomorhpism \(\psi\)
  such that the diagram
  \[\begin{tikzcd}
    M \ar[swap,"\Id"]{rd} \ar["\varphi"]{rr} & & A \\
    & T(M) \ar[swap, "\psi"]{ru}
  \end{tikzcd}\]
  where \(\Id: M \to M = T^1(M)\) is the identity.
\end{theorem}

\begin{definition}
  \(R\): comm. ring with 1, \(M\): \(R\)-module, \(T(M)\): tensor algebra.
  Let \(C(M) \subseteq T(M)\) be the two sided ideal generated by
  elements of the form \(m_1 \otimes m_2 - m_2 \otimes m_1\).
  Let
  \[S(M) = \Sym(M) := T(M)/C(M)\]
  Let \(S^k(M) = \Sym^k(M)\) be the image of \(T^k(M)\).

  \(\Sym(M)\) is called \emph{the symmetric algebra of \(M\) over \(R\)}.
\end{definition}
\begin{example}
  For \(M = R^n\), the free \(R\)-module of rank \(n\),
  \(\Sym(M) \simeq R[t_1, \cdots, t_n]\)
  where \(\Sym^k(M)\) as an \(R\)-module is spanned by th emonomials of degree \(k\),
  and free of rank \({k + n - 1 \choose n - 1}\).
\end{example}

\begin{theorem}
  \(\Sym(M)\) satisfies the universal property for \emph{commutative} \(R\)-algebra \(A\).
\end{theorem}

\begin{definition}
  \(R\): comm. ring with 1, \(M\): \(R\)-module, \(T(M)\): tensor algebra.
  Let \(A(M) \subseteq T(M)\) be the two sided ideal generated by
  elements of the form \(m \otimes m\).
  Let
  \[\wedge(M) = T(M)/A(M)\]
  Let \(\wedge^k(M)\) be the image of \(T^k(M)\).

  \(\wedge(M)\) is called \emph{the exterior algebra of \(M\) over \(R\)}.
\end{definition}

\begin{lemma}
  When \(m, m' \in M\), \(m \wedge m' = -m' \wedge m\).
\end{lemma}

\begin{theorem}
  \(\wedge^k(M)\) satisfies the universal property
  with the alternating \(R\)-multilinear \(\varphi: M \times \cdots \times M \to N\).
\end{theorem}

\begin{example}
  Let \(M\) be a free \(r\)-module of rank \(n\).
  Then \(\wedge^k(M)\) is free of rank \({n \choose k}\).
  In particular, \(\wedge(M)\) is in fact ``bounded above'' in that
  \[\wedge(M) = \bigoplus_{k=0}^n \wedge^k(M)\]
\end{example}

\begin{definition}(Lie Algebra)
  An \(F\)-vector space \(L\) is called a \emph{Lie algebra},
  if there is an alternating (\([x, x] = 0\)) bilinear map
  \[[-, -]: L \times L \to L\]
  satisfying the Jacobi identity:
  \[[x, [y, z]] + [z, [x, y]] + [y, [z, x]] = 0\]
\end{definition}
\begin{example}
  Let \(L\) be an \(F\)-algebra. Take \([x, y] = xy - yx\).
  Then, \(L\) gives a Lie algebra.
\end{example}

\begin{theorem}(Universal Envelopping Algebra)
Let \(L\) be a Lie algebra over \(F\)
and let \(A\) be an associative \(F\)-algebra
with the induced Lie algebra structure.
Let \(\phi: L \to A\) be a Lie algebra homomorphism,
i.e. \(\phi([x, y]) = \phi(x)\phi(y) - \phi(y)\phi(x)\)
for \(x, y \in L\).

Then there is an \(F\)-algebra \(U(L)\) together with
an \(F\)-linear map \(i: L \to U(L)\)
such that there is a unique \(F\)-algebra homomorphism
\(\psi: U(L) \to A\) such that the following commutes:
\[\begin{tikzcd}
  L \ar["i"]{rd} \ar["\phi"]{rr} & & A \\
  & U(L) \ar["\psi"]{ru} &
\end{tikzcd}\]

This is constructed by taking the two-sided ideal \(I(L)\) generated by
elements of the form \(x \otimes y - y \otimes x - [x, y]\),
and take \(U(L) = T(L) / I(L)\).
\end{theorem}

\begin{theorem}(Poincar\'e-Birkhoff-Witt)
  Let \(F\) be a field, \(L\) be an \(F\)-Lie algebra with a basis \(\calB\).
  Give a well-ordering on \(\calB\).

  A canonical monomial over \(\calB\) is a sequence \((x_1, \cdots, x_r)\)
  with \(x_1 \le \cdots \le x_r\), \(x_i \in \calB\).
  For the natural map \(i: L \to U(L)\),
  define \(i(x_1, \cdots, x_r) := i(x_1) \cdots i(x_r)\).

  Then \(i\) is injective on the set of all canonical monomials,
  and the images form an \(F\)-basis of \(U(L)\).
\end{theorem}
\begin{corollary}
  \(i: L \to U(L)\) is injective.
\end{corollary}

%-------------------------------------------------------------------------------
\section*{Linear Algebra}

\begin{definition}
  Let \(R\) be an integral domain, \(M\) be an \(R\)-module.
  The \emph{rank of \(M\) over \(R\)}
  is the maximum cardinality of \(R\)-linearly independent elements of \(M\).
\end{definition}

\begin{theorem}(A)
  Let \(R\) be a PID, \(M\) be a free \(R\)-module of rank \(n\),
  and \(N \subseteq M\).
  Then
  (1) \(N\) is free of rank \(m \le n\);
  (2) We can find a basis \(y_1, \cdots, y_n \in M\) such that
  for some \(a_1 \mid a_2 \mid \cdots \mid a_m\),
  \(a_1y_1, \cdots, a_my_m \in N\)
  and they form a basis of \(N\).
\end{theorem}

\begin{theorem}(Fundamental Theorem for Finitely Generated Modules over PID)
  
  Let \(R\) be a PID, \(M\) be a finitely generated \(R\)-modules.
  Then,
  \[M \simeq R^r \oplus R/(a_1) \oplus \cdots \oplus R/(a_m)\]
  for some \(a_i \in R\) such that \(a_1 \mid a_2 \mid \cdots \mid a_m\).
  The number \(r\) is unique and \(a_1, \cdots, a_m\)
  are uniquely decided up to units in \(R\).
\end{theorem}

\begin{corollary}
  Fundamental Theorem of Finitely Generated Abelian Group holds.
\end{corollary}

\begin{corollary}
  For \(R = k[t]\), FTFGMPID gives the Cyclic Decomposition Theorem.
\end{corollary}

%-------------------------------------------------------------------------------
\section*{Representation}

\begin{definition}
  Let \(G\) be a group, \(F\) be a field, \(V\) be an \(F\)-vector space.
  \begin{enumerate}
  \item A \emph{(linear) representation of \(G\) (over F)} is a group homomorphism
    \(\varphi: G \to \GL(V)\).

    The \emph{degree} of the representation if \(\dim_F(V)\).
  \item A \emph{matrix representation} of \(G\) is a homomorphism
    \(G \to \GL_m(V)\).

    When \(\dim_F V = m\), we have \(\GL(V) \simeq \GL_m(V)\),
    so we generally do not distinguish these two,
    unless we have reason to do so.

  \item A representation \(G \to \GL(V)\) is \emph{faithful} if it is injective.
  \end{enumerate}
\end{definition}

\begin{definition}
  Let \(G\) be a group and \(R\) be a ring.
  \(RG\) is the group ring, where
  (1) each element is in a form of \(\sum_{g \in G} \alpha_g \cdot g\);
  (2) addition is sum term-by-term;
  (3) multiplication is sum of multiplication of mult. of coefficients
  and mult. of \(G\)-terms.
\end{definition}

\begin{lemma}
  Let \(V\) be a set.
  \(V\) is an \(FG\)-module
  iff
  \(V\) is an \(F\)-vector space and there is a group homomorphism
  \(\phi: G \to \GL(V)\).
\end{lemma}

\begin{definition}
  Let \(V, W\) be representations of \(G\) over \(F\).
  A morphism of representation \(G\), \(\phi: V \to W\), is an \(FG\)-module homomorphism.
  Two representations \(V, W\) are \emph{equivalent}
  if they are isomorphic as \(FG\)-modules.
\end{definition}

\begin{corollary}
  The representations of \(G\) over \(F\) form a category
  \(G-\Rep/F\)
  and there is a natural equivalence of categories:
  \[FG-\Mod \Leftrightarrow G-\Rep/F\]
\end{corollary}

\begin{definition}
  Let \(G\) be a group.
  Let \(V = FG\) with the left \(FG\)-module structure.

  The induced representation \(\phi: G \to \GL(FG)\) is called the
  \emph{regular representation of \(G\)}.
\end{definition}

\begin{theorem}
  Regular representations are faithful.
\end{theorem}

\begin{theorem}(Maschke)
  Let \(G\) be a finite group, \(F\) be a field
  such that \(\textnormal{char}(F) = 0\) or \(\textnormal{char}(F) = p > 0\) with \(p \nmid |G|\).

  Let \(V\) be an \(FG\)-module and \(U \subseteq V\)
  be any \(FG\)-submodule.
  Then there is an \(FG\)-submodule \(W \subseteq V\)
  such that
  \[V \simeq U \oplus W\]
  as \(FG\)-modules.
\end{theorem}

\begin{theorem}(Wedderburn-Artin)
  Let \(R\) be a ring with 1. Then, TFAE
  \begin{enumerate}
  \item \(R\) is a \emph{semi-simple ring}.
  \item \(R\) is Artinian and its Jacobson radical is zero.
  \item Every \(R\)-module is projective.
  \item Every \(R\)-module is injective.
  \item Every \(R\)-module is completely reducible.
  \item The ring \(R\) considered as a left \(R\)-module is a direct sum
    \(R = L_1 \oplus \cdots \oplus L_n\) of simple \(R\)-modules \(L_i\),
    with \(L_i = Re_i\),
    such that \(e_i e_j = \delta_{ij}e_i\) and \(\sum e_i = 1\).
  \item As rings, \(R\) is isomorphic to \(R_1 \times \cdots \times R_r\)
    where \(R_j = M_{n_j}(D_j)\), for some division ring \(D_j\).
    The integer \(r, n_j\) and the ring \(D_j\) are unique.
  \end{enumerate}
  Note, semi-simple ring are Artinian and Noetherian.
\end{theorem}

\begin{corollary}(Corollary of Maschke's)
  \(F, G, FG\) be as before, and \(M\) be a finitely generated \(FG\)-module.
  Then, \(M\) is completely reducible.
  i.e. the group ring \(FG\) is a semi-simple ring.
\end{corollary}

\begin{theorem}(Schur's Lemma)
  Let \(R\) be a non-zero ring with 1.
  Let \(M, N\) be simple \(R\)-modules.
  Let \(\varphi: M \to N\) be an \(R\)-module homomorphism.
  Then, either \(\varphi\) is 0 or an isomorphism.
\end{theorem}

\begin{corollary}(Special Case of Schur's Lemma)
  If \(M\) is a simple \(R\)-module,
  then \(\Hom_R(M, M) = \End_R(M)\) is a division ring.
\end{corollary}

\begin{theorem}
  Let \(D\) be a division ring and \(R = M_n(D)\).
  Then \(R\) is a simple ring,
  i.e. the only two-sided ideals of \(R\) are 0 and \(R\).
\end{theorem}

\begin{theorem}
  Let \(D\) be a division ring and \(R = M_n(D)\).
  Then \(Z(R)\), the center of \(R\)
  is \(\{\alpha I_n \mid \alpha \in Z(D)\}\),
  where \(I_n\) is the \(n\)-by-\(n\) identity matrix.
\end{theorem}

\begin{theorem}
  Let \(D\) be a division ring and \(R = M_n(D)\).
  Let \(e_i = E_{ii}\).
  Then,
  \begin{itemize}
  \item \(e_i e_j = \delta_{ij} e_i\) and \(\sum e_i = 1\).
  \item Let \(L_i = Re_i\). Then they are simple left \(R\)-modules.
  \item Every simple left \(R\)-module is isomorphic to \(L_1\).
  \item As a left \(R\)-module, \(R = L_1 \oplus \cdots \oplus L_n\).
  \end{itemize}
\end{theorem}

\begin{theorem}
  Let \(D\) be a division ring,
  which is a finite dimension vector space over a field \(F\) with \(F \subseteq Z(D)\),
  and \(F = \overline{F}\).
  Then, \(D = F\).
\end{theorem}

\begin{theorem}
  Let \(G\) be a finite group.
  Then for some \(n_1, \cdots, n_r\) and \(r \ge 1\),
  we have
  \[\bbC G \simeq M_{n_1}(\bbC) \times \cdots \times M_{n_r}(\bbC)\]
\end{theorem}

\begin{corollary}
  \(|G| = \sum_{i=1}^r n_i^2\)
\end{corollary}

\begin{theorem}
  \(\bbC G\) has exactly \(r\) distinct isomorphism types of irreducible modules,
  i.e. there are exactly \(r\) non-equivalent irreducible representations
  of \(G\).
  Here, each \(M_{n_i}(\bbC)\) decomposes into a direct sum of
  \(n_i\) isomorphic irreducible modules.
\end{theorem}

\begin{theorem}
  Let \(G\) be a finite group.
  In the decomposition
  \[\bbC G \simeq M_{n_1}(\bbC) \times M_{n_r}(\bbC)\]
  into simple rings, the number \(r\) is equal to the number of
  conjugacy classes of \(G\).
\end{theorem}

\begin{corollary}
  When \(A\) is a finite abelian group,
  every irreducible representation over \(\bbC\) is of degree \(1\)
  (or 1-dimensional),
  and \(A\) has exactly \(|A|\) inequivalent irreducible representations.
\end{corollary}

\begin{theorem}
  Let \(G\) be a finite group.
  Then the number of inequivalent irreducible representations
  of degree 1 is \(|G / [G, G]|\).
\end{theorem}

\begin{definition}
  Let \(G\) be a group and \(F\) be a fixed field.

  A \emph{class function} \(\varphi: G \to F\) is a set-function,
  constant on conjugacy classes.
  i.e. \(\varphi(gxg^{-1}) = \varphi(x)\) for every \(g, x \in G\).

  Given a representation \(\varphi: G \to \GL(V)\), 
  the \emph{character of \(\varphi\)} is the set-function \(\chi: G \to F\)
  such that \(\chi(g) = \Tr \varphi(g)\).
\end{definition}

\begin{definition}
  Let \(G\) be a group, \(F\) be a fixed field, \(\varphi: G \to \GL(V)\) be a representation, and \(\chi = \Tr \varphi\) be the character of \(\varphi\).

  \(\chi\) is irreducible if \(\varphi\) is an irreducible representation.

  \(\chi\) is reducible if \(\varphi\) is an reducible representation.
\end{definition}

\begin{definition}
  The character of the trivial representation is the \emph{principal character}.
\end{definition}

\begin{theorem}
  For a representation \(\varphi: G \to \GL(V)\)
  and \(\chi = \Tr \varphi\), 
  \(\chi\) is a class function.
\end{theorem}

\begin{lemma}
  For a representation \(\varphi: G \to \GL(V)\)
  and \(\chi = \Tr \varphi\), 
  \(\chi(1_G) = \dim_F V = \deg \varphi\).
\end{lemma}

\begin{theorem}
  Every character of linear representation of group, \(\chi: G \to F\),
  can be extended \(F\)-linearly to \(\chi: FG \to F\).
\end{theorem}

\begin{theorem}
  For irreducible modules \(M_i\) with the irreducible character \(\chi_i\)
  and
  \[M \simeq M_1^{\bigoplus a_1} \oplus \cdots \oplus M_r^{\bigoplus a_r}\]
  then, the character \(\chi\) of \(M\) is
  \[\chi = \sum_i a_i \chi_i\]
\end{theorem}

\begin{theorem}
  Let \(G\) be a finite group.
  Let \(M, N\) be two finite dimensional representations of \(G\).
  Let \(\chi, \psi\) be their characters.
  Then, \(M \simeq N\) as \(\bbC G\)-modules iff \(\chi = \psi\).
\end{theorem}

\begin{corollary}
  For a given finite group \(G\),
  \begin{itemize}
  \item Irrducible characters \(\chi_i\) completely determine all finite dimensional representations of \(G\) up to equivalence.
  \item Irreducible characters completely determine all finitely generated \(\bbC G\)-modules up to isomorphism.
  \item There is an one-to-one correspondence between each set of irreducible characters and each irreducible \(\bbC G\)-module.
  \end{itemize}
\end{corollary}

\begin{definition}
  Let \(\calF\) be the vector space of \(\bbC\)-valued class functions on \(G\).
\end{definition}

\begin{definition}
  \(s_i\) is a step function such that
  \[s_i(K_j) = \left\{\begin{array}{ll}
    0 & \text{if } i \neq j \\
    1 & \text{if } i = j
  \end{array}\right.\]
\end{definition}

\begin{proposition}
  Irreducible characters \(\chi_i\) are in \(\calF\).
  \(e_i\) are also in \(\calF\).
  Since \(e_i\) are linearly independent and span \(\calF\),
  \(\dim_\bbC \calF = r\).
  where \(r\) is the number of distinct irreducible characters.
\end{proposition}

\begin{theorem}
  The irreducible characters \(\chi_1, \cdots, \chi_r \in \calF\)
  are linearly independent.
  In particular \(\chi_1, \cdots, \chi_r\) form a basis for \(\calF\).
\end{theorem}

\begin{definition}
  Define a hermitian inner product on \(\calF\) as follows:
  for \(\theta, \psi \in \calF\),
  define
  \[(\theta, \psi) := \frac{1}{|G|} \sum_{g \in G} \theta(g) \overline{\psi(g)}\]
\end{definition}

\begin{theorem}(1st Schur Orthogonality Theorem)
  Let \(G\) be a finite group, and \(\chi_1, \cdots, \chi_r\) be irreducible characters of \(G\).
  Then \((\chi_i, \chi_j) = \delta_{ij}\).

  i.e.  \(\{\chi_i\}_i\) is an orthonormal basis of the space of class functions \(\calF\).

  In particular, for \(\theta \in \calF\),
  \[\theta = \sum_{i=1}^r (\theta, \chi_i) \chi_i\]
\end{theorem}

\begin{lemma}
  In \(\bbC G\), we have
  \[e_i = \frac{\chi_i(1)}{|G|} \sum_{g \in G} \chi_i(g^{-1}) g\]
\end{lemma}

\begin{lemma}
  Let \(\psi: G \to \bbC\) be any character.
  Then,
  \begin{itemize}
  \item \(\psi(x)\) is a sum of roots of unity.
  \item \(\psi(x^{-1}) = \overline{\psi(x)}\) for all \(x \in G\).
  \end{itemize}
\end{lemma}

\begin{theorem}
  Let \(\varphi: G \to \GL(V)\) be a representation
  for a finite dimensional vector space \(V\).
  Let \(\{v_1, \cdots, v_n\}\) be a basis of \(V\)
  and let \(\{v_1^*, \cdots, v_n^*\}\) be its dual basis.

  Then \(\Tr\varphi(g) = \sum_{i=1}^n v^*_i(g \cdot v_i)\).
\end{theorem}

\begin{definition}
  Let \(V\) be a representation of \(G\).
  We can define the dual representation of \(V\) as follows.

  First let \(V^*\) be the dual vector space of \(V\).
  We need to define an action of \(G\).
  Here, one important thing is taht for \(g \in G\)
  and \(f \in V^*\),
  we take
  \[(g \cdot f)(v) := f(g^{-1}v)\]
  for \(v \in V\), using the inverse of \(g\).
\end{definition}

\begin{theorem}
  Let \(\psi_1, \psi_2\) be characters of \(G\).
  Then so is \(\psi_1 \psi_2\).
  In particular, \(\calF\) is closed under the product of class functions.

  More precisely, if \(\psi_i = \Tr\varphi_i\) for representations \(\varphi_i\),
  then \(\psi_1 \psi_2 = \Tr \varphi_1 \otimes \varphi_2\).
\end{theorem}

\begin{theorem}
  For a representation \(V\) of \(G\),
  let \(\chi\) be its character.

  Then the character for the dual representation \(V^*\)
  is the complex conjugate \(\overline{\chi}\).
\end{theorem}

\begin{corollary}
  \(\calF\) is closed under sum, product and complex conjugation.
\end{corollary}

%-------------------------------------------------------------------------------
\section*{Applications of Representation}

\begin{definition}
  Let \(G\) be a finite group.
  The \emph{character table} of \(G\) means a table of the following form is
  \begin{center}
    \begin{tabular}{c | c c c c}
      & \(1 = K_1\) & \(K_2\) & \(\cdots\) & \(K_r\) \\
      & \(1 = d_1\) & \(d_2\) & \(\cdots\) & \(d_r\) \\
      \hline
      \(\chi_1\) & \(1\) & \(1\) & \(\cdots\) & \(1\) \\
      \(\chi_2\) & \(*\) & \(*\) & \(\cdots\) & \(*\) \\
      \(\vdots\) & \(\vdots\) & \(\vdots\) & \(\ddots\) & \(\vdots\) \\
      \(\chi_r\) & \(*\) & \(*\) & \(\cdots\) & \(*\) \\
    \end{tabular}
  \end{center}
  where \(K_1, \cdots, K_r\) are the conjugacy classes,
  \(d_1, \cdots, d_r\) are the sizes of the orbits,
  \(\chi_1, \cdots, \chi_r\) are the irreducible characters.

  The values are \(\chi(g_j)\) where \(g_j \in K_j\).
\end{definition}

\begin{theorem}(1st Schur Orthogonality Theorem for Character Table)
  \[(\chi_i, \chi_j) = \frac{1}{|G|} \sum_{k=1}^r d_k \chi_i(g_k) \overline{\chi_j(g_k)} = \delta_{ij}\]

  i.e. the weighted rows of the character table are orthogonal.
\end{theorem}

\begin{theorem}(2nd Schur Orthogonality Theorem for Character Table)
  For \(x, y \in G\),
  \[\sum_{i=1}^r \chi_i(x) \overline{\chi_j(y)} = \left\{\begin{array}{ll}
    |C_G(x)| & \text{if } x, y \text{ conjugate,} \\
    0 & \text{otherwise.}
  \end{array}\right.\]

  i.e. the columns of the character table are orthogonal
\end{theorem}

Let \(F(G, \bbC) = \Mor(G, \bbC)\) be the set of all set-functions from \(G\) to \(\bbC\).

\begin{definition}
  For two functions \(f_1, f_2: G \to \bbC\),
  define the \emph{convolution} to be a function \(f_1 * f_2: G \to \bbC\)
  given by
  \[(f_1 * f_2)(g) = \sum_{h \in G} f_1(gh^{-1}) f_2(h)\]
\end{definition}

\begin{corollary}
  Consider the ring \((F(G, \bbC), +, *)\)
  with the coordinatewise sum and the convolution as the product.
  Then, the natural map \(\bbC G \to (F(G, \bbC), +, *)\)
  is a ring isomorphism.
\end{corollary}

\begin{definition}
  Let \(f \in F(G, \bbC)\) and let \(\varphi: G \to \GL(V)\)
  be a representation.
  Then the \emph{Fourier transform of \(f\) at \(\varphi\)}
  is defined to be
  \[\widehat{f}(\varphi) := \sum_{g \in G} f(g) \varphi(g)\]
\end{definition}

\begin{theorem}(Peter-Weyl)
  For simple ring decomposition:
  \[\bbC G \simeq M_{n_1}(\bbC) \times \cdots \times M_{n_r}(\bbC)\],
  we can write it as
  \[\bbC G \simeq \bigoplus_{i=1}^r \End(M_i)\]
\end{theorem}

\begin{definition}
  Let \(F\) be a field, \(G\) be a group, \(H \le G\) be a subgroup.
  For a representation \(\varphi: G \to \GL(M)\) of \(G\),
  denote by \(\Res_H^G M\) be the representation of \(H\)
  given by \(H \to G \xrightarrow{\varphi} \GL(M)\).
  This is called the \emph{restriction to \(H\)}.
\end{definition}

\begin{definition}
  Let \(F, G, H\) be as before.
  Let \(\varphi: H \to \GL(L)\) be a representation.
  The induced representation \(\Ind_H^G L\) is the representation of \(G\)
  given by \(FG \otimes_{FH} L\).
  This is called the \emph{induction to \(G\)}.
\end{definition}

\begin{theorem}
  \(\Ind_H^G\) and \(\Res_H^G\) are adjoint functors.
\end{theorem}

\begin{theorem}(Frobenius Reciprocity for Group Representation)
  Let \(M\) be a representation of \(H, N\) be a representation of \(G\).
  Then we have a natural bijection
  \[\Hom_{FG}(\Ind_H^G M, N) \simeq \Hom_{FH}(M, \Res_H^G N)\]
\end{theorem}

This can be proved by the Adjunction Theorem.

% --- Put all contents above, using \input{<FILE>} or \include{<FILE>}
\end{document}
