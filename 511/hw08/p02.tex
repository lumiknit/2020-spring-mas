\section*{Problem 2}

Prove below:
\begin{theorem} (Wedderburn-Artin)

Let \(R\) be a ring with unity.
Then TFAE:
\begin{enumerate}[label=(\arabic*)]
\item Every \(R\)-module is projective.
\item Every \(R\)-module is injective.
\item Every \(R\)-module is completely reducible.
\item The ring \(R\) considered as a left \(R\)-module
  is a direct sum \(R = L_1 \oplus \cdots \oplus L_n\)
  of simple \(R\)-modules \(L_i\), with \(L_i = Re_i\),
  such that \(e_i e_j = \delta_{ij} e_i\)
  and \(\sum e_i = 1\).
\item As rings, \(R\) is isomorphic to \(R_1 \times \cdots \times R_r\)
  where \(R_j = M_{n_j}(D_j)\), for some division ring \(D_j\).
  The integer \(r, n_j\) and the ring \(D_j\) are unique.
\end{enumerate}
\end{theorem}

\(M_n(R)\) is a matrix algebra for the ring \(R\).

\subsection*{Proof}

First, let's introduce slightly different statement of Wedderburn-Artin Theorem (4):
\begin{theorem}
Continue from Wedderburn-Artin:

(4-1) The ring \(R\) considered as a left \(R\)-module
  is a direct sum \(R = L_1 \oplus \cdots \oplus L_n\)
  of simple \(R\)-modules \(L_i\), with \(L_i = Re_i\),
  such that
  \(e_i\) is non-zero idempotent,
  \(e_i e_j = \delta_{ij} e_i\)
  and \(\sum e_i = 1\).
  The integer \(n\) is unique, and \(L_i\) is unique up to reordering.
\end{theorem}

Proof direction:
\begin{itemize}
\item (1) \(\Leftrightarrow\) (2)
\item (1), (2) \(\Leftrightarrow\) (3)
\item (3) \(\Leftrightarrow\) Existency of (4-1)
\item (4) \(\Leftrightarrow\) Existency of (4-1)
\item Existency of (4-1) \(\Leftrightarrow\) Existency of (5)
\item Uniqueness of (5)
\end{itemize}

%-----------------------------------------------------------
\subsection*{(1) \(\Rightarrow\) (2)}

Suppose all \(R\)-modules are projective.
Let \(I\) be an arbitrary \(R\)-module.

Take an exact sequence \(0 \to I \to M \to N \to 0\).
Since every \(R\)-modules are projective, \(N\) is projective.
Thus, the exact sequence is split.
Therefore, every exact sequence \(0 \to I \to M \to N \to 0\) are split.

Therefore, \(I\) is injective.
\qedsq

%-----------------------------------------------------------
\subsection*{(2) \(\Rightarrow\) (1)}

Suppose all \(R\)-modules are injective.
Let \(P\) be an arbitrary \(R\)-module.

Take an exact sequence \(0 \to M \to N \to P \to 0\).
Since every \(R\)-modules are injective, \(M\) is injective.
Thus, the exact sequence is split.
Therefore, every exact sequence \(0 \to M \to N \to P \to 0\) are split.

Therefore, \(P\) is projective.
\qedsq

%-----------------------------------------------------------
\subsection*{(1), (2) \(\Rightarrow\) (3)}

Suppose all \(R\)-modules are projective and injective.

Note that for this \(R\),
\(R\)-module is irreducible if and only if it's indecomposable.
Irreducible \(\Rightarrow\) indecomposible is trivial.
If \(R\)-module \(M\) is reducible,
Then, there is non-zero proper submodule \(N \subsetneq M\).
Then, take \(N / M\) and make a exact sequence:
\(0 \to N \to M \to M/N \to 0\).
Since every \(R\)-modules are injective, \(N\) is injective and
the above short exact sequence splits.
Thus, \(M = N \oplus M/N\).
This shows that \(M\) is decomposible.

Also, above process show that if we take any non-zero proper submodule \(N\)
of \(M\), there is a direct summand \(N'\) such that \(M = N \oplus N'\).
As we know that \(M/0 = M\) and \(M = M \oplus 0 = 0 \oplus M\),
every \(R\)-module's \(R\)-submodule is a direct summand.

Note that for any \(R\)-module \(M\), there is an irreducible submodule of \(M\).
First, let \(M = Rx\) for some \(x \in R\).
Then, every submodule of \(M\) is an ideal of \(R\).
In this case, there is a maximal right ideal of \(Rx\), \(N\),
by taking the union of all ideal of \(Rx\) which does not contain \(x\).
Then, \(M = N \oplus N'\) for some \(R\)-module \(N'\).
If \(N'\) is reducible,
there is a submodule \(S\), \(S'\) of \(N'\) such that \(N' = S \oplus S'\).
Then, \(x\) must be contained in exactly one of \(S\) or \(S'\).
WLOG let's assumed that \(x \in S\).
Then, \(x \not\in S'\).
Then, \(S' \cup N\) is larger than \(N\), which does not contain \(x\).
Thus, it violates the maximality of \(N\).
Therefore, \(N'\) must irreducible.
For arbitrary \(R\)-module \(M\),
because \(M\) is proective,
\(M\) is a direct summand of some free \(R\)-module \(\bigoplus_\Lambda R\).
Thus, \(M\) is isomorphic to some direct sum of
ideals of \(R\).
If \(M\) is non-zero reducible,
there is some \(\alpha \in \Lambda\) such that the \(\alpha\)-th entry of the isomorphic image of \(M\) is the non-zero ideal of \(R\).
Then, it has an irreducible submodule \(N\).
Then, the inverse image of \(N\) into \(M\) is an irreducible submodule.

Let \(S\) be a sum of every irreducible submodules of \(M\).
Then, there is a \(R\)-submodule \(S'\) of \(M\) such that \(M = S \oplus S'\). (\(S'\) may be zero.)
Suppose that \(S' \neq 0\).
If \(S'\) is irreducible, it's a contradiction because it should be in \(S\).
Then, \(S'\) should contain an irreducible submodule direct summand.
But this also makes a contradiction
since every irreducible submodule should be in \(S\).
Thus, \(S'\) must be \(0\) and \(M = S\).
Thus, \(M\) is a sum of some irreducible submodules.
Let \(M = \sum_{\alpha \in \Lambda} M_\alpha\)

Then, for some index subset \(S \subseteq \Lambda\),
\(\sum_{\alpha \in S} M_\alpha\) can be a direct sum \(\bigoplus_{\alpha \in S} M_\alpha\)
(if \(M_\alpha\) are disjoint except 0...)
For this kind of sums, which are direct sums, we can give a partial order \(\subseteq\),
which is the set inclusion relation.
Then, if we pick any chains of \(\subseteq\), by taking union of all items,
we have a maximum element,
and the element also a direct sum.
(Because, if \(\sum_{\alpha \in S} M_\alpha\) is maximum, for any \(M_a\), \(M_b\) for \(a, b \in S\),
there is a element \(\sum_{\alpha \in T} M_\alpha\) in the chain
such that \(S \subseteq T\)
and \(a, b \in T\). Thus, \(M_a\) and \(M_b\) should be disjoint except 0.)
Since \(M\) has at elast one irreducible submodule,
which can be considered as a direct sum of irreducible submodules,
by Zorn's Lemma, there is a maximal element.
Let \(M' = \sum_{\alpha \in \Gamma} M_\alpha = \bigoplus_{\alpha \in \Gamma} M_\alpha\)
be the maximal element in \(M\).
If \(M = M'\) we are done.
And suppose that \(M' \neq M\). (i.e. \(M'' \neq 0\))
As we showed above,
there is \(R\)-module \(M''\) such that \(M = M' \oplus M''\).
If \(M''\) is irreducible,
then \(M\) is a direct sum of a direct sum of irreducible modules, \(M'\), and a irreducible module \(M''\).
Thus \(M\) is a direct sum of irreducible modules.
If \(M''\) is reducible, there is a irreducible submoddule \(M''' \subseteq M''\).
However, since \(M'''\) and \(M'\) are disjoint except 0,
\(M' + M''' = M' \oplus M'''\).
It violtaes the maximality of \(M'\).

Therefore, \(M\) is a direct sum of irreducible modules.
It means \(M\) is completely reducible.
\qedsq

%-----------------------------------------------------------
\subsection*{(3) \(\Rightarrow\) (1), (2)}

Suppose all \(R\)-modules are completely reducible.

Let's take an arbitrary exact sequence
\[0 \to K \xrightarrow{f} L \xrightarrow{g} M \to 0\]
Since each \(K, L, M\) are completely reducible,
there are some irreducible \(R\)-modules
\(\{K_\alpha\}_{\alpha \in \Gamma}, \{L_\alpha\}_{\alpha \in \Lambda}, \{M_\alpha\}_{\alpha \in \Omega}\)
such that
\(K = \bigoplus_{\alpha \in \Gamma} K_\alpha\),
\(L = \bigoplus_{\alpha \in \Lambda} L_\alpha\),
\(M = \bigoplus_{\alpha \in \Omega} M_\alpha\).
Then, the below is exact:
\[0 \to \bigoplus_{\alpha \in \Gamma} K_\alpha \xrightarrow{f} \bigoplus_{\alpha \in \Lambda} L_\alpha \xrightarrow{g} \bigoplus_{\alpha \in \Omega} M_\alpha \to 0\]

Let's think about submodules of \(L\).
Since each \(L_\alpha\) are irreducible,
only submodules of \(L_\alpha\) is \(0\) or \(L_\alpha\).
Let \(\pi_\alpha: L \to L_\alpha\) be a canonical projection.
Suppose that there is some submodule \(L' \subseteq L\).
Then, \(\pi_\alpha(L')\) must be a submodule,
since \(\pi_\alpha\) is a homomorphism.
Thus, there are only two choices about \(\pi_\alpha(L')\):
\(\pi_\alpha(L') = L_\alpha\)
or \(\pi_\alpha(L') = 0\).
Thus, \(L'\) must be a form of
\(L' \simeq \bigoplus_{\alpha \in \Lambda'} L_\alpha\)
where \(\Lambda' \subseteq \Lambda\).

Note that \(\im f\) is an isomorphic image of \(K\) since \(f\) is injective.
In addition, \(\im f \subseteq L\).
It means, there is a subset \(S \subseteq \Lambda\)
such that \(\im f \simeq \bigoplus_{\alpha \in S} L_\alpha\).
And, \(K \simeq \bigoplus_{\alpha \in S} L_\alpha\).
Let \(\varphi: \bigoplus_{\alpha \in S} L_\alpha \to K\) be an inverse of \(f\).

Also, we can make a projection \(\pi: L \to \im f\) such that:
\(\pi(\bfa) = \bfa'\)
where the \(\alpha\)-th entry of \(\bfa'\) is
the \(\alpha\)-th entry of \(\bfa\) if \(\alpha \in S\),
otherwise \(\alpha\)-th entry of \(\bfa'\) is zero.

Thus, \(\varphi \circ \pi\) give a homomorphism from \(L\) to \(K\).
And since \(\Cod(\pi) = \im f\), \(\pi \circ f = f\) and
\(\varphi \circ \pi \circ f = \varphi \circ f = \Id_K\).
Therefore, \(0 \to K \to L \to M \to 0\) splits.

Since \(0 \to K \to L \to M \to 0\) was chosen arbitrarily,
every exact sequence of \(R\)-modules splits.
If we fix \(K\) and just change \(L, M\),
we obtain the result that \(K\) is injective.
If we fix \(M\) and just changed \(K, L\),
we obtain the result that \(M\) is projective.

Thus, it shows that every \(R\)-module is injective and projective.
\qedsq

%-----------------------------------------------------------
\subsection*{(3) \(\Rightarrow\) Existency of (4-1)}

Suppose that every \(R\)-module is completely reducible.

Then, \(R\) as a left \(R\)-module is also completely reducible.
Then, there are irreducible \(R\)-modules \(\{R_\alpha\}_{\alpha \in \Lambda}\)
such that
\[R = \bigoplus_{\alpha \in \Lambda} R_\alpha\]
Since \(1 \in R\), there is some finite subset \(S\) of \(\Lambda\)
and \(\{r_\alpha\}_{\alpha \in \Lambda}\)
where \(r_\alpha \neq 0\) iff \(\alpha \in S\),
\[1 = \bigoplus_{\alpha \in \Lambda} r_\alpha\]

Note that an ideal \(R \cdot 1\) is \(R\). Therefore,
\[R = R \cdot 1 = \bigoplus_{\alpha \in \Lambda} R'_\alpha \simeq \bigoplus_{\alpha \in S} R_\alpha\]
where \(R'_\alpha = R \cdot r_\alpha\) if \(\alpha \in S\), and \(R'_\alpha = 0\) otherwise.
Since \(S\) is a finite set,
for \(n = |S|\),
we can reindex \(S\) into \(\{1, \cdots, n\}\).
Then, we obtain
\[R = \bigoplus_{k=1}^n R_k = R_1 \oplus \cdots \oplus R_k\]
Thus, \(R\) can be expressed as a finite direct sum of some \(R\)-submodules
of \(R\).

Note that since each \(R_k\) is closed under a multiplication by \(R\),
they are left ideals of \(R\).
Also, since \(R_k\) is irreducible,
there is no proper non-zero \(R\)-submodule of \(R_k\) for each \(k\).
It means, each \(R_k\) are minimal (simple?) left ideal.

Note that every minimal left ideal \(I\) of \(R\) is a left principal ideal.
Because, if not, for any \(x \in I \bs \{0\}\),
\(Rx\) is a left ideal of \(R\).
Since \(x \neq 0\), \(Rx \neq \{0\}\) (\(\because\) \(Rx\) must contains \(1 \cdot x = x\))
Thus \(Rx = I\).
Note that since \(x\) is chosen arbitrarily,
\(I\) can be expressed for any non-zero \(x \in I\) as \(Rx\).

Let \((r_1, \cdots, r_n)\) be the image of \(1_R\) into the \(\bigoplus_{k=1}^n R_k\).
Because of the construction of index set \(S\),
every \(r_k\) must be non-zero.
Thus, \(R_k = R r_k\) for each \(k\).
Also, the direct sum of \(r_k\) is 1.

Then, 
\[r_k \cdot (r_1, \cdots, r_n)
= (0, \cdots, 0, r_k, 0, \cdots 0)\]
must holds, because \((r_1, \cdots, r_n)\) is an identity.
Thus, \(r_k r_j = \delta_{jk} r_k\).

Therefore, if we take \(e_k = r_k\) and \(L_k = Re_k\),
\(R = L_1 \oplus \cdots \oplus L_n\),
each \(L_k\) are simple,
\(e_ie_j = \delta_{ij}e_i\)
and
\(\sum e_i = 0\) holds.
\qedsq

%-----------------------------------------------------------
\subsection*{Existency of (4-1) \(\Rightarrow\) (3)}

Note that (4-1) means \(R\) as a left \(R\)-module is completely reducible.

Let \(M\) be an arbitrary left \(R\)-module.
Then, there is a generating set \(\calB = \{b_\alpha\}_{\alpha \in \Lambda}\) 
of \(M\).
Then, there is a homomorphism \(\varphi: \bigoplus_\Lambda R \to M\) such that
\[\varphi((r_\alpha)_{\alpha \in \Lambda}) = \bigoplus_{\alpha \in \Lambda} r_\alpha b_\alpha\]
Then, the image of \(\varphi\) is \(M\).
Therefore, \(M\) is isomorphic to the submodule of \(\bigoplus_\Lambda R\).

Since \(R\) is a direct sum of simple \(R\)-submodules,
\(\bigoplus_\Lambda R\) is also a direct sum of simple \(R\)-submodules.

Since a submodule of a direct sum of simple \(R\)-submodules
is a direct sum of simple \(R\)-submodules,
\(M\) is isomorphic to some direct sum of simple \(R\)-submodules.
(See the proof of (3) \(\Rightarrow\) (1), (2).)

Thus \(M\) is a direct sum of \(R\)-submodules.

Since \(M\) is chosen arbitrarily,
every \(R\)-modules are completely reducible.
\qedsq

%-----------------------------------------------------------
\subsection*{(4) \(\Rightarrow\) Existency of (4-1)}

Note that \(e_i e_j = \delta_{ij} e_i\) shows that \(e_i\) is idempotent.
Then, there are two possibility:
\(e_i = 0\) or not.
If \(e_i = 0\), then \(R e_i = 0\).
Thus, even if we remove \(L_i\) from the direct sum \(\bigoplus_{i=1}^n L_i\),
it is still \(R\).
Also, subtracting by \(e_i\), which is zero, from \(\sum e_i = 1\)
does not change the sum.
Therefore,
\[R = \bigoplus_{k=1}^{i-1} Re_k \oplus \bigoplus_{k=i+1}^{n} Re_k\]
and \(\sum_{k \neq i} e_k = 1\)

Since there are only finite number of \(e_k\),
just find all \(e_k\) which is zero and omit them.
Then, after reordering,
we obtain new \(n' \le n\), non-zero idempotent
\(\{e'_k\}_{k=1}^{n'} \subseteq \{e_1, \cdots, e_n\}\)
such that
\[R = \bigoplus_{k=1}^{n'} Re'_k\],
\(e'_ie'_j = \delta_{ij}e'_i\) and \(\sum e'_i = 1\).
\qedsq

%-----------------------------------------------------------
\subsection*{Existency of (4-1) \(\Rightarrow\) (4)}
This is trivial since the statement (4-1) is stronger than one of (4).
\qedsq

%-----------------------------------------------------------
\subsection*{Existency of (4-1) \(\Rightarrow\) Existency of (5)}

Let's begin with some notes.

First, \(e_k R e_k\) is a division ring for each \(k\).
To show this fact, it's enough to show that \(e_k R e_k\) contains units of
every non-zero elements.
First \(e_k = e_k^2 = e_k 1 e_k\) is an identity of \(e_k R e_k\).
(\(\because\)
\(e_k a e_k e_k 1 e_k = e_k a e_k^3 = e_k a e_k\)
and
\(e_k 1 e_k e_k a e_k = e_k^3 a e_k = e_k a e_k\)
for any \(a \in R\).)
Let \(e_k a e_k \in e_k R e_k\) be a non-zero element.
Then, \(R (e_k a e_k)\) is an ideal contained in \(R e_k\).
(\(\because\) For any \(r \in R\),
\(r e_k a e_k \in R (e_k a e_k)\)
and since \(r e_k a \in R\)
\((r e_k a) e_k \in R e_k\).)
However, since \(R e_k\) is simple, \(R e_k a e_k\) is \(0\) or \(R e_k\).
But since the non-zero element \(e_k a e_k = 1 \cdot e_k a e_k\)
is in \(R e_k a e_k\),
\(R e_k a e_k\) is non-zero.
Thus, \(R e_k a e_k = R e_k\).
Then, there is \(r \in R\) such that
\(r e_k a e_k = e_k\).
Then, because \(e_k^2 = e_k\),
\((e_k r e_k) (e_k a e_k) = e_k r e_k a e_k = e_k^2 = e_k\).
Thus, \(e_k r e_k\) is the inverse of \(e_k a e_k\) in \(e_k R e_k\).

\((Re_k)e_k(Re_k) = Re_k\).
First, since \((Re_k)e_k \subseteq R\) and \(Re_k\) is a left ideal,
\((Re_k)e_k(Re_k) \subseteq Re_k\).
Since \(Re_k\) is simple, \((Re_k)e_k(Re_k)\) is \(0\) or \(Re_k\).
Since \((e_k e_k)e_k(e_k e_k) = e_k^5 = e_k\), which is non-zero,
is in \((Re_k)e_k(Re_k)\),
\((Re_k)e_k(Re_k)\) is non-zero.
Thus, \((Re_k)e_k(Re_k) = Re_k\).
Since every element of \((Re_k)e_k(Re_k)\) is a finite sum of
\((a_i e_k) e_k (b_i e_k)\) for \(a_i, b_i \in R\),
there is \(n_k \in \bbN\) and \(a_i, b_i \in R\) such that
\[1 = \sum_{i=1}^{n_k} (a_i e_k) e_k (b_i e_k)
= \sum_{i=1}^{n_k} a_i e_k b_i e_k\]

Note that for any division ring \(D\),
every module \(M\) over \(D\) is free.
This can be shown as follows:
since \(D\) is a simple ring,
as the proof of Problem 1(1),
we can show that every \(D\)-module is injective using Baer's Criterion.
Then, by (1) \(\Leftrightarrow\) (2), we obtain every \(D\)-module is projective.
Thus, for any \(D\)-module \(M\),
it's a direct summand of \(\bigoplus_\Lambda D\).
However, since \(D\) is simple,
the only possible submodule of \(\bigoplus_\Lambda D\)
is another free \(D\)-module \(\bigoplus_\Gamma D\).
(\(\because\) each entry of the submodule of \(M\)
must be a submodule of \(D\), but there are only two possible submodule of \(D\)
one is \(0\) and the other one is \(D\).)
Thus, \(M\) must be some free \(D\)-module.
This implies that we can say about basis for \(D\)-module.

\(Re_k\) is a finitely generated right \(e_k Re_k\)-module.
First, since \(Re_k\) is an ideal, it's an abelian group.
Let \(e_k Re_k\) acts on the left side of \(Re_k\) as:
\[(re_k) \cdot (e_k se_k) = (re_k)(e_k se_k) = (re_k s)e_k\]
Since \(re_ks\) is in \(R\), it's in \(Re_k\).
Thus the action is closed.
Since it's an action between rings,
it satisfies all properties of module.
Also, for arbitrary \(r \in R\),
\[re_k = \left(\sum_{i=1}^{m_k} a_i e_k b_i e_k\right) re_k
= \sum_{i=1}^{m_k} a_i (e_k (b_i e_k r) e_k) \]
Since \(b_i r \in R\), \(e_k b_i e_k r e_k \in e_k R e_k\).
Thus, \(\calA_k = \{a_i\}_{i=1}^{m_k}\) generates every element
of \(Re_k\).
Thus, \(Re_k\) is finitely generated,
such that \(Re_k \simeq (e_k R e_k)^{m_k}\).

In this case, we can reduce the number of \(n_k\) as small as possible
to obtain \(\calA_k\) which is a basis of \(Re_k\).

What we want to say is \(Re_k \simeq M_{n_k}(e_k R e_k)\).

Note that \(M_{n_k}(e_k R e_k) \simeq \Hom_{e_k R e_k}(Re_k, Re_k)\).
It's because \(Re_k\) is \(n\)-dimensional module over \(e_k R e_k\).


Since \(Re_k\) is simple,
by Schur's Lemma,
\(\Hom_R(Re_k, Re_k)\) is a division ring.
Then, take \(f \in \Hom_R(Re_k, Re_k)\).
There is \(f(e_k) = ae_k\) for some \(a \in R\).
Then, \(f(re_k) = rf(e_k) = rae_k\).
This shows that \(\Hom_R(Re_k, Re_k)\)
is determined that where \(e_k\) is mapped to.

Make a homomorphism \(F: \Hom_R(Re_k, Re_k) \to e_k R e_k\)
such that \(F(f) = e_k f(e_k)\).
Since both \(\Hom_R(Re_k, Re_k)\) and \(e_k R e_k\) are simple as division ring,
and \(F\) is non-zero map since \(R e_k\) contains at least one non-zero
element \(a\) and it makes a homomorphism \(e_k \mapsto a e_k\), which 
is non-zero map,
\(F\) is an isomorphism.
Thus, \(e_k R e_k \simeq \Hom_R(Re_k, Re_k)\).

Make a homomorphism \(G: Re_k \to \Hom_{e_k R e_k}(Re_k, Re_k)\)
such that \(G(re_k): ae_k \mapsto re_k ae_k\).
This is injective since if \(G(re_k) \equiv 0\),
\(G(re_k)(e_k) = re_k e_k = re_k = 0\),
and \(\ker G\) is zero.
Also, for some \(f \in \Hom_{e_k R e_k}(Re_k, Re_k)\),
\begin{align*}
  f(c e_k) = f(1 \cdot ce_k)
  &= f(\sum_{i=1}^{n_k} a_i e_k b_i e_k c e_k)
  \\&= \sum_{i=1}^{n_k} f(a_i e_k b_i e_k c e_k)
  \\&= \sum_{i=1}^{n_k} f(a_i e_k) e_k b_i e_k c e_k
  \\&= \left(\sum_{i=1}^{n_k} f(a_i e_k) e_k b_i e_k\right) c e_k
\end{align*}
for arbitrary \(c \in R\).
Thus, if we take \(r = \sum_{i=1}^{n_k} f(a_i e_k) e_k b_i e_k\),
\(G(r e_k) = f\).
Therefore, \(G\) is an isomorphism.
This shows that \(Re_k \simeq \Hom_{e_k R e_k}(Re_k, Re_k)\).

Therefore, for each \(k\),
\(Re_k \simeq \Hom_{e_k R e_k}(Re_k, Re_k) \simeq M_{n_k}(e_k R e_k)\).

Then, just take \(r\) be a number of direct sumamnds in \((4-1)\),
\(n_k = \dim_{e_k R e_k}(Re_k)\),
\(D_k = e_k R e_k\).
And since we show that \(L_k = Re_k \simeq M_{n_k}(e_k R e_k)\)
and the finitely many direct sum is just a cartesian product,
we obtain
\begin{align*}
  R &= L_1 \oplus \cdots \oplus L_n
  \\&= M_1(D_1) \times \cdots \times M_r(D_r)
\end{align*}
\qedsq

%-----------------------------------------------------------
\subsection*{Existency of(5) \(\Rightarrow\) Existency of (4-1)}

Let \(R \simeq R_1 \times \cdots \times R_r\)
where \(R_j = M_{n_j}(D_j)\) for some division rings \(D_j\).
Since it's a finite product,
\(R_1 \times \cdots \times R_r = R_1 \oplus \cdots \oplus R_r\).

Each \(M_{n_k}(D_k)\) is simple.
It's because,
if an ideal of \(M_{n_k}(D_k)\) contains a non-zero matrix,
it must be \(M_{n_k}(D_k)\).
Let's show it.
First, let \(e_{i,j}\) be a matrix of \(M_{n_k}(D_k)\)
such that only \(i\)-th row \(j\)-th column entry is \(1\)
and all other entries are zero.
Let's pick any non-zero matrix \(m \in M_{n_k}(D_k)\).
Then \(m\) must contains a non-zero entry
at \(i\)-th row, \(j\)-th column entry for some \(i, j \in \{1, \cdots, n\}\).
Let the entry be \(u\).
Since \(M_{n_k}(D_k)\) is a division ring,
there is \(u^{-1} \in M_{n_k}(D_k)\).
Then,
\((u^{-1}e_{x, i}) m  = e_{x, j}\)
and
\(m (u^{-1}e_{j, x})  = e_{i, x}\).
Then, if an ideal of \(M_{n_k}(D_k)\) contains \(m\),
it contains \(e_{i, x}, e_{x, j}\) for every \(x\).
Also, if we take \(e_{y, i}\) for some \(y\),
we obtain \(e_{y, x} = e_{y, i} e_{i, x}\).
Thus, the ideal contains all \(e_{y,x}\) for every \(y,x \in \{1, \cdots, n\}\).
Then, for \((a_{i,j})_{i,j \in \{1, \cdots, n_k}\} \in M_{n_k}(D_k)\),
\[(a_{i,j})_{i,j \in  \{1, \cdots, n_k\}} = \sum_{i=1}^{n_k} \sum_{j=1}^{n_k} a_{i,j} e_{i,j}\]
Thus, the ideal is \(M_{n_k}(D_k)\).
Therefore, \(M_{n_k}(D_k)\) has only two ideal: zero or itself.
it shows \(M_{n_k}(D_k)\) is simple.

Let \(R' = R'_1 \oplus \cdots \oplus R'_r\)
where \(R'_k = 0 \times \cdots \times 0 \times R_k \times 0 \times \cdots \times 0\).
Then, \(R'_k \simeq R_k\)
and \(R \simeq R'\).

Then, the left product of \(R\) to \(R'_k\)
is same as \(R_k\) to \(R'_k\),
because every entry of \(R'_k\) are zero except \(k\)-th one,
\begin{align*}
  (a_1, \cdots, a_r) \cdot (0, \cdots, 0, b_k, 0, \cdots, 0)
  &= (a_1 \cdot 0, \cdots, a_{k-1} \cdot 0, a_k \cdot b_k, a_{k+1} \cdot 0, \cdots, a_r \cdot 0)
  \\&= (0, \cdots, 0, a_k \cdot b_k, 0, \cdots, 0)
\end{align*}
Thus if there is an ideal \(I\) contained in \(R'_k\),
since \(R'_k\) is simple,
\(I\) is isomorphic to \(0\) or \(R'_k\) which are only possible ideals of \(R_k\).
Therefore, each \(R_k\) is simple.

Let \(e_k \in R'\) such that all entries are zero except \(k\)-th entry which is \(1\).
First, multiplication by \(e_k\) is a projection to \(R'_k\).
Thus, \(R'_k = R' \cdot e_k\).
\(e_j \cdot e_k = \delta_{jk} e_j\).
It's because, if \(j = k\), zero entries are multiplicated with zero-valued entries
and the one-valued entry, at \(j\)-th, is multiplicated wiht one-valued entries,
and if \(j \neq k\), then each one-valued entries are vanished
since they are multiplicated with some zero-valued entries.
Also, note that the multiplicative identity of
finite direct sum of rings with unity
is \((1, 1, \cdots, 1)\).
Thus, \((1, 1, \cdots, 1) = \sum_{k=1}^r e_k = 1_R\).

Therefore, \(R = \bigoplus_{k=1}^r R e_k\)
where \(e_j e_k = \delta_{j, k} e_j\),
\(\sum_{j=1}^r e_j = 1_R\).

\qedsq

%-----------------------------------------------------------
\subsection*{Uniqueness of (4-1)}

First, instead of the result of (5),
let's show some uniqueness of the result of (4).

Suppose that there are two distinct set of 
simple \(L_i = Re_i\) for each \(i = 1, \cdots, n\)
and
simple \(L'_i = Re'_i\) for each \(i = 1, \cdots, n'\)
which satisfies all properties given in (4).
(i.e.
\(R = \bigoplus_{i=1}^n L_i = \bigoplus_{i=1}^{n'} L'_i\),
\(e_ie_j = \delta_{ij}e_i\),
\(e'_ie'_j = \delta_{ij}e'_i\),
\(\sum e_i = \sum e'_i = 1\).)

Just take a product with \(e_k\) for some \(k\).
Then we obtain
\begin{align*}
  R e_k &= \bigoplus_{i=1}^n Re_i e_k
  \\&= \bigoplus_{i=1}^{n'} R e'_i e_k
\end{align*}
Note that \(R(e'_ie_k)\) is a left principal ideal
and since \(Re'_i\) is simple and \(Re'_ie_k \subseteq Re'_i\),
\(Re'_ie_k\) is \(0\) or \(Re'_i\) for each \(i\).
However, if two of \(Re'_ie_k\) is non-zero for the fixed \(k\),
then, \(Re_k\), which is simple too, can be expressed as the
direct sum of more than one non-zero modules.
Thus it's a contradiction.
Therefore, for each \(k\), at most one of \(Re'_ie_k\)
can be non-zero, i.e. \(Re'_ie_k = Re_k\). \(\cdots\)(a)

Also, suppose that if every \(Re'_ie_k\) is zero for each \(i\).
then, we obtain \(R e_k = 0\).
But since \(R e_k\) is not zero (because at least it contains an idempotent
element \(e_k\)),
it's a contradiction.
Thus, for each \(k\), at least one of \(Re'_ie_k\)
must be non-zero, i.e. \(Re'_ie_k = Re_k\). \(\cdots\)(b)

Therefore to satisfy (a) and (b),
\(n\) must be equal to \(n'\).
(If \(n > n'\),
to satisfy (b),
for each \(k\),
there exists \(i\) such that \(Re'_ie_k = Re_k\).
However, since \(n > n'\), by Pigeonhole Principle,
there are two \(k, k'\) such that
\(Re'_ie_k = Re_k\) and \(Re'_ie_k' = Re_k'\) for some \(i\).
This violates (a).
If \(n < n'\),
switch \((L_*, n, e_*)\) and \((L'_*, n', e'_*)\)
to obtain the case of \(n > n'\), since the equivalence is symmetric,
then, as we showed above, it violates (a).
Thus, the previous \((L_*, n, e_*)\) and \((L'_*, n', e'_*)\)
\(n < n'\) is impossible.)
Also, there must be a ono-to-one correspondence between \(k \in \{1, \cdots, n\}\) and \(i \in \{1, \cdots, n'\}\) such that \(Re'_ie_k = Re_k\).

Let \(k \in \{1, \cdots, n\}\) and \(i \in \{1, \cdots, n'\}\) such that \(Re'_ie_k = Re_k\).
Then, since \(e_k \in Re_k\), \(e_k \in Re'_i e_k \subseteq Re'_i\) too.
But as we showed in the proof of (3)\(\Rightarrow\)(4),
minimal ideal \(I\) can be expressed as \(Rx\) for any non-zero element \(x\)
of \(I\).
Since \(e_k\) is non-zero,
\(Re'_i = Re_k\).

Therefore, this shows that \(n = n'\) and there is some permutation
\(\pi \in S_n\) such that \(L_i = L'_{\pi(i)}\) for \(i = 1, \cdots, n\).
\qedsq

%-----------------------------------------------------------
\subsection*{Uniqueness of (5)}

Let's use the uniqueness of (4-1) and construction between (4-1) and (5)
we shown above.

Suppose that there are two set of
\((r, n_*, R_*, D_*)\)
and
\((r', n'_*, R'_*, D'_*)\)
such that
\(R_k = M_{n_k}(D_k)\), 
\(R'_k = M_{n'_k}(D'_k)\)
and
\[R \simeq R_1 \times \cdots \times R_r \simeq R'_1 \times \cdots \times R_{r'}\]

First, in the construction from (5) to (4-1),
\(n\) in (4-1) is equal to \(r\).
(i.e. the number of term in the product of (5) is equal to
the number or direct summands of (4-1).)
Thus, there is \(e_1, \cdots, e_r\) and \(e'_1, \cdots, e'_{r'}\) in \(R\)
such that
\[R = R e_1 \oplus \cdots \oplus R e_r = R e'_1 \oplus \cdots \oplus R e'_{r'}\]
and each \(e_*\) and \(e'_*\) satisfies all conditions in (4-1).
Then, by the uniqueness of (4-1),
\(r = r'\) and there is some permutation \(\pi \in S_r\) such that
\(R e_k = R e'_{\pi(k)}\)
for every \(k = 1, \cdots, r\).

First, since \(n = r\), \(n' = r'\) by the construction (5) to (4-1),
and \(r = r'\) by the uniqueness of (4-1),
\(n = n' = r = r'\) must holds.

Next, in the construction (5) to (4-1),
each \(M_{n_k}(D_k)\) is correspond to \(Re_k\).
In the same way, each \(M_{n'_k}(D'_k)\) is correspond to \(R e'_k\).
Since \(R e_k = Re'_{\pi(k)}\),
by reordering \(n'_*, D'_*, e'_*\) using \(\pi\),
we obtain \(R e_k = R e'_k\) for each \(k\).
Note that \(R e_k\) is constructed
so that \(M_{n_k}(D_k)\) is isomorphic to \(R e_k\).
Thus,
\[M_{n_k}(D_k) \simeq R e_k = R e'_k \simeq M_{n'_k}(D'_k)\]

Therefore, at this point,
we know that \(n = n'\)
and each \(M_{n_k}(D_k)\) is unique up to up to isomorphism
after reordering.

The last thing what we need to show is,n
\(n_k = n'_{\pi(k)}\) and \(D_k \simeq D'_{\pi(k)}\) holds for each \(k\)
where \(\pi \in S_n\) is a permutation such that
\(M_{n_k}(D_k) \simeq M_{n'_{\pi(k)}}(D'_{\pi(k)})\).

Suppose that \(D, D'\) be division rings, \(n, n' \in \bbN\)
and \(R = M_n(D), R' = M_{n'}(D')\) ,
such that \(R \simeq R'\).

Then, by the theorem in the lecture,
\(R\)

Let \(e_{i,j}\) be an matrix of \(R\)
such that every entry are zero except the \(i\)-th row \(j\)-th column entry
which is valued by \(1\).
In the same way, let \(f_{i,j}\) be an matrix of \(R'\)
such that all entries are valued by zero except \((i, j)\)-entry which is valued
by \(1\).

Let \(\varphi: R \to R'\) be an isomorphism between \(R\) and \(R'\).

Also, since
\begin{align*}
  Z(D) \simeq \{\alpha I \mid \alpha \in Z(D)\} &\simeq Z(R)
  \\&\simeq Z(R') \simeq Z(D')
\end{align*}
then, for \(k = Z(D)\),
\(\varphi\) is a \(k\)-module isomorphism.
(Note that since \(cM = (cI)M\)
for \(c \in k\) and matrix \(M\),
scalar multiplication can be considered as a matrix multiplication,
and \(\varphi\) is a ring homomorphism for matrices.
Thus, each \(R\) and \(R'\) can be considered as a \(k\)-module.)

Suppose that \(n' > n\).
Note that \(\calB = \{e_{i, j}\}\) is a basis of \(R = M_n(D)\)
as a \(k\)-module.
Then, isomorphic image of \(\calB\) by \(\varphi\)
should be a basis of \(R'\).
However, since \(|\calB| = n^2 < n'^2\).
Thus, it cannot be a basis of \(R'\).
In the same way, \(n' < n\) cannot hold.
Thus \(n = n'\).

Then, \(M_n(D) \simeq M_{n'}(D') = M_n(D')\).
Therefore \(D \simeq D'\).

This shows that \(n_k = n'_k\) and \(D_k = D'_k\) for each \(k\).
\qedsq
