\section*{Problem 1}

%-----------------------------------------------------------
\subsection*{P1(1)}
When \(R\) is a field,
prove that the projective dimensions of all \(R\)-modules are 0.

\subsection*{Answer for P1(1)}

Let \(M\) be an arbitrary \(R\)-module.
Let \(I \subseteq R\) be a left ideal and \(g: I \to M\) is a homomorphism.
Since \(R\) is a field, \(I = (0)\) or \(I = R\).
(By commutativity of \(R\), left ideal is two-sided,
and simple as a field.)
Suppose that \(I = (0)\).
Then, \(g(0) = 0\).
Then, we can extend \(g\) to \(\tilde g: R \to M\) such that \(\tilde g \equiv 0\).
Suppose that \(I = R\).
Then \(g\) is already a function \(R \to M\).
Then, just take a \(\tilde g = g\).
It's an extension of \(g\).
Therefore, for every left ideal \(I \subseteq R\),
any homomorphisms \(g: I \to M\) can be extended to the map \(R \to M\).
Thus, by Baer's Criterion, \(M\) is injective.
Therefore, every \(R\)-module is injective.
This shows that any field is semisimple.

Then, by Wedderburn's Theorem,
every \(R\)-module is projective.
(See Problem 2 \((2) \Rightarrow (1)\))

Let \(M\) be an arbitrary \(R\)-module.
Then, \(M\) is projective.
Take
\[0 \to M \xrightarrow{\Id_M} M \to 0\]
This is an exact sequence since \(\Id_M\) is a bijective homomorphism.
And since \(M\) is projective,
it's a projective resolution of \(M\).
Therefore, the projective dimensions of \(M\) is at most 0.
Since projective dimension is laid on \(\bbZ^{\ge 0} \cup \{\infty\}\),
\(pd_R M = 0\).
\qedsq

%-----------------------------------------------------------
\subsection*{P1(2)}
Let \(R\) be a PID.
Let \(M\) be a finitely generated \(R\)-module.
Prove that \(M\) always has a projective resolution
of length \(\le 1\).
(Hint: Theorem A)

\subsection*{Answer for P1(2)}

Let \(M\) be a finitely generaetd \(R\)-module.
Then, there is a generator \(\calB = \{b_1, \cdots, b_n\} \subseteq M\).
In other words, for any element \(m\) of \(M\),
there are \(r_1, \cdots, r_n \in R\) such that
\[m = \sum_{k=1}^n r_k b_k\]

Then, let \(\varphi: R^n \to M\) such that
\[\varphi(r_1, \cdots, r_n) = \sum_{k=1}^n r_k b_k\]
Then, it's well-defined and surjective, since \(\im \varphi\) takes
every possible linear combinations of the basis \(\calB\) of \(M\).

Note that \(\varphi\) is a homomorphism:
Let \(\bfr = (r_1, \cdots, r_n) \in R^n\),
\(\bfs = (s_1, \cdots, s_n) \in R^n\)
and \(a \in R\).
Then,
\begin{align*}
  \varphi(\bfr + \bfs)
  &= \sum_{k=1}^n (r_k + s_k) b_k
  \\&= \sum_{k=1}^n (r_k b_k + s_k b_k)
  \\&= \sum_{k=1}^n r_k b_k  + \sum_{k=1}^n s_k b_k
  = \varphi(\bfr) + \varphi(\bfs)
\end{align*}
\begin{align*}
  \varphi(a \bfr)
  &= \sum_{k=1}^n (a r_k)  b_k
  \\&= a \sum_{k=1}^n r_k  b_k
  = a \varphi(\bfr)
\end{align*}
This shows that \(\varphi\) is an homomorphism.

Let \(K = \ker \varphi\).
Then, we know two facts:
First, there is a canonical injective homomorhpism \(\iota\) from \(K\) to \(R^n\),
which is an identity for every element of \(K\).
In this case, \(\im \iota = K = \ker \varphi\).
Second, since \(R\) is PID, \(R^n\) is a free \(R\)-module of rank \(n\)
and \(K\) is a \(R\)-submodule of \(R^n\),
\(K\) is also free by the Theorem (A).

Then, we can construct an short exact sequence:
\[\begin{tikzcd}
  0 \ar{r} & K \ar["\iota"]{r} & R^n \ar["\varphi"]{r} & M \ar{r} & 0
\end{tikzcd}\]
because \(\iota\) is injective, \(\im \iota = K = \ker \varphi\), and \(\varphi\) is surjective.
Since \(K\) and \(R^n\) are free, they are projective.
Thus, \(\cdots \to 0 \to K \to R^n \to M\)
is a projective resolution of length 1.
Thus, the projective dimension of \(M\) should be less or equal to \(1\).
\qedsq

%-----------------------------------------------------------
\subsection*{P1(3)}
Give an example of a concrete PID \(R\)
and a finitely generated \(R\)-module
\(M\) such that \(pd_R M\) is precisely 1.

\subsection*{Answer for P1(3)}

Since every finitely generated \(R\)-module for PID \(R\) has a projective dimension \(0\) or \(1\),
we need to find a PID \(R\) and a finitely generated \(R\)-modlue
which does not have a projective resolution of length 0.

Let \(R\) be a PID and \(M\) be an arbitrary finitely generaetd \(R\)-module.
Suppose that \(pd_R M = 0\).
Then, there is a projective resolution
\[\cdots \to 0 \to N \xrightarrow{\epsilon} M \to 0\]
In this case, \(\epsilon\) must be injective and surjective.
Thus, \(\epsilon\) is an isomorphism between \(N\) and \(M\).
Since \(M\)'s isomorphic image, \(N\), is projective,
\(M\) should be projective too.

In other words, if \(M\) is not projective,
\(pd_R M > 0\).
Therefore, it's enough to find some non-projectiev \(R\)-module.

Let \(R = \bbZ\).
\(\bbZ\) is an ED thus a PID, but not a field since \(2^{-1} \in \bbQ \bs \bbZ\).
Let \(M = \bbZ_2 = \bbZ / 2\bbZ\).

\(\bbZ_2\) is not projective.
Because, if we take
\[\begin{tikzcd}
  \bbZ \ar["f"]{r} & \bbZ_2 \ar{r} & 0
  \\
  & \bbZ_2 \ar[dashed, "\tilde g"]{ul} \ar["g"]{u}
\end{tikzcd}\]
where \(f: \bbZ \to \bbZ_2\) is a canonical surjection such that \(x \mapsto x + 2\bbZ\),
and \(g\) is an identity map.
If \(\bbZ_2\) is projective, since \(f\) is surjective,
ther should be a lift \(\tilde g\) of \(g\)
such that \(g = f \circ \tilde g\).
Note that \(\tilde g\) must be a \(R\)-module homomorphism.
Because module homomorphism maps \(0\) to \(0\),
\(\tilde g(0) = 0\).
Also,
\[0 = \tilde g(0) = \tilde g(1 + 1) = \tilde g(1) + \tilde g(1)\]
However, for \(n \in \bbZ\), \(n + n = 0\) iff \(n = 0\).
Thus, \(\tilde g(1) = 0\).
This shows that there is only one homomorphism from \(\bbZ_2\) to \(\bbZ\),
which is a zero map.
Thus, \(\tilde g \equiv 0\).
However, in this case
\(0 = (f \circ \tilde g)(1) \neq g(1) = 1\).
Thus, there cannot be a lift of \(g\) by \(f\).

\(\bbZ_2\) is finitely generated.
More specifically, the basis of \(\bbZ_2\) is \(\{1\}\).
(\(\because\)
Since \(\bbZ_2\) is non-zero, the generating set of \(\bbZ_2\)
must contains at least one non-zero element.
Since \(0 = 0 \cdot 1\), \(1 = 1 \cdot 1\),
\(1\) generates every element of \(\bbZ_2\).)

Therefore,
the projective dimension of \(\bbZ_2\) as a \(\bbZ\)-module
must be greater than 0 since \(\bbZ_2\) is not projective,
and the projective dimension of \(\bbZ_2\) as a \(\bbZ\)-module
must be less or equal to 1 since \(\bbZ_2\) is a finitely generated module of PID.
Thus, the projection dimension of \(\bbZ_2\) as a \(\bbZ\)-module is 1.
\qedsq

Note that we can easily find a projective resolution of \(\bbZ_2\) of length 1,
which is,
\[\cdots \to 0 \to \bbZ \xrightarrow{f} \bbZ \xrightarrow{g} \bbZ_2 \to 0\]
where \(f: \bbZ \to \bbZ\) is a map such that \(x \mapsto 2x\)
and \(g: \bbZ \to \bbZ_2\) is a canonical surjection such that
\(x \mapsto x + 2\bbZ\).
In this case, \(f\) is injective, \(g\) is surjective,
\(\im f = 2\bbZ = \ker g\).
Thus, the above sequence is exact.
Also, since \(\bbZ\) is a free \(\bbZ\)-module of rank 1, it's projective.
Thus the above sequence gives a projective resolution of \(\bbZ_2\).

%-----------------------------------------------------------
\subsection*{P1(4)}
Let \(R = k[y_1, y_2]\).
Give an example of a finitely generated \(R\)-module \(M\),
such that \(pd_R M \ge 2\).

\subsection*{Answer for P1(4)}

For convinience let's change the indeterminates \(y_1\) and \(y_2\) to \(x\) and \(y\).
Since \(k\) is a field, \(k[x]\) is an ED and \(k[x, y]\) is a UFD.

\(M = k[x, y] / (x, y) \simeq k\).

First, I'll introduce a lemma:
\begin{definition}
  \(R\)-module \(M\) is torsion free,
  if for any non-zero element \(m\) of \(M\),
  \(rm = 0\) if and only if \(r = 0\) for \(r \in R\).
\end{definition}
\begin{lemma}
  Let \(R\) be an integral domain.
  Every flat \(R\)-module is torsion-free.
\end{lemma}
\begin{proof}
  Suppose that \(M\) is flat but not a torsion-free.
  Then there is a non-zero \(m \in M\) and non-zero \(r \in R\)
  such that \(rm = 0\).

  Note that \(r\) is a non-unit.
  (If not, \(m = r^{-1}rm = r^{-1} 0 = 0\) and it gives a contradiction.)
  This shows that
  if \(sm = 0\) for \(s \in R\)
  then \(s = 0\) or \(s\) is a non-unit.

  Let \(f: R \to R\) such that \(x \mapsto xr\).
  Then, \(f\) is injective, because
  if \(f(x) = f(y)\) for \(x, y \in R\),
  \(xr = yr\) and \(x = y\) by cancellation law.

  Take a exact sequence
  \[0 \to R \xrightarrow{f} R \to \coker f \to 0\]
  Because \(M\) is flat,
  \[0 \to R \otimes_R M \xrightarrow{f \otimes_R M} R \otimes_R M \to \coker f \otimes_R M \to 0\]
  is exact.
  It means, \(f \otimes_R M = f \otimes \Id_M\) is injective.

  Note that \(1 \otimes m\) is non-zero.
  (\(\because\)
  If \(1 \otimes m = 0\), there should be some \(s \in S^\times\)
  such that \(sm = 0\)
  so that \(1 \otimes m = s^{-1} \otimes sm = s^{-1} \otimes 0 = 0\).
  However, since \(s\) is a unit, \(sm\) cannot be zero.)

  However, \((f \otimes_R M)(1 \otimes m) = r \otimes m = 1 \otimes rm = 1 \otimes 0\).
  Thus, \(\ker (f \otimes_R M)\) contains a non-zero element \(1 \otimes m\).
  Therefore, \(f \otimes_R M\) cannot be injective
  and it's an contradiction.

  Therefore, if \(M\) is flat, it must be torsion-free.
\end{proof}

Since \(R\) is an ID and every projective \(R\)-module is flat,
projective \(R\)-module is torsion-free.

\(M = k[x, y] / (x, y)\) is not projective,
because it's not torsion-free.
(e.g. \(x \cdot \overline{1} = \overline{x} = 0\))
Therefore, \(pd_R M > 0\).
(Because \(pd_R M = 0\) then \(M\) should be projective.
See the Proof of (3).)
\br
\noindent
Let's show that there cannot be a projective resolution of length 1.
Suppose that there is a exact sequence with projective \(R\)-modules \(P, Q\):
\[0 \to P \to Q \xrightarrow{f} k[x, y] / (x, y) \to 0\]
where \(P\) is non-zero.
Then, because the injectivity of \(P \to Q\),
\(P \simeq \im (P \to Q) = \ker f\) and we obtain an exact sequence
\[0 \to \ker f \to Q \xrightarrow{f} k[x, y] / (x, y) \to 0\]

For each \(q \in Q\), if there is \(r \in Q\) such that
\(q = xr\),
then \(f(xr) = xf(r) = \overline{xr} = \overline{0}\).
Thus, \(q \in \ker f\).
In the same way, if there is \(r \in Q\) such that
\(q = yr\),
\(q \in \ker f\).
Thus, \((x, y) \cdot Q = \{pq \mid q \in Q, p \in (x, y)\} \subseteq \ker f\).

Since \(f\) is surjective, there is \(q \in Q\)
such that \(f(q) = \overline{1}\).
Note that since this \(q\) is not in \(\ker f\),
\(xq, yq \in \ker f\)
has a special property that there is no \(r \in \ker f\)
such that \(xr = xq\) or \(yr = yq\).
(It's because, if such \(r\) exists,
since \(R\) is UFD,
we can use Cancellation Law to obtain \(r = q\).
It makes a contradiction since \(r \in \ker f\) but \(q \not\in \ker f\).)

Then, make a exact sequence:
\[0 \to (x, y) \xrightarrow{g} k[x, y] = R \xrightarrow{h} k[x,y]/(x,y) \to 0\]
where \(g\) is an injection, and \(h\) is a canonical surjection.
Then, apply \(- \otimes_{R} \ker f\) functor.
\[0 \to
(x, y) \otimes_{R} \ker f \xrightarrow{g \otimes_{R} \ker f}
k[x, y] \otimes_{R} \ker f \xrightarrow{h \otimes_{R} \ker f}
k[x,y]/(x,y) \otimes_{R} \ker f \to 0\]
Since \(\ker f\) is projective, thus flat, the above sequence must exact.

Note that \(x \otimes (y \cdot q) \neq y \otimes (x \cdot q)\)
in \((x, y) \otimes_R \ker f\).
Because there are no other representation of \(x \otimes (yq)\).
In other words, for a single tensor product term,
there are only one equivalence relation such that
\(ar \otimes b = a \otimes rb\) for \(r \in R\).
However, for \(x \otimes (yq)\),
If \(x = ar\) where \(a \in (x, y)\), since the degree of \(a\) must be greater
than 0, \(\deg r\) must be 0.
Thus, in this case, only the element of \(k\) can be passed from left to right.
Also, since \(q\) is not in \(\ker f\),
polynomial with degree greater than 0 cannot be passed from right to left.
Thus, we cannot make any common terms between
\(x \otimes (y \cdot q)\) and \(y \otimes (x \cdot q)\).
And they cannot be equal.

However,
\begin{align*}
  (g \otimes_R \ker f)(x \otimes (y \cdot q) - y \otimes (x \cdot q))
  &= x \otimes (y \cdot q) - y \otimes (x \cdot q)
  \\&= 1 \otimes (xy \cdot q) - 1 \otimes (yx \cdot q)
  \\&= 1 \otimes (xy \cdot q) - 1 \otimes (xy \cdot q)
  \\&= 0 \otimes (xy \cdot q) = 0
\end{align*}
This shows that \(\ker (g \otimes_R \ker f)\)
contains a non-zero element \(x \otimes (y \cdot q) - y \otimes (x \cdot q)\)
Thus, \(g \otimes_R \ker f\) is not injective.
But it's a contradiction,
because \(g \otimes_R \ker f\) is injective by the flatness of \(\ker f\).

Therefore, \(\ker f\) cannot be projective,
and \(P\) cannot be projective too.

Therefore, the length of any projective resolution of \(k[x, y]/(x, y)\)
over \(k[x, y]\) must be at least \(2\) (or infinity).
\qedsq